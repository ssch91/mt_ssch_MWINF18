\section{Controllingansatz}
\subsection{Definitionsans�tze}
Die Diskussion der Definitionsans�tze des Controllings soll das Ziel der Arbeit an allgemein anerkannten Vorstellungen ausrichten und damit sicherstellen, dass die sp�tere Konzeption zu erwartenden Anspr�chen gen�gen kann.\\
Controlling ist als Wissenschaftsdisziplin in Deutschland seit 1973 etabliert, als der erste Lehrstuhl in Darmstadt mit Peter Horv\'ath besetzt wurde.\footnote{Vgl. \cite[S.16]{weber2005internationalisierung}}
Dessen Publikation ,,Controlling'', aktuell in 13. Auflage, pr�gt bis heute ma�geblich das Verst�ndnis des Controllings.\footnote{Google Scholar z.B. listet das Buch als das mit der deutlich h�chsten Anzahl Zitationen anderer Autoren, vgl. $https://scholar.google.com/scholar?hl=de\&as_sdt=0\%2C5\&q=controlling\&btnG=$, abgerufen am 14.01.2020.}
Eine allgemeing�ltige Definition des Controllings zu formulieren, bezeichnet er als schwierig\footnote{Vgl. \cite[S.13]{Horvath2015}}, da es internationale Unterschiede im Verst�ndnis der zugeordneten Aufgaben gibt\footnote{Vgl. \cite[S.23]{Horvath2015}} und Controlling im praktischen Vergleich stark unterschiedlich ausgelegt wird.\footnote{Vgl. \cite[S.9-14]{Horvath2015}}
Die Ansicht, dass Controlling allgemeing�ltig schwer zu definieren ist, hat zu der wissenschaftlichen Aufgabe der Controlling-Konzeption gef�hrt, die davon ausgeht, dass Controlling nicht ausschlie�lich induktiv oder deduktiv definiert werden kann.\footnote{Vgl. \cite[S.13]{ossadnik2009controlling}}.
Die Controllingkonzeptionen sind als normative Aussagensysteme zu verstehen, die eine Grundvorstellung ausdr�cken, welche in der Praxis zu finden und gleichzeitig theoretisch fundiert ist.\footnote{Vgl. \cite[S.13]{ossadnik2009controlling}}
Sie stellen Konglomerate von Controlling-Aufgaben in den Kontext des daraus f�r Unternehmen resultierenden Nutzens.\footnote{Vgl. \cite[S.7]{Hubert2018}}
Neben Horv\'aths diesbez�glichem Ansatz gelten die Ans�tze von K�pper et al. sowie Weber \& Sch�ffer als einflussreich.\footnote{Vgl. \cite[S.24,60]{Horvath2015} sowie \cite[S.8]{Hubert2018}}
\newline
Horv\'ath sieht Controlling als ein Subsystem des Managements, welches koordinierend f�r die Subsysteme der \gls{PK} und der \gls{IV} wirkt.\footnote{Vgl. \cite[S.47-48,60]{Horvath2015}}
\newline
K�ppers Definitionsansatz unterscheidet sich davon nur graduell.\footnote{Vgl. \cite[S.59]{Horvath2015}}
Er fasst das Controlling als als Koordination des gesamten F�hrungssystems mit dem Ziel der zielgerichteten Lenkung auf.\footnote{Vgl. \cite[S.27]{kuepper2013controlling}}
\newline
Dieses Ziel geben auch Weber \& Sch�ffer an, indem Sie Controlling als das Aufgabensystem zur Sicherung der Rationalit�t in der F�hrung wiedergeben.\footnote{Vgl. \cite[S.48]{weber2015einfuehrung}}
\newline
Abseits prozess- oder strukturorientierter Controlling-Konzeptionen sind in verbreiteter Literatur jedoch auch klassische Definitionsans�tze zu finden.
Eine dieser simpleren Definitionen findet sich z.B. bei W�he. 
Dieser fasst Controlling zusammen als ,,die Summe aller Ma�nahmen, die dazu dienen, die F�hrungsbereiche Planung, Kontrolle, Organisation, Personalf�hrung und Information so zu koordinieren, dass die Unternehmensziele optimal erreicht werden.''\footnote{\cite[S.176]{woehe2016einfuehrung}}


\subsection{Aufgaben und Ziele des Controllings}
Ausgehend von den f�nf durch W�he formulierten Aufgaben- bzw. F�hrungs-bereichen ist festzuhalten, dass Controllinginstrumente Koordination und Lenkung erm�glichen sollen.
Intention ist dabei immer, egal ob ein struktur- oder prozessorientierter Definitionsansatz geltend gemacht wird, dass die Instrumente unternehmerisches Handeln auf ein Ziel ausrichten und dabei rationalit�tssichernd wirken sollen, also  das Management in die Lage des objektiven und damit faktengest�tzten Entscheidens und Verhaltens versetzen sollen.
Hierbei stellt sich die Frage, wie das Controlling in der Praxis zu entwickeln ist.
Eine diesbez�glich g�ngige Unterscheidung liegt in der zeitlichen Ausrichtung\footnote{Vgl. \cite[S.109]{Horvath2015}}, bei der zwischen operativem\footnote{Vgl. \cite[S.109-110]{Horvath2015}, \cite[S-42-50]{Buchholz2013} und \cite[S.69-91]{Schroeter2002}} und strategischem\footnote{Vgl. \cite[S.109-118]{Horvath2015}, \cite[S.42-58]{Buchholz2013} sowie \cite[S.91]{Reichmann2017}, wobei letzterer das strategische Controlling weniger �ber seine zeitliche Ausrichtung definiert, sondern es als Teilbereich auf Basis seiner Inhalte von anderen Controlling-Disziplinen wie dem Produktionscontrolling abgrenzt.} Controlling unterschieden wird.
%----Tabelle Anfang
\begin{table}
\setlength\extrarowheight{1pt} %etwas mehr Luft
\captionsetup{justification=centering} %Caption zentrieren
\centering
\begin{tabularx}{\textwidth}{|C|C|C|}
\hline
\backslashbox{\textbf{Merkmale}}{\textbf{C.-Typen}}  & {\textbf{Strategisches Controlling}}           & {\textbf{Operatives Controlling}}                                  \\ \hline
Orientierung                                          & Umwelt und Unternehmung: Adaption   & Unternehmung: Wirtschaftlichkeit betrieblicher Prozess  \\ \hline
Planungsstufe                                         & Strategische Planung                & Taktische und operative Planung, Budgetierung           \\ \hline
Dimensionen                                           & Chancen/Risiken, St�rken/Schw�chen  & Aufwand/Ertrag, Kosten/Leistungen                       \\ \hline
Zielgr��en                                            & Existenzsicherung, Erfolgspotential & Wirtschaftlichkeit, Gewinn, Rentabilit�t   \\ \hline           
\end{tabularx}
\caption[Controlling-Parameter nach Horv\'ath]{Controlling-Parameter nach Horv\'ath\footnotemark.}\label{tab:1}
\end{table}
\footnotetext{\cite[S.109]{Horvath2015}}
%----Tabelle Ende
               
\subsection{Controllingbereiche}
\subsubsection{Finanzcontrolling}
\subsubsection{Beschaffungscontrolling}
\subsubsection{Produktionscontrolling}
\subsubsection{Logistikcontrolling}
\subsubsection{Projektcontrolling}
\subsection{Kennzahlensysteme}
\subsubsection{Du-Pont-Kennzahlensystem}
\subsubsection{Diebold-Kennzahlensystem}
\subsubsection{SVD-Kennzahlensystem}
\subsubsection{Balanced Score Card}
\subsubsection{Statuskonzept von K\"utz}
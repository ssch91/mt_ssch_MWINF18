\section{Produktionscontrollings}
\subsection{Definition}
Nachdem in \ref{Produktionscontrolling} das \gls{Pr-C} aufgrund seiner zu Flexibilit�t einschl�gigen Inhalte als aussichtsreiches Portfolio identifiziert wurde, ist es erforderlich, das \gls{Pr-C} umfangreich zu erfassen, die Methoden und Techniken zu strukturieren und auf Einschl�gigkeit zu Flexibilit�t zu pr�fen und schlie�lich eine Auswahl von in der Konzeption einzuschlie�enden bzw. zu �bertragenden Elementen zu formulieren. 
Grunds�tzlich ist das \gls{Pr-C} die Disziplin bzw. betriebliche T�tigkeit, die dazu dient, die Anspr�che des Controllings in der Produktion zu platzieren und umzusetzen.\footnote{Vgl. \cite[S.20]{Gottmann2019}}
Die Produktion hat dabei die Aufgabe, Wertsteigerung von Produkten zu erwirken, indem ein Input einem Output gegen�bergestellt wird.\footnote{Vgl. \cite[S.20]{Gottmann2019}}
Dabei handelt es sich neben direktem Input in Form von Produktionsanlagen, Material und Arbeitsleistung auch um indirekten Input wie die Organisation, Planung und Steuerung.\footnote{Vgl. \cite[S.19]{Gottmann2019}}
Das Controlling, dessen Ziel wiederum die ergebnisorientierte Planung und Steuerung von Ma�nahmen durch Beschaffung, Aufbereitung, Analyse und Kommunikation von Daten ist\footnote{Vgl. \cite[S.20]{Gottmann2019}}, muss also in den entscheidenden Parametern auf die Produktion und die kaufm�nnischen Zielsetzungen ausgerichtet werden\footnote{Vgl. \cite[S.20]{Gottmann2019}} und letztlich einen effizienten und erfolgreichen Betrieb sicherstellen\footnote{Vgl. \cite[S.20]{Gottmann2019}}, eine ganzheitliche Optimierung von Investitionsentscheidungen zu erm�glichen\footnote{Vgl. \cite[S.20]{Gottmann2019}} und vor allem Kompromisse zwischen bei den kaufm�nnischen und produktionsrelevanten Zielsetzungen\footnote{Vgl. \cite[S.20]{Gottmann2019}, \cite{schnell2018produktion} und \cite[S.24-26]{klein2012controlling1} sowie} zu finden. 
Dahingehend ist es also Aufgabe des \gls{Pr-C}, Produktions- und Controllingziele zu verbinden\footnote{Vgl. \cite[S.21]{Gottmann2019}} , den angesprochenen Input und Output zu optimieren\footnote{Vgl. \cite[S.21]{Gottmann2019}} und daf�r die richtigen Instrumente ausw�hlen, zu implementieren und einzusetzen.\\


Das \gls{Pr-C} differenziert in seinen T�tigkeiten die zeitlichen Dimensionen grunds�tzlich.
Je nach Interpretation wird lediglich zwischen strategischem und taktisch-operativen \gls{Pr-C} unterschieden\footnote{Vgl. \cite[S.25]{klein2012controlling1}}, w�hrend andere auch letzteres als unterschiedliche Dimensionen auslegen.\footnote{Vgl. \cite[S.9]{Gottmann2019}}
Letztlich ist die Controlling-Konzeption dabei aufgrund der Mangementunterst�tzung immer am Management-System auszurichten.
Auch hierbei ist eine Unterscheidung nach strategischem\footnote{Vgl. \cite[S.20]{zaepfel2014strategisches}}, taktischem\footnote{Vgl. \cite[S.20]{zaepfel2010taktisches}} und operativem\footnote{Vgl. \cite[S.20]{zaepfel1982operatives}} Produktionsmanagement m�glich.
Eine m�gliche Auslegung ist z.B., in der strategischen Perspektive langfristige Ziele innerhalb des Marktes zu betrachten, in der taktischen das Produktionsprogramm in Breite und Tiefe zu fokussieren und in der operativen die laufenden Fertigungsauftr�ge.\footnote{Vgl. \cite[S.4]{zaepfel2010taktisches}} 
\subsection{Betrachtungsgegenst�nde}
\subsection{Teilbereiche}
\subsubsection{Strategisches Produktionscontrolling}
\subsubsection{Taktisches Produktionscontrolling}
\subsubsection{Operatives Produktionscontrolling}
\subsection{Methoden und Techniken}
\subsubsection{Strategische Instrumente}
\paragraph{Produktlebenszyklus-Analyse}
\paragraph{Balanced Scorecard}
\subsubsection{Operative Instrumente}
\paragraph{Kennzahlen}
\paragraph{Kennzahlensysteme}
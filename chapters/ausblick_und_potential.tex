\section{Fazit und Potential}
Mit Abschluss der Konzeption kann nun die Zielerreichung �berpr�ft, die Ergebnisqualit�t und -bedeutung diskutiert sowie ein Blick auf darauf aufbauende M�glichkeiten geworfen werden.\\\\
Das Ziel, Flexibilit�t in der IT bewertbar und steuerbar zu machen, ist insofern erreicht, als dass das konzeptionelle Ergebnis die in \ref{sec:aufgabenundziele} aufgestellten Ergebniskriterien erf�llt:\\
Das in Form einer \gls{BSC} entworfene Resulat zentriert Informationen zur Steuerung von Ma�nahmen unterschiedlichen zeitlichen Horizonts.
Die Ma�nahmen, die die dabei die Flexibilisierung interner Verh�ltnisse anvisieren, um auf eine dynamische Umwelt reagieren zu k�nnen, entsprechen insoweit der Controlling-Konzeption von K�pper et al. und Horv\'ath et al, als dass deren ebenda dargestellte Anspr�che ber�cksichtigt sind.\\
Die modellierten Verbindungen zu gesamtunternehmerischen Zielen der Kunden- und Finanzperspektiven lassen letztendlich sogar die Vermutung einer ganzheitlich m�glichen Steuerung zu.
Die Flexibilisierbarkeit entscheidender IT-Inhalte und deren Tragweite indizieren, dass eine konsequente Flexibilisierung - deren Ideen-Ursprung wie anfangs indiziert in dynamischen Rahmenbedingungen liegt - essentielle Bestandteile der Betrachtung einer IT-Strategie abdecken kann.\\
\\
Ein diesbez�glicher empirischer Beweis zur Validierung des Modells steht allerdings aus, weshalb nur Analogien zu Erkenntnissen �hnlicher Forschung ermittelt werden k�nnen. 
Das Modell, dass allerdings unternehmens�bergreifende Vergleichbarkeit von Flexibilit�t und deren Auswirkungen erm�glicht, ist dahingehend modellimmanent validierbar.
In dieser Hinsicht ist das Modell auch bisherigen Flexibilit�tsuntersuchungen in der IT auch insoweit voraus, als dass z.B. die nicht eindeutig messbaren, zumal nicht eindeutig definierten, gesamtsystemischen Indikatoren von Byrd/Turner (Integration, Konnektivit�t, Modularit�t)\footnote{Vgl. \cite{byrdturner2000}.} nicht unternehmens�bergreifend und nicht im Detail, d.h. je Ma�nahme, sondern nur im in Verbindung validierbar sind.
Nichtsdestotrotz l�sst die Einstufung der Verbindung zum Unternehmenserfolg, die in qualitativer Erhebung damit erbracht wurde\footnote{Vgl. \cite{Tallon20030UF}.}, den Schluss zu, dass der ganzheitliche Steuerungsansatz, der in dieser Arbeit konzipiert ist, einen validen Zugang darstellt und die durch Flexibilit�t angestrebten Ziele die vermutete Tragweise besitzen k�nnen.
\\
\\
Nichtsdestotrotz bleiben Gestaltungsm�glichkeiten des Modells bestehen.
Die Flexibilit�tsans�tze aus dem \gls{L-C} z.B., die mit denen des \gls{P-C} in Verbindung stehen (vgl. \ref{bsc_quellen}), weisen ggf. auch Adaptionspotential auf und k�nnen andere Messungsans�tze liefern.\\\\
F�r die Flexibilit�t der IT-Organisation ist letztlich zu konstatieren, dass die Betrachtung weder in der Breite noch der Tiefe stattgefunden hat wie f�r produktionswirtschaftliche Flexibilit�t, die trotzdem noch nicht zu allgemeinem Konsens gelangt ist.
Auch f�r die IT ist daher in Anbetracht der f�r sie ebenfalls geltenden dynamischen Umweltbedingungen anzunehmen, dass die Diskussion um Flexibilit�t �hnliche Ausma�e annehmen kann und vielleicht auch sollte.\\\\
Diese Arbeit leistet dazu einen konzeptionellen Beitrag.
%Multiprojektcontrolling einbeziehen
%auf logistikcontrolling eingehen

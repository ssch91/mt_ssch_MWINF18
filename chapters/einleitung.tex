\section{Einleitung}
\subsection{Motivation und Zielsetzung}
Steigende Durchdringung unternehmerischen Umfelds durch informationstechnologische Systeme und die damit einhergehende steigende Gr\"o{\ss}e von IT-Organisationen, die unterst\"utzend oder direkt wertsch\"opfend die IT-Services zur Verf\"ugung stellen, zwingen IT-Verantwortliche, M\"oglichkeiten zur objektiven und zielgerichteten Steuerung der Gesamt-IT-Organisation zu etablieren.\newline
Der Ansatz des Controllings, zentrale Aufgaben des Managements mittels dementsprechender Methoden aufeinander abzustimmen, soda{\ss} bestm\"ogliche Rahmenbedingungen zur unternehmerischen Zielerreichung geschaffen werden, ist lange etabliert.\footnote{Vgl. z.B. \cite[S.176f]{woehe2016einfuehrung} sowie \cite[S.25]{Horvath2015} und \cite[S.33ff]{kuepper2013controlling}, au{\ss}erdem \cite[S.20ff]{weber2015einfuehrung} zu anderen Definitionsans\"atzen}\newline
Der Einsatz von Information{\ss}ystemen war fr\"uher prim\"ar technisch orientiert.\footnote{Vgl. \cite[S.VII]{Gadatsch2014}}\newline
Seit etwa 1990 verdichtet sich bei IT-Verantwortlichen allerdings die Ansicht, da{\ss} diese Systeme als Produktionsfaktor mit dem Controlling-Ansatz zu vernetzen sind.\footnote{Vgl. \cite[S.VII]{Gadatsch2014}}
Viele Elemente des kla{\ss}ischen Finanzcontrollings oder anderer Teilbereiche, wie z.B. die Balanced Scorecard, sind auch im IT-Controlling bereits gel\"aufig und k\"onnen anhand bestehender Methoden darauf ausgerichtet werden.\footnote{Vgl. \cite[S.46]{kesten2013}}\newline
Die Rolle der IT-Organisation in einem Unternehmen kann verschieden ausgelegt werden, da die in der Praxis vorzufindenden Konstrukte durch die M\"oglichkeiten externer Dienstleister sowie Technologieanbieter (z.B. Cloud-Dienste) Schwerpunkte setzen m\"u{\ss}en.\footnote{Vgl. \cite[S.581f]{schroederHMD2016}}\newline
In der Folge wird h\"aufig nicht die Gesamtheit einer theoretisch durch eine IT-Abteilung abdeckbaren T\"atigkeiten tats\"achlich erbracht , sondern basierend auf inneren und \"au{\ss}eren Einfl\"u{\ss}en Verantwortlichkeitsverteilung vorgenommen.\footnote{Vgl. \cite[S.585-590]{schroederHMD2016}}\newline
Die in diesem Kontext notwendige Flexibilit\"at, die dazu dienen kann, mit IT-Organisationen auf z.B. organisatorische Ver\"anderungen oder technologische Schwierigkeiten zu reagieren, um sie trotz kontinuierlich komplexer werdenden Umfelds zielsicher steuern zu k\"onnen und innerhalb dieser Rahmenbedingungen \"okonomisch bestm\"ogliche Verh\"altni{\ss}e zu erreichen, ist bisher nicht Bestandteil einer integrierten Betrachtung des IT-Controllings.\newline
Auch dedizierte bzw. isolierte Untersuchungen zu Flexibilit\"atsaspekten existieren nur wenig und veraltet\footnote{Vgl. z.B. die fast 20 Jahre alten Beitr\"age \cite[S.168ff]{byrdturner2000} und \cite[S.21ff]{byrdturner2001}}, ber\"ucksichtigen also nicht die aktuell vorherrschenden Zust\"ande.
Diese f\"ur die IT ausgebliebene Betrachtung von Flexibilit\"at ist allerdings fester Bestandteil des Produktionscontrollings und und dort wird sie auch als konkreter Wertbeitrag verstanden.\footnote{Vgl. \cite[S.8f]{Gottmann2019}}
Angesichts beschriebener Umst\"ande, auf die auch produzierendes Gewerbe (im Sinne der produzierenden Abteilungen) reagieren m\"u{\ss}en, ist Flexibilit\"at als wertsch\"opfender Aspekt auch in informationstechnologischer Hinsicht wahrscheinlich.
Diesen zu definieren, in Anlehnung an andere Teilbereiche des Controllings me{\ss}bar zu machen und zu interpretieren ist Ziel und Bestandteil dieser Arbeit.

\subsection{Forschungsrelevanz}
Das Feld der unternehmerisch genutzten Informationstechnologie ist dynamisch und kurzweilig - ein Charakteristikum, de{\ss}en Auspr\"agung sich bis heute versch\"arft.\footnote{Vgl. \cite[S.15]{capgemini2019}} 
Daher ist nicht verwunderlich, da{\ss} nationale und internationale Studien unabh\"angig voneinander immer wieder darauf hindeuten, dass IT-Projekte scheitern oder zumindest nicht erwartungskonform verlaufen.\footnote{Vgl. \cite[S.172]{fischer2016}}\newline
Ein zu verzeichnender Trend ist zum Beispiel, da{\ss} Projektmanagement-Methoden tendenziell h\"aufiger agil als plangetrieben ausgelegt werden\footnote{Vgl. \cite[S.12]{statusquoagile2015}} und dadurch subjektiv be{\ss}ere Resultate erzielt werden.\footnote{Vgl. \cite[S.22]{statusquoagile2015}} 
Es l\"a{\ss}t sich f\"ur Projekte ein Flexibilisierungstrend erkennen.\newline
Was bedeutet Flexibilit\"at nun aber f\"ur die Gesamtauslegung der IT-Organisation?
Potentiellen Erwartungen steht gegen\"uber, da{\ss} dedizierte Auseinandersetzung bis vor zehn Jahren weder wi{\ss}enschaftlich noch praktisch stattfand.\footnote{Vgl. \cite[S.53]{radermacher2009}} 
Nichtsdestotrotz erkannten bereits 2008 - also in laut einer Studie der Capgemini Unternehmensvertreter, da{\ss} IT-Flexibilisierung als ``Megatrend'' einzustufen ist und Grund f\"ur ``fundamentale Transformationsprozesse'' sein wird.\footnote{Vgl. \cite[S.17]{capgemini2008}}
Ratzer fa{\ss}t die Relevanz von Flexibilit\"at wie folgt zusammen: ``Um diese Situation be{\ss}er kontrollieren zu k\"onnen, wird im Gegenzug eine noch weiter entwickelte IT ben\"otigt, die wiederum erneut den Komplexit\"ats- und Unsicherheitsgrad des Wettbewerbsumfelds erh\"oht. 
Dieser Mechanismus voll- zieht sich in immer k\"urzeren Ver\"anderungszyklen, denen sich IT-Organisationen anpa{\ss}en m\"u{\ss}en. 
Eine deutliche h\"ohere Flexibilit\"at ist n\"otig.''\footnote{\cite{ratzer2009}}
Auch Wiedenhofer sieht in der Dynamik die Notwendigkeit f\"ur Flexibilit\"at gegeben, um damit auf auftretende Probleme zu reagieren: ``Durch die Schaffung von geeigneten Strukturen steigert die IT-Organisation ihre Handlungsflexibilit\"at. 
Mit dieser F\"ahigkeit kann sie schnell auf wechselnde und komplexe Anforderungen reagieren.''\footnote{\cite[S.236]{Wiedenhofer2016}}
Er sieht in k\"urzeren Innovationszyklen, steigender Digitalisierung und der Geschwindigkeit des konjunkturellen Wandels insbesondere eine Bedrohung f\"ur bestehende Gesch\"aftsmodelle\footnote{Vgl. \cite[S.237]{Wiedenhofer2016}}, auf die mit Flexibilit\"at zu reagieren ist.\newline
Zwar ist die Dynamik- bzw. Komplexit\"atsfloskel eine repetitiv paraphrasierte Scheinbegr\"undung, doch ist zu ermitteln, da{\ss} sich die Kontextualisierung der Forderung nach Flexibilit\"at mit dieser Art als problematisch eingestuften Rahmenbedingungen selbst in wi{\ss}enschaftlichen Beitr\"agen bis heute erhalten hat, soda{\ss} diesbez\"ugliche Relevanz tats\"achlich im Zusammenspiel beider Seiten zu begr\"unden ist. 
Tats\"achlich ist die Relevanz hinsichtlich praktischer Forschung weiter auch damit zu begr\"unden, da{\ss} die Behandlung zwar in der Fachwelt erfolgt, konkrete, konsensf\"ahige Beurteilungsmethoden und Handlungsvorschl\"age, z.B. auf Basis von Szenarioeinordnungen aber nicht ihren Weg in einschl\"agige Publikationen (z.B. Gadatsch, Mayer oder Tiemeyer) gefunden haben.


\subsection{Methodisches Vorgehen}
%Analogiemethode
Ziel der Arbeit ist, wie in Kapitel 3 angesprochen, Me{\ss}barkeit von Flexibilit\"at zu untersuchen und ein Rahmenwerk zu definieren, welches Methoden aus dem Produktionscontrolling ableitet und zu eruierenden Zielen und Zwecken zuf\"uhrt, welche wiederum aus allgemeinen Anspru\"uchen des Controllings abzuleiten sind. Auf diesem Weg soll Flexibilita\"at als Wertreiber greifbar und verst\"andlich werden, also auch verdeutlicht werden, welcher Nutzen aus flexiblen IT-Architekturen gezogen werden kann.
Ziel ist allerdings nicht, Flexibilit\"at an konkreten Beispielen zu me{\ss}en und den Wertsch\"pfungsbeitrag zu analysieren.
Grundlage der Forschung ist daher die theoretische, also auf Literatur gest\"utzte Erarbeitung von Grundlagen und Zielen des Controllings, Implementationsweisen und Zielen im Produktionscontrolling, werttreibenden Aspekten unternehmerischer IT, Auswirkungen von ausreichender und mangelnder Flexibilita\"at, Aufbau von Rahmenwerken des IT-Controllings und letztlich die integrierte Konsolidierung in einem Rahmenwerk zur Me{\ss}ung f\"ur das IT-Controlling. 
Dieses Vorhaben hat deduktiven Charakter, wobei allerdings nicht vom ``Allgemeinen auf einen besonderen Einzelfall''\footnote{\cite[S.37]{Sandberg2017}} zu schlie{\ss}en ist, sondern Gesetzm\"a{\ss}igkeiten \"ubertragen werden. 
Insbesondere die Rahmenbedingungen unterliegen hierbei der Notwendigkeit besonders differenzierter Betrachtung.\footnote{\cite[S.37-39]{Sandberg2017}}













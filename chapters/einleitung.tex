\section{Einleitung}
\subsection{Motivation und Zielsetzung}\label{zielsetzung}
Steigende Durchdringung unternehmerischen Umfelds durch informationstechnologische Systeme und die damit einhergehende steigende Gr\"o{\ss}e von IT-Organisationen, die die IT-Dienste f�r Unternehmen zur Verf�gung stellen, zwingen Verantwortliche, M\"oglichkeiten zur objektiven und zielgerichteten Steuerung der Gesamt-IT-Organisation zu etablieren.\newline
Daher bedarf es eines Ansatzes, zentrale Aufgaben des IT-Managements mittels dementsprechender Methoden aufeinander abzustimmen, sodass bestm\"ogliche Rahmenbedingungen zur unternehmerischen Zielerreichung geschaffen werden.
In Form des Controllings existiert ein Ansatz des allgemeinen Managements bereits in langfristig praxiserprobter Form.\footnote{Vgl. z.B. \cite[S.176f]{woehe2016einfuehrung};\\ \cite[S.25]{Horvath2015} und\\ \cite[S.33ff]{kuepper2013controlling}; au�erdem\\ \cite[S.20ff]{weber2015einfuehrung} zu anderen Definitionsans\"atzen.}\\\\
Der Einsatz von Informationssystemen war fr\"uher prim\"ar technisch orientiert.\footnote{Vgl. \cite[S.VII]{Gadatsch2014}.}
Seit etwa 1990 verdichtet sich bei IT-Verantwortlichen allerdings die Ansicht, dass diese Systeme als Produktionsfaktor mit dem Controlling-Ansatz zu vernetzen sind.\footnote{Vgl. \cite[S.VII]{Gadatsch2014}.}
Viele Elemente des klassischen Finanzcontrollings oder anderer Teilbereiche, wie z.B. die Balanced Scorecard, sind auch im IT-Controlling bereits gel\"aufig und k\"onnen anhand bestehender Methoden darauf ausgerichtet werden.\footnote{Vgl. \cite[S.46]{kesten2013}.}\\\\
Die Rolle der IT-Organisation in einem Unternehmen kann verschieden ausgelegt werden, da bei den in der Praxis vorzufindenden Konstrukten durch die M\"oglichkeiten externer Dienstleister sowie Technologieanbieter (z.B. Cloud-Dienste) Schwerpunkte zu setzen sind, um optimale Gesamtfunktionalit�t zu erreichen.\footnote{Vgl. \cite[S.581f]{schroederHMD2016}.}\\\\
In der Folge wird h\"aufig nicht die Gesamtheit einer theoretisch durch eine IT-Organisation abdeckbaren T\"atigkeiten tats\"achlich erbracht, sondern basierend auf inneren und �u�eren Einfl\"ussen eine Verantwortlichkeitsverteilung vorgenommen.\footnote{Vgl. \cite[S.585-590]{schroederHMD2016}.}\\\\
Die in diesem Kontext notwendige Flexibilit\"at, die dazu dienen kann, mit IT-Organisationen auf z.B. organisatorische Ver\"anderungen oder technologische Schwierigkeiten zu reagieren, um sie trotz kontinuierlich komplexer werdenden Umfelds zielsicher steuern zu k\"onnen und innerhalb dieser Rahmenbedingungen \"okonomisch bestm\"ogliche Verh\"altnisse zu erreichen, ist bisher nicht Bestandteil einer integrierten Betrachtung des IT-Controllings.
Auch dedizierte bzw. isolierte Untersuchungen zu Flexibilit\"atsaspekten existieren nur wenige und veraltete\footnote{Vgl. z.B. die fast 20 Jahre alten Beitr\"age von \cite[S.168ff]{byrdturner2000} und\\ \cite[S.21ff]{byrdturner2001}.}, ber\"ucksichtigen also nicht die aktuell vorherrschenden Zust\"ande.\\\\
Diese f\"ur die IT ausgebliebene Betrachtung von Flexibilit\"at durch die �bertragung bzw. Adaption von Methoden aus anderen fachbereichsbezogenen Controllingdisziplinen nachzuholen, scheint daher ein naheliegendes Verfahren zu sein, um auch in der IT ein Wertbeitragsverst�ndnis f�r Flexibilit�t entwickeln zu k�nnen.
Inwiefern die Flexibilit�t einen solchen Wertbeitrag leistet und zur Erreichung unternehmerischer Ziele beitr�gt, soll also dar�ber ermittelt werden, welche Ans�tze und Bewertungen in anderen Controllingbereichen vorliegen.
Flexibilit�t im Kontext der IT-Organisation zu definieren, in Anlehnung an andere Teilbereiche des Controllings messbar zu machen und zu interpretieren, ist Ziel und Bestandteil dieser Arbeit.
\subsection{Forschungsrelevanz}\label{Relevanz}
Das Feld der unternehmerisch genutzten Informationstechnologie ist dynamisch und kurzweilig - ein Charakteristikum, dessen Auspr\"agung sich bis heute versch\"arft.\footnote{Vgl. \cite[S.15]{capgemini2019}.} 
Daher ist nicht verwunderlich, dass nationale und internationale Studien unabh\"angig voneinander immer wieder darauf hindeuten, dass z.B. IT-Projekte scheitern oder zumindest nicht erwartungskonform verlaufen.\footnote{Vgl. \cite[S.172]{fischer2016}.}
Ein zu verzeichnender Trend ist dahingehend, dass Projektmanagement-Methoden tendenziell h\"aufiger agil als plangetrieben ausgelegt werden\footnote{Vgl. \cite[S.12]{statusquoagile2015}.} und dadurch subjektiv bessere Resultate erzielt werden.\footnote{Vgl. \cite[S.22]{statusquoagile2015}.} 
F�r Projekte deutet sich also ein Flexibilisierungstrend an.\newline 
Was bedeutet Flexibilit\"at nun aber f\"ur die Gesamtauslegung der IT-Organisation?\\\\
Potentiellen Erwartungen steht gegen\"uber, dass eine dedizierte Auseinandersetzung wissenschaftlich und praktisch weitgehend ausgeblieben ist.\footnote{Vgl. \cite[S.53]{radermacher2009}.} 
Nichtsdestotrotz stuften bereits 2008 Unternehmensvertreter in einer Studie der Capgemini IT-Flexibilisierung als ,,Megatrend'' ein  und nannten sie als potentiellen Ausl�ser f\"ur ,,fundamentale Transformationsprozesse''\footnote{\cite[S.17]{capgemini2008}.}.
Ratzer fasst die Relevanz von Flexibilit\"at wie folgt zusammen: ,,Um diese Situation besser kontrollieren zu k\"onnen, wird im Gegenzug eine noch weiter entwickelte IT ben\"otigt, die wiederum erneut den Komplexit\"ats- und Unsicherheitsgrad des Wettbewerbsumfelds erh\"oht. 
Dieser Mechanismus vollzieht sich in immer k\"urzeren Ver\"anderungszyklen, denen sich IT-Organisationen anpassen m\"ussen. 
Eine deutliche[sic!] h\"ohere Flexibilit\"at ist n\"otig.''\footnote{\cite{ratzer2009}.}
Auch Wiedenhofer sieht in der Dynamik die Notwendigkeit f\"ur Flexibilit\"at gegeben, um damit auf auftretende Probleme zu reagieren: ,,Durch die Schaffung von geeigneten Strukturen steigert die IT-Organisation ihre Handlungsflexibilit\"at. 
Mit dieser F\"ahigkeit kann sie schnell auf wechselnde und komplexe Anforderungen reagieren.''\footnote{\cite[S.236]{Wiedenhofer2016}.}
Er sieht in k\"urzeren Innovationszyklen, steigender Digitalisierung und der Geschwindigkeit des konjunkturellen Wandels insbesondere eine Bedrohung f\"ur bestehende Gesch\"aftsmodelle\footnote{Vgl. \cite[S.237]{Wiedenhofer2016}.}, auf die mit Flexibilit\"at zu reagieren ist.\\\\
Zwar ist die Dynamik- bzw. Komplexit\"atsfloskel den Ruf einer unabl�ssig wiederholten Scheinbegr\"undung, doch ist zu ermitteln, dass sich die Kontextualisierung der Forderung nach Flexibilit\"at mit dieser Art als problematisch eingestuften Rahmenbedingungen auch in wissenschaftlichen Beitr\"agen bis heute erhalten hat, sodass diesbez\"ugliche Relevanz tats\"achlich im Zusammenspiel beider Faktoren zu begr\"unden ist. 
In der Tat ist die Relevanz hinsichtlich praktischer Forschung weiter auch damit zu begr\"unden, dass die Behandlung zwar wie indiziert in der Fachwelt erfolgt, aber konkrete, konsensf\"ahige Beurteilungsmethoden und Handlungsvorschl\"age, z.B. auf Basis von Szenarioeinordnungen nicht zustande gekommen sind.

\subsection{Methodisches Vorgehen}\label{Methodisches Vorgehen}
%Analogiemethode
Ziel der Arbeit ist, wie in \ref{zielsetzung} angesprochen, ein Wertbeitragsverst�ndnis f�r Flexibilit�t zu entwickeln, indem darin enthaltene Aspekte in �bertragbarer Methodik f�r die IT operationalisiert werden.
Dazu geh�rt neben einer grundlegenden Definition auch die Eruierung von durch Flexibilit�t in der IT zu schaffenden M�glichkeiten und damit korrespondierenden Zielen.\\
Ziel ist dagegen nicht, Flexibilit\"at an konkreten Beispielen zu messen und den Wertsch\"opfungsbeitrag an Realobjekten zu analysieren.\\\\
Das beschriebene Vorhaben soll in drei aufeinander aufbauenden Schritten verfolgt werden.\\
Zun�chst soll explorativ ermittelt werden, welche Anforderungen seitens des �blichen Verst�ndnisses von Controlling-Instrumenten an das konzeptionelle Ziel bestehen und welche Controllingdisziplinen adaptionsf�hige TheorieiMnhalte hervorgebracht haben.\\
Anschlie�end sollen in hermeneutisch-interpretativer Vorgehensweise diese Inhalte untersucht werden, um festzustellen, wie Flexibilit�t im jeweiligen Fachbereich gemessen und interpretiert wird und um zu pr�fen, inwieweit die Methoden �bertragungsf�hig sind.\\
Abschlie�end besteht der konstruktive Teil der Arbeit darin, Ans�tze f�r Flexibilit�t in der IT zu identifizieren und dort Messmethoden anzusetzen.
Dazu sollen Methoden �bertragen, soweit notwendig erg�nzt und in einem System f�r die IT konsolidiert werden.
%, Methoden zu �bertragen, soweit notwendig zu erg�nzen und in einem System f�r die IT zu konsolidieren.
%Grundlage der Forschung ist daher die theoretische, also auf Literatur gest\"utzte Erarbeitung von Grundlagen und Zielen des Controllings, Implmentation von darin enthaltenen Instrumenten zur Bewertung von Flexibilit�t, wiederum deren werttreibernder Aspekt und letztlich die integrierte Konsolidierung in einem Rahmenwerk zur Messung f\"ur das IT-Controlling.\\ 
%Dieses Vorhaben hat deduktiven Charakter, wobei allerdings nicht vom ,,Allgemeinen auf einen besonderen Einzelfall''\footnote{\cite[S.37]{Sandberg2017}} zu schlie�en ist, sondern Gesetzm\"assigkeiten \"ubertragen werden. 
%Insbesondere die Rahmenbedingungen unterliegen hierbei der Notwendigkeit besonders differenzierter Betrachtung hinsichtlich der Vergleichbarkeit.\footnote{\cite[S.37-39]{Sandberg2017}}













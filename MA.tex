  \documentclass[12pt]{report}

\usepackage[latin1]{inputenc}
%zur Benutzung von Grafiken
\usepackage{graphicx}
%f�r Kapitelformatierung
\usepackage{titlesec}
%f�r Zeilenabstand
\usepackage{setspace}
%f�r Deutsche Bezeichungen wie "Tabellenverzeichnis"
\usepackage[ngerman]{babel}
%Seitenlayout
\usepackage[a4paper,top=20mm,bottom=20mm,left=30mm,right=40mm]{geometry}
%Bib-Ressourcen erm�glichen, Stil setzen und genau parametrieren
\usepackage[style=footnote-dw,natbib=true,edsuper=true,nopublisher=false,doi=true,isbn=true,urldate=long,backend=biber]{biblatex}
%F�r das Abk�rzungsverzeichnis alternativ: \usepackage[toc]{glossaries}
\usepackage[acronym,nonumberlist,automake]{glossaries}
%f�r diagonal geteilte Zellen in Tabellen, Befehl ist \backslashbox{}
\usepackage{diagbox}
%Ben�tigt, um Tabellen automatisch am Rand umzubrechen
\usepackage{tabularx,ragged2e}
%Ben�tigt, um Captions bei Tabellen jeweils zu zentrieren, k�nnte auch in den Metadaten f�r alle passieren
\usepackage{caption}
%F�r ToDo-Annotationen, bei Bedarf entfernen
\usepackage{todonotes}
%klickbare �berschriften und Co.
\usepackage{hyperref}

%ben�tigt, um Abbildungen im Anhang nicht in den Verzeichnissen zu haben.
\usepackage{ifthen}

%Bef�llen f�r Abk�rzungen
\newacronym{TOGAF}{TOGAF}{The Open Group Architecture Framework}
\newacronym{PK}{PK}{Planung und Kontrolle}
\newacronym{IV}{IV}{Informationsversorgung}
\newacronym{F+E}{F\&E}{Forschung und Entwicklung}
\newacronym{KuE-C}{KuE-C}{Kosten- und Erfolgscontrolling}
\newacronym[
 user1={Finanzcontrollings}
]{F-C}{F-C}{Finanzcontrolling}
\newacronym{I-C}{I-C}{Investitionscontrolling}
\newacronym{R-C}{R-C}{Risikocontrolling}
\newacronym{B-C}{B-C}{Beschaffungscontrolling}
\newacronym{P-C}{P-C}{Produktionscontrolling}
\newacronym{M-C}{M-C}{Marketingcontrolling}
\newacronym{L-C}{L-C}{Logistikcontrolling}
\newacronym{Pr-C}{Pr-C}{Projektcontrolling}
\newacronym{H-C}{H-C}{Hochschulcontrolling}
\newacronym{Per-C}{Per-C}{Personalcontrolling}
\newacronym{V-C}{V-C}{Vertriebscontrolling}
\newacronym{JIT-L}{JIT-L}{Just-In-Time-Lieferung}
\newacronym{KLR}{KLR}{Kosten- und Leistungsrechnung}
\newacronym{PMBOK}{PMBOK}{Project Management Body of Knowledge}
\newacronym{ROI}{ROI}{Return-On-Investment}
\newacronym{PPS}{PPS}{Produktionsplanung und -steuerung}
\newacronym{BDE}{BDE}{Betriebsdatenerfassung}
\newacronym{BSC}{BSC}{Balanced Scorecard}
\newacronym{SCM}{SCM}{Supply-Chain-Management}
\newacronym{SWOT}{SWOT}{Strengths-Weaknesses-Opportunities-Threats}
%\newacronym{OECD}{OECD}{Organisation for Economic Co-operation and Development}
\glsaddall

\renewcommand*{\glspostdescription}{} % Removes dots at the end of each entry.

%Ein neuer Verzeichnisstil f�r Abk�rzungen
         \newglossarystyle{acronym}{
            \setglossarystyle{long3col}
            \renewenvironment{theglossary}  
            {\begin{longtable}{@{} p{0.2\textwidth} @{} p{0.8\textwidth} @{} l @{}}}  
            {\end{longtable}
            }
            % Kopf der Tabelle
               \renewcommand*{\glossaryheader}{
                  \bfseries Abk�rzung
                  & \bfseries Bedeutung
                  & \bfseries
                  \\[3mm]
                  \endhead
               }
               \renewcommand*{\glossaryentryfield}[5]{
                  \glstarget{##1}{##2} %Abk�rzung
                  & ##5   %Beschreibung
                  & ##1 %Site Number
                  \\
               }
         }
\setglossarystyle{acronym}
\setglossarypreamble[acronym]{\vspace*{-\baselineskip}}

%Sonderverwendung von Akronymen 
\newcommand{\myglsgen}[1]{%
 \glsdoifexists{#1}%
 {%
   \ifglsused{#1}{%
     \acrshort{#1}%
   }% 
   {%
     \glsuseri{#1} (\acrshort{#1})%
     \glsunset{#1}%
   }%
 }%	
}%



%\renewcommand{\footnote}[1]{
%\foottrue\origfootnote{#1}
%\footfalse
%}

%%Acronyme in Fu�noten
%\ifglsacrfootnote
% \renewcommand{\acrfullformat}[2]{%
%   #2\protect\footnote{#2\space(#1)}%
%    \resetacronym{#1}
% }
%\else
% \renewcommand{\acrfullformat}[2]{#2\space(#1)}
%\fi
%
%%\SetCustomStyle
%Parametrierung der Packages
%section als 1 statt 1.1, damit Chapter nicht ben�tigt wird
\renewcommand{\thesection}{\arabic{section}}

%auch ab Ebene 3 bis 5 nummerieren
\setcounter{secnumdepth}{5}

%Stil der Kopf- und Fu�zeilen
\newpagestyle{mainmatter}{%
  \setheadrule{1pt}
  \sethead{\thesection ~\sectiontitle}{}{\thesubsection ~\subsectiontitle}%
  \setfoot{}{\thepage}{}
}
\newpagestyle{rest}{%
  \sethead{}{}{}
  \setfoot{}{\thepage}{}
}
%damit auch der Name der Subsection erkannt wird
\settitlemarks{section,subsection,subsubsection}



%Bilder in Unterordner suchen
\graphicspath{{image/}}

%Zeilenabstand
\setstretch{1.5}

%Bib-Ressourcen einbinden
\addbibresource{bibliography.bib}

%Tiefe des Inhaltsverzeichnisses
\setcounter{tocdepth}{2}

%neue Seite nach Kapitelende
\newcommand{\sectionbreak}{\clearpage}


%URLs im Literaturverzeichnis umbrechen
\setcounter{biburllcpenalty}{7000}
\setcounter{biburlucpenalty}{8000}

%Fu�zeilen nicht auf n�chster Seite fortf�hren
\interfootnotelinepenalty=10000

%Kurzreferenz bei zweiter Benutzung ohne "wie Anm. x"
\renewcommand\citenamepunct{\addcomma\space}
\renewbibmacro*{cite:seenote}{}

%Vertikale Zentrierung von Zelleninhalten
\renewcommand\tabularxcolumn[1]{m{#1}}
\newcolumntype{C}{>{\Centering\arraybackslash}X} % centered "X" column




%Seitennummerierungen
\newcounter{savepage}
\def\frontmatter{
	\pagestyle{rest}
    \pagenumbering{Roman}
    \setcounter{page}{1}
}

\def\mainmatter{
	\pagestyle{mainmatter}
	\setcounter{savepage}{\number\value{page}}
	\newpage
	\pagenumbering{arabic}
}

\def\backmatter{
	\pagestyle{rest}
  	\newpage
	\pagenumbering{Roman}
	\setcounter{page}{\value{savepage}}
}


%Latex Smartdiagram, B�umc zeichnen und Stil
\usepackage{forest}
\useforestlibrary{edges}

%footnote ohne Indent
\usepackage[hang,flushmargin]{footmisc}

\usepackage{array}

%Anhang nicht im TOC
\usepackage{etoolbox}
\appto\appendix{\addtocontents{toc}{\protect\setcounter{tocdepth}{1}}}
\appto\listoffigures{\addtocontents{lof}{\protect\setcounter{tocdepth}{1}}}
\appto\listoftables{\addtocontents{lot}{\protect\setcounter{tocdepth}{1}}}

%Abk�rzungsverzeichnis erstellen
\makeglossaries
%\makeindex
\begin{document}
\frontmatter
%\title{
%{Thesis Title}\\
%{\large Nordakademie Graduate School}\\
%{\includegraphics{nak.jpg}}
%}
%\author{Sebastian Schack}
%\date{17.04.2019}
%
%\maketitle


\thispagestyle{empty}
\vspace{2cm}
\begin{center}
\includegraphics[width=0.85\textwidth]{nak} 
\\\vspace{2cm}

\Large{Konzeptioneller Beitrag zur Bewertung von Flexibilit�t als Werttreiber in der gesch�ftlichen IT}

\vspace{1cm}
\large{Masterarbeit}

\vspace{5mm}
\large{Wirtschaftsinformatik $|$
 NORDAKADEMIE}
\vspace{1	cm}
\end{center}
{\large
\begin{flushleft}
%\linespread{3}
\textbf{Vorgelegt von:} Sebastian Schack \\
\textbf{Geboren am:} 29.10.1991\\
\textbf{Matr.-Nr.:} 9018 \\
\textbf{Gutachter:} \\
\begin{tightlist}
	\item \hspace{1cm} Prof. Dr. Hinrich Schr�der
	\item \hspace{1cm} Prof. Dr. Michael Schulz\\
\end{tightlist}
\vspace{5mm}
\textbf{Abgabe:} \today\\
\end{flushleft}
}
\vspace{1cm}

\newpage

\tableofcontents 
\newpage
\addcontentsline{toc}{section}{Abbildungsverzeichnis}
\listoffigures
\newpage
\addcontentsline{toc}{section}{Tabellenverzeichnisverzeichnis}
\listoftables
\newpage
%\addcontentsline{toc}{section}{Abk�rzungsverzeichnis}
\printglossary[type=\acronymtype,title=Abk�rzungsverzeichnis]

\newpage
\mainmatter
\section{Einleitung}
\subsection{Motivation und Zielsetzung}\label{zielsetzung}
Steigende Durchdringung unternehmerischen Umfelds durch informationstechnologische Systeme und die damit einhergehende steigende Gr\"o{\ss}e von IT-Organisationen\todo{Wort}, die unterst\"utzend oder direkt wertsch\"opfend die IT-Services zur Verf\"ugung stellen, zwingen IT-Verantwortliche, M\"oglichkeiten zur objektiven und zielgerichteten Steuerung der Gesamt-IT-Organisation zu etablieren.\newline
Daher bedarf es eines Ansatzes, zentrale Aufgaben des IT-Managements mittels dementsprechender Methoden aufeinander abzustimmen, sodass bestm\"ogliche Rahmenbedingungen zur unternehmerischen Zielerreichung geschaffen werden.
In Form des Controllings existiert Ansatz des allgemeinen Managements bereits in langfristig praxiserprobter Form.\footnote{Vgl. z.B. \cite[S.176f]{woehe2016einfuehrung} sowie \cite[S.25]{Horvath2015} und \cite[S.33ff]{kuepper2013controlling}, au{\ss}erdem \cite[S.20ff]{weber2015einfuehrung} zu anderen Definitionsans\"atzen}\\\\
Der Einsatz von Informationssystemen war fr\"uher prim\"ar technisch orientiert.\footnote{Vgl. \cite[S.VII]{Gadatsch2014}}\newline
Seit etwa 1990 verdichtet sich bei IT-Verantwortlichen allerdings die Ansicht, dass diese Systeme als Produktionsfaktor mit dem Controlling-Ansatz zu vernetzen sind.\footnote{Vgl. \cite[S.VII]{Gadatsch2014}}
Viele Elemente des klassischen Finanzcontrollings oder anderer Teilbereiche, wie z.B. die Balanced Scorecard, sind auch im IT-Controlling bereits gel\"aufig und k\"onnen anhand bestehender Methoden darauf ausgerichtet werden.\footnote{Vgl. \cite[S.46]{kesten2013}}\\\\
Die Rolle der IT-Organisation\todo{IT-Abteilung?} in einem Unternehmen kann verschieden ausgelegt werden, da bei den in der Praxis vorzufindenden Konstrukten durch die M\"oglichkeiten externer Dienstleister sowie Technologieanbieter (z.B. Cloud-Dienste) Schwerpunkte zu setzen sind, um optimale Gesamtfunktionalit�t zu erreichen.\footnote{Vgl. \cite[S.581f]{schroederHMD2016}}\\\\
In der Folge wird h\"aufig nicht die Gesamtheit einer theoretisch durch eine IT-Abteilung\todo{Auf Todo davor achten} abdeckbaren T\"atigkeiten tats\"achlich erbracht, sondern basierend auf inneren und \"ausseren Einfl\"u{\ss}en eine Verantwortlichkeitsverteilung vorgenommen.\footnote{Vgl. \cite[S.585-590]{schroederHMD2016}}\\\\
Die in diesem Kontext notwendige Flexibilit\"at, die dazu dienen kann, mit IT-Organisationen auf z.B. organisatorische Ver\"anderungen oder technologische Schwierigkeiten zu reagieren, um sie trotz kontinuierlich komplexer werdenden Umfelds zielsicher steuern zu k\"onnen und innerhalb dieser Rahmenbedingungen \"okonomisch bestm\"ogliche Verh\"altnisse zu erreichen, ist bisher nicht Bestandteil einer integrierten Betrachtung des IT-Controllings.
Auch dedizierte bzw. isolierte Untersuchungen zu Flexibilit\"atsaspekten existieren nur wenig und veraltet\footnote{Vgl. z.B. die fast 20 Jahre alten Beitr\"age von \cite[S.168ff]{byrdturner2000} und \cite[S.21ff]{byrdturner2001}}, ber\"ucksichtigen also nicht die aktuell vorherrschenden Zust\"ande.\\\\
Diese f\"ur die IT ausgebliebene Betrachtung von Flexibilit\"at durch die �bertragung bzw. Adaption von Methoden aus anderen fachbereichsbezogenen Controlling-Disziplinen nachzuholen, scheint daher ein naheliegendes Verfahren zu sein, um auch in der IT ein Wertbeitragsverst�ndnis f�r Flexibilit�t entwickeln zu k�nnen.
Adaptierbare Flexibilit�tsaspekte in diesen Disziplinen zu identifizieren ist zun�chst erforderlich, um dann erfassen zu k�nnen, inwiefern diese zur Erreichung unternehmerischer Ziele beitragen zu k�nnen.
Flexibilit�t im Kontext der IT-Organisation zu definieren, in Anlehnung an andere Teilbereiche des Controllings messbar zu machen und zu interpretieren ist Ziel und Bestandteil dieser Arbeit.
\subsection{Forschungsrelevanz}\label{Relevanz}
Das Feld der unternehmerisch genutzten Informationstechnologie ist dynamisch und kurzweilig - ein Charakteristikum, dessen Auspr\"agung sich bis heute versch\"arft.\footnote{Vgl. \cite[S.15]{capgemini2019}} 
Daher ist nicht verwunderlich, dass nationale und internationale Studien unabh\"angig voneinander immer wieder darauf hindeuten, dass IT-Projekte scheitern oder zumindest nicht erwartungskonform verlaufen.\footnote{Vgl. \cite[S.172]{fischer2016}}
Ein zu verzeichnender Trend ist zum Beispiel, dass Projektmanagement-Methoden tendenziell h\"aufiger agil als plangetrieben ausgelegt werden\footnote{Vgl. \cite[S.12]{statusquoagile2015}} und dadurch subjektiv bessere Resultate erzielt werden.\footnote{Vgl. \cite[S.22]{statusquoagile2015}} 
Es l\"asst sich f\"ur Projekte also ein Flexibilisierungstrend erkennen.\newline \todo{Flexibilit�t von agilen Methoden mit dem Manifest in Fu�note erl�utern?}
Was bedeutet Flexibilit\"at nun aber f\"ur die Gesamtauslegung der IT-Organisation?\\\\
Potentiellen Erwartungen steht gegen\"uber, dass eine dedizierte Auseinandersetzung bis vor zehn Jahren weder wissenschaftlich noch praktisch stattfand.\footnote{Vgl. \cite[S.53]{radermacher2009}} 
Nichtsdestotrotz erkannten bereits 2008 in einer Studie der Capgemini Unternehmensvertreter, dass IT-Flexibilisierung als ,,Megatrend'' einzustufen ist und Grund f\"ur ,,fundamentale Transformationsprozesse'' sein wird.\footnote{Vgl. \cite[S.17]{capgemini2008}}
Ratzer fasst die Relevanz von Flexibilit\"at wie folgt zusammen: ,,Um diese Situation besser kontrollieren zu k\"onnen, wird im Gegenzug eine noch weiter entwickelte IT ben\"otigt, die wiederum erneut den Komplexit\"ats- und Unsicherheitsgrad des Wettbewerbsumfelds erh\"oht. 
Dieser Mechanismus vollzieht sich in immer k\"urzeren Ver\"anderungszyklen, denen sich IT-Organisationen anpassen m\"ussen. 
Eine deutliche[sic!] h\"ohere Flexibilit\"at ist n\"otig.''\footnote{\cite{ratzer2009}}
Auch Wiedenhofer sieht in der Dynamik die Notwendigkeit f\"ur Flexibilit\"at gegeben, um damit auf auftretende Probleme zu reagieren: ,,Durch die Schaffung von geeigneten Strukturen steigert die IT-Organisation ihre Handlungsflexibilit\"at. 
Mit dieser F\"ahigkeit kann sie schnell auf wechselnde und komplexe Anforderungen reagieren.''\footnote{\cite[S.236]{Wiedenhofer2016}}
Er sieht in k\"urzeren Innovationszyklen, steigender Digitalisierung und der Geschwindigkeit des konjunkturellen Wandels insbesondere eine Bedrohung f\"ur bestehende Gesch\"aftsmodelle\footnote{Vgl. \cite[S.237]{Wiedenhofer2016}}, auf die mit Flexibilit\"at zu reagieren ist.\\\\
Zwar ist die Dynamik- bzw. Komplexit\"atsfloskel eine repetitiv paraphrasierte Scheinbegr\"undung, doch ist zu ermitteln, dass sich die Kontextualisierung der Forderung nach Flexibilit\"at mit dieser Art als problematisch eingestuften Rahmenbedingungen selbst in wissenschaftlichen Beitr\"agen bis heute erhalten hat, sodass diesbez\"ugliche Relevanz tats\"achlich im Zusammenspiel beider Faktoren zu begr\"unden ist. 
Tats\"achlich ist die Relevanz hinsichtlich praktischer Forschung weiter auch damit zu begr\"unden, dass die Behandlung zwar in der Fachwelt erfolgt, konkrete, konsensf\"ahige Beurteilungsmethoden und Handlungsvorschl\"age, z.B. auf Basis von Szenarioeinordnungen aber nicht ihren Weg in einschl\"agige Publikationen (z.B. Gadatsch, Mayer oder Tiemeyer) gefunden haben.


\subsection{Methodisches Vorgehen}\label{Methodisches Vorgehen}
%Analogiemethode
Ziel der Arbeit ist, wie in \ref{zielsetzung} angesprochen, die Messbarkeit von Flexibilit\"at zu untersuchen und ein Rahmenwerk zu definieren, welches Methoden auf zuvor in anderen Controlling-Disziplinen als adaptierbar identifizierten Methoden basiert und zu eruierenden Zielen und Zwecken zuf\"uhrt, welche wiederum aus allgemeinen Anspr\"uchen des Controllings abzuleiten sind. 
Auf diesem Weg soll Flexibilit\"at als Wertreiber greifbar und verst\"andlich werden, also auch verdeutlicht werden, welcher Nutzen aus flexiblen IT-Architekturen gezogen werden kann.\\
Ziel ist allerdings nicht, Flexibilit\"at an konkreten Beispielen zu messen und den Wertsch\"opfungsbeitrag an Realobjekten zu analysieren.\\\\
Grundlage der Forschung ist daher die theoretische, also auf Literatur gest\"utzte Erarbeitung von Grundlagen und Zielen des Controllings, Implmentation von darin enthaltenen Instrumenten zur Bewertung von Flexibilit�t, wiederum deren werttreibernder Aspekt und letztlich die integrierte Konsolidierung in einem Rahmenwerk zur Messung f\"ur das IT-Controlling.\\ 
Dieses Vorhaben hat deduktiven Charakter, wobei allerdings nicht vom ,,Allgemeinen auf einen besonderen Einzelfall''\footnote{\cite[S.37]{Sandberg2017}} zu schliessen ist, sondern Gesetzm\"assigkeiten \"ubertragen werden. 
Insbesondere die Rahmenbedingungen unterliegen hierbei der Notwendigkeit besonders differenzierter Betrachtung hinsichtlich der Vergleichbarkeit.\footnote{\cite[S.37-39]{Sandberg2017}}













\section{Controllingansatz}
\subsection{Definitionsans�tze}
Die Diskussion der Definitionsans�tze des Controllings soll das Ziel der Arbeit an allgemein anerkannten Vorstellungen ausrichten und damit sicherstellen, dass die sp�tere Konzeption zu erwartenden Anspr�chen gen�gen kann.\\
Controlling ist als Wissenschaftsdisziplin in Deutschland seit 1973 etabliert, als der erste Lehrstuhl in Darmstadt mit Peter Horv\'ath besetzt wurde.\footnote{Vgl. \cite[S.16]{weber2005internationalisierung}}
Dessen Publikation ,,Controlling'', aktuell in 13. Auflage, pr�gt bis heute ma�geblich das Verst�ndnis des Controllings.\footnote{Google Scholar z.B. listet das Buch als das mit der deutlich h�chsten Anzahl Zitationen anderer Autoren, vgl. $https://scholar.google.com/scholar?hl=de\&as_sdt=0\%2C5\&q=controlling\&btnG=$, abgerufen am 14.01.2020.}
Eine allgemeing�ltige Definition des Controllings zu formulieren, bezeichnet er als schwierig\footnote{Vgl. \cite[S.13]{Horvath2015}}, da es internationale Unterschiede im Verst�ndnis der zugeordneten Aufgaben gibt\footnote{Vgl. \cite[S.23]{Horvath2015}} und Controlling im praktischen Vergleich stark unterschiedlich ausgelegt wird.\footnote{Vgl. \cite[S.9-14]{Horvath2015}}
Die Ansicht, dass Controlling allgemeing�ltig schwer zu definieren ist, hat zu der wissenschaftlichen Aufgabe der Controlling-Konzeption gef�hrt, die davon ausgeht, dass Controlling nicht ausschlie�lich induktiv oder deduktiv definiert werden kann.\footnote{Vgl. \cite[S.13]{ossadnik2009controlling}}.
Die Controllingkonzeptionen sind als normative Aussagensysteme zu verstehen, die eine Grundvorstellung ausdr�cken, welche in der Praxis zu finden und gleichzeitig theoretisch fundiert ist.\footnote{Vgl. \cite[S.13]{ossadnik2009controlling}}
Sie stellen Konglomerate von Controlling-Aufgaben in den Kontext des daraus f�r Unternehmen resultierenden Nutzens.\footnote{Vgl. \cite[S.7]{Hubert2018}}
Neben Horv\'aths diesbez�glichem Ansatz gelten die Ans�tze von K�pper et al. sowie Weber \& Sch�ffer als einflussreich.\footnote{Vgl. \cite[S.24,60]{Horvath2015} sowie \cite[S.8]{Hubert2018}}
\newline
Horv\'ath sieht Controlling als ein Subsystem des Managements, welches koordinierend f�r die Subsysteme der \gls{PK} und der \gls{IV} wirkt.\footnote{Vgl. \cite[S.47-48,60]{Horvath2015}}
\newline
K�ppers Definitionsansatz unterscheidet sich davon nur graduell.\footnote{Vgl. \cite[S.59]{Horvath2015}}
Er fasst das Controlling als als Koordination des gesamten F�hrungssystems mit dem Ziel der zielgerichteten Lenkung auf.\footnote{Vgl. \cite[S.27]{kuepper2013controlling}}
\newline
Dieses Ziel geben auch Weber \& Sch�ffer an, indem Sie Controlling als das Aufgabensystem zur Sicherung der Rationalit�t in der F�hrung wiedergeben.\footnote{Vgl. \cite[S.48]{weber2015einfuehrung}}
\newline
Abseits prozess- oder strukturorientierter Controlling-Konzeptionen sind in verbreiteter Literatur jedoch auch klassische Definitionsans�tze zu finden.
Eine dieser simpleren Definitionen findet sich z.B. bei W�he. 
Dieser fasst Controlling zusammen als ,,die Summe aller Ma�nahmen, die dazu dienen, die F�hrungsbereiche Planung, Kontrolle, Organisation, Personalf�hrung und Information so zu koordinieren, dass die Unternehmensziele optimal erreicht werden.''\footnote{\cite[S.176]{woehe2016einfuehrung}}


\subsection{Aufgaben und Ziele des Controllings}
Ausgehend von den f�nf durch W�he formulierten Aufgaben- bzw. F�hrungs-bereichen ist festzuhalten, dass Controllinginstrumente Koordination und Lenkung erm�glichen sollen.
Intention ist dabei immer, egal ob ein struktur- oder prozessorientierter Definitionsansatz geltend gemacht wird, dass die Instrumente unternehmerisches Handeln auf ein Ziel ausrichten und dabei rationalit�tssichernd wirken sollen, also  das Management in die Lage des objektiven und damit faktengest�tzten Entscheidens und Verhaltens versetzen sollen.
Hierbei stellt sich die Frage, wie das Controlling in der Praxis zu entwickeln ist.
Eine diesbez�glich g�ngige Unterscheidung liegt in der zeitlichen Ausrichtung\footnote{Vgl. \cite[S.109]{Horvath2015}}, bei der zwischen operativem\footnote{Vgl. \cite[S.109-110]{Horvath2015}, \cite[S-42-50]{Buchholz2013} und \cite[S.69-91]{Schroeter2002}} und strategischem\footnote{Vgl. \cite[S.109-118]{Horvath2015}, \cite[S.42-58]{Buchholz2013} sowie \cite[S.91]{Reichmann2017}, wobei letzterer das strategische Controlling weniger �ber seine zeitliche Ausrichtung definiert, sondern es als Teilbereich auf Basis seiner Inhalte von anderen Controlling-Disziplinen wie dem Produktionscontrolling abgrenzt.} Controlling unterschieden wird.
%----Tabelle Anfang
\begin{table}
\setlength\extrarowheight{1pt} %etwas mehr Luft
\captionsetup{justification=centering} %Caption zentrieren
\centering
\begin{tabularx}{\textwidth}{|C|C|C|}
\hline
\backslashbox{\textbf{Merkmale}}{\textbf{C.-Typen}}  & {\textbf{Strategisches Controlling}}           & {\textbf{Operatives Controlling}}                                  \\ \hline
Orientierung                                          & Umwelt und Unternehmung: Adaption   & Unternehmung: Wirtschaftlichkeit betrieblicher Prozess  \\ \hline
Planungsstufe                                         & Strategische Planung                & Taktische und operative Planung, Budgetierung           \\ \hline
Dimensionen                                           & Chancen/Risiken, St�rken/Schw�chen  & Aufwand/Ertrag, Kosten/Leistungen                       \\ \hline
Zielgr��en                                            & Existenzsicherung, Erfolgspotential & Wirtschaftlichkeit, Gewinn, Rentabilit�t   \\ \hline           
\end{tabularx}
\caption[Controlling-Parameter nach Horv\'ath]{Controlling-Parameter nach Horv\'ath\footnotemark.}\label{tab:1}
\end{table}
\footnotetext{\cite[S.109]{Horvath2015}}
%----Tabelle Ende
               
\subsection{Controllingbereiche}
\subsubsection{Finanzcontrolling}
\subsubsection{Beschaffungscontrolling}
\subsubsection{Produktionscontrolling}
\subsubsection{Logistikcontrolling}
\subsubsection{Projektcontrolling}
\subsection{Kennzahlensysteme}
\subsubsection{Du-Pont-Kennzahlensystem}
\subsubsection{Diebold-Kennzahlensystem}
\subsubsection{SVD-Kennzahlensystem}
\subsubsection{Balanced Score Card}
\subsubsection{Statuskonzept von K\"utz}
\section{Produktionscontrollings}
\subsection{Definition}\label{DefinitionProduktionscontrolling}
Nachdem in \ref{Produktionscontrolling} das \gls{Pr-C} aufgrund seiner zu Flexibilit�t einschl�gigen Inhalte als aussichtsreiches Portfolio identifiziert wurde, ist es erforderlich, das \gls{Pr-C} umfangreich zu erfassen, die Methoden und Techniken zu strukturieren und auf Einschl�gigkeit zu Flexibilit�t zu pr�fen und schlie�lich eine Auswahl von in der Konzeption einzuschlie�enden bzw. zu �bertragenden Elementen zu formulieren. 
Grunds�tzlich ist das \gls{Pr-C} die Disziplin bzw. betriebliche T�tigkeit, die dazu dient, die Anspr�che des Controllings in der Produktion zu platzieren und umzusetzen.\footnote{Vgl. \cite[S.20]{Gottmann2019}}
Die Produktion hat dabei die Aufgabe, Wertsteigerung von Produkten zu erwirken, indem ein Input einem Output gegen�bergestellt wird.\footnote{Vgl. \cite[S.20]{Gottmann2019}}
Dabei handelt es sich neben direktem Input in Form von Produktionsanlagen, Material und Arbeitsleistung auch um indirekten Input wie die Organisation, Planung und Steuerung.\footnote{Vgl. \cite[S.19]{Gottmann2019}}
Das Controlling, dessen Ziel wiederum die ergebnisorientierte Planung und Steuerung von Ma�nahmen durch Beschaffung, Aufbereitung, Analyse und Kommunikation von Daten ist\footnote{Vgl. \cite[S.20]{Gottmann2019}}, muss also in den entscheidenden Parametern auf die Produktion und die kaufm�nnischen Zielsetzungen ausgerichtet werden\footnote{Vgl. \cite[S.20]{Gottmann2019}} und letztlich einen effizienten und erfolgreichen Betrieb sicherstellen\footnote{Vgl. \cite[S.20]{Gottmann2019}}, eine ganzheitliche Optimierung von Investitionsentscheidungen zu erm�glichen\footnote{Vgl. \cite[S.20]{Gottmann2019}} und vor allem Kompromisse zwischen bei den kaufm�nnischen und produktionsrelevanten Zielsetzungen\footnote{Vgl. \cite[S.20]{Gottmann2019}, \cite{schnell2018produktion} und \cite[S.24-26]{klein2012controlling1} sowie} zu finden. 
Dahingehend ist es also Aufgabe des \gls{Pr-C}, Produktions- und Controllingziele zu verbinden\footnote{Vgl. \cite[S.21]{Gottmann2019}} , den angesprochenen Input und Output zu optimieren\footnote{Vgl. \cite[S.21]{Gottmann2019}} und daf�r die richtigen Instrumente ausw�hlen, zu implementieren und einzusetzen.\\
Das \gls{Pr-C} differenziert in seinen T�tigkeiten die zeitlichen Dimensionen grunds�tzlich.
Je nach Interpretation wird lediglich zwischen strategischem und taktisch-operativen \gls{Pr-C} unterschieden\footnote{Vgl. \cite[S.25]{klein2012controlling1}}, w�hrend andere auch letzteres als unterschiedliche Dimensionen auslegen.\footnote{Vgl. \cite[S.9]{Gottmann2019}}
Letztlich ist die Controlling-Konzeption dabei aufgrund der Mangementunterst�tzung immer am Management-System auszurichten.
Auch hierbei ist eine Unterscheidung nach strategischem\footnote{Vgl. \cite[S.20]{zaepfel2014strategisches}}, taktischem\footnote{Vgl. \cite[S.20]{zaepfel2010taktisches}} und operativem\footnote{Vgl. \cite[S.20]{zaepfel1982operatives}} Produktionsmanagement m�glich.
Eine m�gliche Auslegung ist z.B., in der strategischen Perspektive langfristige Ziele innerhalb des Marktes zu betrachten, in der taktischen das Produktionsprogramm in Breite und Tiefe zu fokussieren und in der operativen die laufenden Fertigungsauftr�ge.\footnote{Vgl. \cite[S.4]{zaepfel2010taktisches}} 
\\\\
Die Begriffe des \gls{P-C}, der Produktionsplanung und des Produktionsmanagement sind nicht vollst�ndig klar gegeneinander abzugrenzen.
Gottmanns Definition schlie�t Planung als Bestand des \gls{P-C} ein, w�hrend z.B. L�dding die Planung als prim�ren Vorgang beschreibt und das \gls{P-C} davon trennt und im Controlling-Aspekt lediglich die operative Zielerreichungsbestimmung sieht.\footnote{Vgl. \cite[S.120-121]{Loedding2016}} 
Zwar w�re das Produktionsmanagement als F�hrungsaufgabe der Produktion, die durch das \gls{P-C} zu unterst�tzen ist, logisch von diesen abzugrenzen, doch es existieren Definitionsans�tze zum Produktionsmanagement, die darin ebenfalls Planung und Steuerung verorten und dazu deutlich �berschneidende Methodenportfolios vorschlagen.\footnote{Vgl. \cite[S.6]{grap1998produktion}}\\
Hier stellt sich nun die Frage, inwieweit die Differenzierung der Funktionsbereiche dem Vorhaben dieser Arbeit zutr�glich ist.
Da vor allem der Gesamtbereich der planerischen und steuernden Aspekte der Produktion einschl�gige �berlegungen zu Flexibilit�t aufweist und deren �bertragbarkeit gepr�ft werden soll, scheint eine harte Begriffstrennung insofern nicht hilfreich, als dass Methoden aufgabenbereichs�bergreifend zum Einsatz kommen k�nnen.
Die Unterscheidung  der zeitlichen Planungshorizonte (strategisch, taktisch, operativ) ist ferner �bergreifend in immer �hnlicher Auslegung zu bemerken, sodass eine weniger strikte Trennung dar�ber hinaus nicht trivialisierend scheint.
Die alleinige Betrachtung von Steuerungsmethoden, also die Ausklammerung von Planungsmethoden w�re sowieso eine unangemessene Reduktion des Untersuchungsbereichs.
\subsection{Betrachtungsgegenst�nde}
\subsubsection{Bedarfsplanung}
%bild produktionsplanung
\todo{einleiten mit produktionsplanungsprozess}
\begin{itemize}
\item Prim�rbedarf\\
Die Prim�rbedarfsplanung ermittelt die herzustellende Menge der zum Absatz bestimmten, d.h. verkaufsf�higen Erzeugnisse, Baugruppen oder Einzelteile nach Art, Menge und Termin bzw. Planungsperiode.\footnote{Vgl. \cite[S.199]{Abts2017} und \cite[S.108]{Loedding2016}}
Dieser Prozess wird ma�geblich durch Kalkulation bestehender Auftr�ge sowie Absatzprognosen auf der einen Seite und maschinelle sowie personelle Kapazit�t auf der anderen Seite beeinflusst.\footnote{Vgl. \cite[S.199]{Abts2017}}
Ebenfalls gel�ufig ist die Bezeichnung Produktionsprogramm bzw. Produktionsprogrammplanung.\footnote{Vgl. \cite[S.108]{Loedding2016}}
\item Sekund�rbedarf\\
Der darauf aufbauende Materialbedarf bzw. Sekund�rbedarf und dessen Planung ermittelt anhand von St�cklisten oder fr�herer Verbrauchswerte abz�glich Lagerkapazit�ten\footnote{Vgl. \cite[S.200]{Abts2017}} den f�r den Prim�rbedarf notwendige Menge an Komponenten und Teilen und ordnet diese periodengerecht zu\footnote{Vgl. \cite[S.110]{Loedding2016}}.
\item Terti�rbedarf\\
Dar�ber hinaus kann ein Terti�rbedarf erfasst werden, der den Bedarf an Betriebs- und Hilfsstoffen sowie Verschlei�material\footnote{Vgl. \cite[S.104]{Pfohl2018}} anzeigt.\footnote{Vgl. \cite[S.243]{Alpar2019}}
\end{itemize}
Die ermittelten Sekund�r- und Terti�rbedarfe stellen zun�chst grunds�tzlich Bruttobedarfe dar, die sich durch Lagerbestandsfortschreibung in Nettobedarfe �berf�hren lassen.\footnote{Vgl. \cite[S.243]{Alpar2019}}\\
Die Zusammenh�nge sind in Abbildung \ref{img:Materialbedarfsarten} dargestellt.
%Bild ->
\begin{figure}[!htb]
\centering
\includegraphics[width=1\linewidth]{pfohl_Materialbedarfsarten}
\captionof{figure}[Materialbedarfsarten]{Materialbedarfsarten\footnotemark}
\label{img:Materialbedarfsarten}
\end{figure}
%Bild <-
\addtocounter{footnote}{+0}\footnotetext{\cite[S.287]{hartmann2002materialwirtschaft}}
Die Berechnungsmethoden f�r die skizzierten Zwecke werden literatur�bergreifend unterschieden zwischen deterministischen, stochastischen und Sch�tzungs-Ans�tzen.\footnote{Vgl. \cite[S.284]{hartmann2002materialwirtschaft}, \cite[S.105]{Pfohl2018} und \cite[S.443ff und S.489ff]{Schoensleben2016}}
\begin{itemize}
\item Deterministisch Verfahren\\
Deterministische Verfahren existieren sowohl analytischer als auch synthetischer Natur.
W�hrend analytische Verfahren die exakte Kalkulation anhand von St�cklisten vornehmen\footnote{Vgl. \cite[S.105]{Pfohl2018}}, geht die synthetische Bedarfsermittlung mit Teileverwendungsnachweisen an die Ermittlung heran.\footnote{Vgl. \cite[S.105]{Pfohl2018}}
Deterministische Verfahren sind deduktiv.
\item Stochastisch Verfahren\\
Stochastische nutzen zur Bedarfsermittlung historische Verbrauchsdaten vergleichbarer Produktionen.
Auf deren Basis wird eine Prognose der geplanten Produktion vorgenommen.\footnote{Vgl. \cite[S.106]{Pfohl2018}}
Je nach Tendenz (steigend, gleichbleibend) sind daf�r Methoden wie die Mittelwerbildung, exponentielle Gl�ttung oder Regressionsrechnung m�glich.\footnote{Vgl. \cite[S.106]{Pfohl2018}}
Stochastische Verfahren sind induktiv.
\item Sch�tzverfahren\\
Sind f�r keine der beiden genannten Methoden die Voraussetzungen gegeben, so bleiben lediglich Sch�tzmethoden �brig.
Hierbei ist lediglich zu unterscheiden zwischen rein intuitiven Sch�tzungen einer oder mehrerer Personen und logisch begr�ndbaren und damit intersubjektiv �berpr�fbaren Sch�tzungen.\footnote{Vgl. \cite[S.106]{Pfohl2018}}
\end{itemize}
Die Zusammenh�nge der Berechnungsmethoden sind in Abbildung \ref{img:berechnungsmethoden} dargestellt.
%Bild ->
\begin{figure}[!htb]
\centering
\includegraphics[width=1\linewidth]{methoden_bedarfsermittlung}
\captionof{figure}[Methoden der Bedarfsermittlung]{Methoden der Bedarfsermittlung\footnotemark}
\label{img:berechnungsmethoden}
\end{figure}
\addtocounter{footnote}{+0}\footnotetext{\cite[S.289]{hartmann2002materialwirtschaft}}
%Bild <-
%---------------
%LOSGR�SSEN
\subsubsection{Losgr��en}
	Ein Los besteht ,,aus einer bestimmten Anzahl konstruktiv und technologisch gleicher oder �hnlicher Einzelteile, die unabh�ngig davon, ob sie zu einem oder mehreren Endprodukten geh�ren, gemeinsam in einem Fertigungsauftrag unter einmaliger Gew�hrung der R�stzeit\footnote{,,Als R�sten bezeichnet man den Vorgang, die Maschine auf die Fertigung eines neuen Teiles oder Loses einzurichten. Teil des R�stens sind auch Probel�ufe der Maschine.'' - \cite[S.182]{vahrenkamp2008produktionsmanagement}} je Arbeitsgang und Arbeitsplatz gefertigt werden.''\footnote{\cite[S.670]{nebl2007produktionswirtschaft}}
	Eine Losgr��e beschreibt demnach die Menge gleichartiger Objekte, die nacheinander in einem R�stvorgang angefertigt werden.
	Losgr��en sind sowohl f�r Prim�r- als auch Sekund�r- und Terti�rbedarf festzulegen.\footnote{Vgl. \cite{siepermann2018}}
	Bei der Losgr��enplanung handelt sich um ein Methodenportfolio zur Kosten- oder Flussoptimierung\footnote{Vgl. \cite[S.153]{vahrenkamp2008produktionsmanagement}}
	Es wird zwischen Durchlaufzeitminimierung, Flussoptimierung mit Engpassber�cksichtigung, Kostenminimierung und Lager- und-Produktionskostenoptimierung unterschieden.
	\begin{itemize}
		\item Durchlaufzeitminimierung\\
		Der Ansatz durchlaufzeitminimaler Lose stammt aus der Lean Production\footnote{Vgl. \cite[S.154]{vahrenkamp2008produktionsmanagement}} und fokussiert exklusiv die Minimierung der Produktionszeit eines Loses.\footnote{Vgl. \cite[S.154]{vahrenkamp2008produktionsmanagement}}
		Die R�stvorg�nge werden dabei genau wie die Produktionsvorg�nge lediglich hinsichtlich der Dauer betrachtet.\footnote{Vgl. \cite[S.154]{vahrenkamp2008produktionsmanagement}}
		Das Verfahren versucht zu hohen Anteil an R�stzeiten gegen�ber zu langer Bearbeitungsdauer zu optimieren.\footnote{Vgl. \cite[S.154]{vahrenkamp2008produktionsmanagement}}
		\item Flussoptimierung mit Engpassber�cksichtigung\\
		Wie auch die Durchlaufzeitminimierung besteht auch dieser Ansatz in zeitlicher Optimierung.\footnote{Vgl. \cite[S.155]{vahrenkamp2008produktionsmanagement}}
		Der Ansatz ist vor allem dann relevant, wenn verschiedene Produkte in vorgegebenem Zyklus hintereinander auf einer Maschine produziert werden muss.
		Die Problematik besteht weniger in diesem Vorgang als in der Synchronisierung mit anschlie�enden Vorg�ngen die von dessen Erzeugnissen abh�ngig sind bzw. darauf aufbauen.\footnote{Vgl. \cite[S.155]{vahrenkamp2008produktionsmanagement}}
		Das Verfahren stimmt die Losgr��e auf den Bedarf ab.\footnote{Vgl. \cite[S.155-157]{vahrenkamp2008produktionsmanagement}}
		\item Kostenminimierung\\
		Die Kostenminimierung ist hingegen ein klassisches betriebswirtschaftliches Losgr��enbestimmungsverfahren.
		Die fixen R�stkosten zuz�glich der variablen Herstellungskosten sind rein �konomisch anhand des Bedarfes und der m�glichen Laufzeiten so zu kalkulieren, dass die Kosten m�glichst gering sind.\footnote{Vgl. \cite[S.158]{vahrenkamp2008produktionsmanagement}}
		\item Lager- und-Produktionskostenoptimierung\\
		In diesem Verfahren werden zus�tzlich zu R�st- und Produktionskosten die Lagerkosten ber�cksichtigt und das Verh�ltnisse f�r einen isolierten Teil der Produktionsstufe optimiert.\footnote{Vgl. \cite[S.159]{vahrenkamp2008produktionsmanagement}}
		Die Pr�misse des Verfahrens ist, dass Erzeugnisse mit Fertigstellung Lagerkosten verursachen.
		Dabei sind vor allem h�ufige R�stkosten hohen Lagerkosten gegen�ber zu optimieren.
		Die optimale Losgr��e nach Andler z.B. ermittelt eine Losgr��e, welche die Summe von R�st- und Lagerkosten minimiert und ist auch auf Einkaufslosgr��en �bertragbar, wenn R�stkosten durch bestellfixe Kosten ersetzt werden.\footnote{Vgl. \cite{andler1929}}
	\end{itemize}
	Die Berechnungsmethoden f�r die skizzierten Zwecke sind entweder statischer oder dynamischer Natur.\footnote{Vgl. \cite[S.284]{hartmann2002materialwirtschaft}, \cite[S.105]{Pfohl2018} und \cite[S.443ff und S.489ff]{Schoensleben2016}}
	\begin{itemize}
		\item Statische Verfahren
		Statische Verfahren wie der Ansatz von Andler verwenden lediglich die Kosten (R�st-, Lager- und variable Produktionskosten) und berechnen die Losgr��e einer Planungsperiode.\footnote{Vgl. \cite[S.162]{vahrenkamp2008produktionsmanagement}}
		\item Dynamische Verfahren
		Dynamische Verfahren sind dagegen auf zeitlich ver�nderliche Nachfragemengen ausgerichtet.
		Au�erdem existieren Verfahren f�r ein- und mehrstufige Verfahren.
	\end{itemize}	
	%---------------
%TERMIN- und KAPAZIT�TSPLANUNG	
\subsubsection{Termin- und Kapazit�tsplanung}
\todo{In der Leseprobe von Vahrenkamp fehlten hier Seiten}
Nach Abschluss der Planung der Produktionsmengen ist festzulegen, in welcher Weise Auftr�ge die Produktion zu durchlaufen\footnote{Vgl. \cite[S.181]{vahrenkamp2008produktionsmanagement}} haben und welche Zeitstrukturen dabei einzuhalten sind.\footnote{Vgl. \cite[S.214]{Abts2017}}
Dabei ist auch die Kapazit�t von Infrastruktur und Personal zu ber�cksichtigen.\footnote{Vgl. \cite[S.200]{Abts2017} und \cite[S.181]{vahrenkamp2008produktionsmanagement}}
Der Planungsprozess setzt sich aus der Durchlauf- und Kapazit�tsterminierung zusammen.\footnote{Vgl. \cite[S.181]{vahrenkamp2008produktionsmanagement}}\\
%Durchlaufterminierung
Die Durchlaufterminierung legt vorl�ufige Star- und Endtermine der Arbeitsvorg�nge sowie deren Koordination grob fest.\footnote{Vgl. \cite[S.181]{vahrenkamp2008produktionsmanagement}}
Kapazit�tsrestriktionen bleiben bis zu diesem Punkt unber�cksichtigt.\footnote{Vgl. \cite[S.181]{vahrenkamp2008produktionsmanagement}}
Zentraler Aspekt bei dieser Planung ist die Arbeitsplatzdurchlaufzeit, die die Zeitspanne f�r jeden Arbeitsschritt definiert, um diesen zwischen dem davor und dem danach liegenden Arbeitsschritt einzuordnen.\footnote{Vgl. \cite[S.182]{vahrenkamp2008produktionsmanagement}}
Die Arbeitsplatzdurchlaufzeit setzt sich dabei aus aus den Komponenten Transportzeit, Wartezeit, R�stzeit und der eigentlichen Bearbeitungszeit zusammen\footnote{Vgl. \cite[S.182]{vahrenkamp2008produktionsmanagement}}, vgl. Abbildung \ref{img:arbeitsplatzdurchlaufzeit}, wobei sowohl ablauforganisatorische oder technische Gr�nde f�r Wartezeit verantwortlich sein k�nnen (z.B. Materialaush�rtung).\footnote{Vgl. \cite[S.181]{vahrenkamp2008produktionsmanagement}}
%Bild ->
\begin{figure}[!htb]
\centering
\includegraphics[width=0.5\linewidth]{arbeitsplatzdurchlaufzeit}
\captionof{figure}[Arbeitsplatzdurchlaufzeit]{Arbeitsplatzdurchlaufzeit\footnotemark}
\label{img:arbeitsplatzdurchlaufzeit}
\end{figure}
\addtocounter{footnote}{+0}\footnotetext{\cite[S.182]{vahrenkamp2008produktionsmanagement}}
%Bild <-
Die Summe aller Arbeitsplatzdurchlaufzeiten ergibt die Sch�tzung f�r die Durchlaufzeit eines gesamten Auftrags.
Aufgrund m�glicher Konkurrenzen um Arbeitsstationen k�nnen sich Wartezeiten ver�ndern und, da R�stzeiten reihenfolgen- und zustandsabh�ngig sind, k�nnen sich diese ebenfalls ver�ndern, sodass ohne Kapazit�tsber�cksichtigung die Durchlaufzeitenkalkulatio lediglich eine zu interpretierende Sch�tzung darstellt.\footnote{Vgl. \cite[S.182]{vahrenkamp2008produktionsmanagement}}
Der Pfad der Gesamtdurchlaufzeit stellt den kritischen Pfad der Produktion dar.\footnote{Begriff aus der Netzplantechnik, vgl. \cite[S.182]{vahrenkamp2008produktionsmanagement}}
Mithilfe von Vorw�rts- oder R�ckw�rtsterminierung werden letztlich alle Zeitpunkte bzw. Termine f�r die Produktion festgelegt.\footnote{Vgl. \cite[S.184-185]{vahrenkamp2008produktionsmanagement}}\\\\
%Kapazit�tsterminierung
Aus der Durchlaufterminierung resultieren terminierte Auftr�ge, deren Durchf�hrbarkeit noch nicht best�tigt ist.\footnote{Vgl. \cite[S.185]{vahrenkamp2008produktionsmanagement}}
Diese Verifikation ist Aufgabe der Kapazit�tsterminierung in Form der Ermittlung von Unter- bzw. �berauslastungen, die untereinander ausgeglichen werden m�ssen.\footnote{Vgl. \cite[S.200]{Abts2017}}
M�gliche Kapazit�tseinschr�nkungen resultieren aus Wartungs- und Instandhaltungsarbeiten, Produktionsst�rungen sowie Urlaubs- und Krankheitszeiten des Personals.\footnote{Vgl. \cite[S.200]{Abts2017}}
Solche Kapazit�tsengunstimmigkeiten bedingen entweder die Anpassung des Kapazit�tsangebots an die Kapazit�tsnachfrage (Kapazit�tsanpassung) oder umgekehrt (Belastungsanpassung).\footnote{Vgl. \cite[S.126-128]{kurbel2016enterprise},\\ \cite[S.186-187]{vahrenkamp2008produktionsmanagement},\\ \cite[S.190-193]{zaepfel2010grundzuege} und\\ \cite[S.689-690]{schweitzer1994industriebetriebslehre}}
Kapazit�tsanpassungen sind z.B. m�glich durch zeitliche Modifikation (�berstunden oder Kurzarbeit bei �berlastung, Schichtabbau bei Unterlastung, etc.), Intensit�tsanpassung (Durchsatzerh�hung oder -verringerung durch Anpassung der Produktionsgeschwindigkeit) oder quantitativer Anpassung (Nutzung von Reserven bei �berlastung, tempor�re Stilllegung bei Unterlastung, Umschichtung von Personal aus anderen Bereichen, etc.).\footnote{Vgl. \cite[S.267-269]{kiener2012produktions} \cite[S.229]{guenther2011produktion}}\\\\
Belastungsanpassungen sind z.B. durch zeitliche Verschiebung von Fertigungsauftr�gen, die nicht bereits zum fr�hesten Zeitpunkt geplant sind, auf Zeitpunkte mit geringerer Auslastung zu realisieren.
Ferner sind Stauchungen und Streckungen durch geringere oder h�here Kapazit�tsinanspruchnahme m�glich, Anpassung der Auftragsgr��e (falls nur ein Teil des Loses zur Auftragserf�llung notwendig ist (�berlastung) oder �berproduzierte Errzeugnisse auf Lager gelegt werden k�nnen (Unterlastung)), externe Auftragsvergabe bis hin zu Auftragsverzicht (�berlastung) oder Auftragsannahme (Unterlastung) oder, sofern technisch m�glich, alternative Durchf�hrung von Arbeiten auf anderen Betriebsmitteln.\footnote{Vgl. \cite[S.269-271]{kiener2012produktions} und\\ \cite[S.716-720]{Nebl2011}}\\
Die Ma�nahmen sind dabei nicht immer klar voneinander abzugrenzen, da z.B. Intensit�tsanpassungen auch Stauchungen bzw. Streckungen bedingen.
\subsubsection{Auftragsfreigabe}
\subsubsection{Ablaufplanung}



\subsection{Teilbereiche}
%https://refa.de/service/refa-lexikon/produktionsmanagement mal gucken
\subsubsection{Strategisches Produktionscontrolling}
\subsubsection{Taktisches Produktionscontrolling}
\subsubsection{Operatives Produktionscontrolling}
\subsection{Methoden und Techniken}
KLR selbst digital extrem relevant\cite{kuenzel2018}

\subsubsection{Strategische Instrumente}
\paragraph{Produktlebenszyklus-Analyse}
\paragraph{Balanced Scorecard}
\subsubsection{Operative Instrumente}
\paragraph{Kennzahlen}
\paragraph{Kennzahlensysteme}
%\section{Grundlagen des IT-Controllings}
\subsection{Definition}
\subsection{Einbettung in das IT-Management}
\subsection{Organisation }
\subsection{Ziele und Aufgaben}
\subsection{Teilbereiche}
\subsubsection{IT-Portfoliocontrolling}
\subsubsection{IT-Projektcontrolling}
\subsubsection{IT-Produktcontrolling}
\subsubsection{IT-Infrastrukturcontrolling}
\subsection{Methoden und Techniken}
\subsubsection{IT-Kennzahlen}
\subsubsection{IT-Balanced Scorecard}
\subsubsection{IT-Kosten- und Leistungsrechnung}
\subsubsection{Total Cost of Ownership}
\subsubsection{IT-Outsourcing}
\section{Flexibilit\"at}
\subsection{Allgemeines Verst\"andnis von Flexibilit\"at}
Nachdem also die strukturelle Basis f�r Konzeptionsm�glichkeiten erfasst ist und Betrachtungsgegenst�nde sowie zur Adaption erw�genswerte Messmethoden identifiziert sind, ist es notwendig, den fixierten Aspekt der Flexibilit�t insoweit allgemein zu umrei�en, im Kontext produktionswirtschaftlicher Strukturen zu etablieren und letztlich daraus emergierend auf die IT adaptierbare Aspekte zu aggregieren, dass dieses Aggregat m�gliche Betrachtungsgegenst�nde innerhalb der IT konstituiert.
F�r diese Betrachtungsgegenst�nde sind dar�ber hinaus dann quantifizierbare Aspekte zu ermitteln, die Flexibilit�t in der IT-Organisation ausdr�cken und sie dahingehend zu operationalisieren, dass daraus den IT-Wertbeitrag steigernde Ma�nahmen im Sinne einer IT-Strategie mit dem zumindest subordinierten Ziel der Flexibilisierung abzuleiten sind.
Sofern Systematisierungsans�tze identifizierbar sind, sollen diese ber�cksichtigt werden.\\\\
Diachronisch ist der Begriff Flexibilit�t auf das lateinische \textit{flectere} zur�ckzuf�hren (biegen, beugen, kr�mmen\footnote{ \cite{ponsflectere}.}) und bedeutet daher etwa Biegsamkeit oder Elastizit�t.\footnote{Vgl. \cite{brockhaus2006}: flexibel (allgemein).}
Bildungssprachlich meint Flexibilit�t die Anpassungsf�higkeit an ver�nderte Rahmenbedingungen\footnote{Vgl. \cite{brockhaus2006}: flexibel (bildungssprachlich).} und Flexibilisierung die ,,Lockerung beziehungsweise Aufl�sung von erstarrten Strukturen''\footnote{Vgl. \cite{brockhaus2006}: Flexibilisierung.}.
Komposita indizieren die  �bertragung auf unterschiedlichste Verwendungsbereiche, z.B. �konomie\footnote{Vgl. \cite{brockhaus2006}: Flexibilisierung bzgl. Arbeitszeitflexibilisierung.} und Psychologie\footnote{Vgl. \cite{brockhaus2006}: Flexibilit�t (Psychologie).}.
Inhaltlich verwandte aber nicht zwingend synonyme Begriffe - diese Bedeutung ist jeweils im Anwendungskontext zu erfassen - sind Anpassungsf�higkeit, Elastizit�t, Wandlungsf�higkeit und Agilit�t.\footnote{Vgl. \cite[S.16-24]{seebacher2013ansaetze} sowie \\\cite[S.222-223]{Bellmann2009}}
\subsection{Flexibilit\"at im Anwendungskontext der Produktion}
Da vor allem das \gls{P-C} dedizierte �berlegungen zu Flexibilit�t beweist und diese auch soweit operationalisiert, sie zu messen und konzeptionell einzubeziehen, ist es zur Adaption in dieser Hinsicht der vielversprechendste Teilbereich des Controllings.
Flexibilit�t muss nun auf die Anwendungskontexte der Produktion und der IT �bertragen werden.\\\\
%\subsubsection{Flexibilit\"at im Kontext der Produktion}
Als Antwort auf die dynamische Unternehmensumwelt r�ckt die produktionswirtschaftliche Flexibilit�t in den Fokus der strategischen Ausrichtung der Produktion und deren operativer Optimierung.\footnote{Vgl. \cite[S.221]{Bellmann2009}.}
Zur Beschreibung, Interpretation, Konzeptualisierung und Messung ist allerdings innerhalb mehrerer Jahrzehnte kein einheitliches Begriffsbild und kein koh�rentes Konstruktverst�ndnis entstanden.\footnote{Vgl. \cite[S.221]{Bellmann2009}.}
Da Flexibilit�t keinen Selbstzweck, sondern Mittel zur Erreichung von Systemzielen darstellt, ist es daher erforderlich, in diese Diskussion die situationsbezogene Gegen�berstellung von Flexibilit�tspotentialen und Flexibilit�tsbedarfen zu integrieren\footnote{Vgl. \cite[S.221]{Bellmann2009}.}, und darauf aufbauend zu ermitteln, wie das Flexibilit�tspotential entsprechend seines Bedarfes aufgebaut und quantitativ bzw. qualitativ dessen Erreichung verifiziert werden kann.
%Bellmann et al. haben in einer umfassenden Literaturrecherche viele Aspekte dieser Diskussion zusammengetragen und gegen�bergestellt.\todo{weg?}
\\\\
Flexibilit�t wurde urspr�nglich als reaktive Kompetenz ausgelegt, die zur Kompensation von wirtschaftskrisebedingten Nachfrageschwankungen diente.\footnote{Vgl. \cite[S.222]{Bellmann2009}.}
Das Verst�ndnis wurde in den 1950ern Jahren erweitert auf Potentiale zur Nutzung im wirtschaftlichen Aufschwung entstehender Marktchancen und erhielt dadurch proaktiven Charakter.\footnote{Vgl. \cite{dormayer1986konjunkturelle}; \\\cite[Zusammenfassung]{Voigt2007} sowie \cite[S.222]{Bellmann2009}.}
Zum Ende des 20. Jahrhunderts wurde Flexibilit�t zur strategischen Variable hinsichtlich Mengen- und Variantenreichtum und erlangte dadurch �hnlichen Stellenwert wie operativ flexible Kapazit�tsgestaltung.\footnote{Vgl. \cite{Voigt2007} und \cite[S.222]{Bellmann2009}.}\\
Eine derartige Priorisierung von Flexibilisierung hat allerdings in der Praxis unter der Annahme des Aufbaus von z.B. �berkapazit�ten antithetischen Charakter zur Aufrechterhaltung eines stabilen und �konomischen Betriebs.\footnote{Vgl. \cite[S.83]{vonGarrel2014}.}
Die erfolgskritischen Rahmenbedingungen der Innovationsf�higkeit und Wettbewerbsf�higkeit in Hinsicht auf den technischen Fortschritt zu erhalten, bedingt also einen Kompromiss zwischen Stabilit�t und Flexibilit�t\footnote{Vgl. \cite{vonGarrel2013}.} im Sinne einer m�glichst exakt bedarfsdeckenden Flexibilit�tspotentialschaffung.
Das Zusammenspiel ist in Abbildung \ref{img:garrel_flexibilitaet} dargestellt.\\
 \begin{figure}[H]
\centering
\includegraphics[width=0.85\linewidth]{garrel_flexibilitaet}
\captionof{figure}[Grundmodell der Flexibilit�t]{Grundmodell der Flexibilit�t\footnotemark}
\label{img:garrel_flexibilitaet}
\end{figure}
\addtocounter{footnote}{+0}\footnotetext{Eigene Darstellung in Anlehnung an \cite[S.84-85 bzw. Abb. 1]{vonGarrel2014}, wiederum in Anlehnung an \cite{Brehm2003} sowie \cite[S.87-92]{Brehm2004}.}
\noindent Begr�ndet ist diese Konstellation im Zusammenhang der korrespondierenden Kosten. Diesbez�glich besteht die Annahme, dass der Aufbau und die �nderungen von Flexibilit�tspotentialen sowie deren Aufrechterhaltung genauso Kosten verursachen, wie Opportunit�tskosten bei Inflexibilit�t entstehen.\footnote{Vgl. \cite[S.226]{Bellmann2009}.}
Einerseits k�nnen die Flexibilit�tskosten insofern als Versicherungspr�mie interpretiert werden\footnote{Vgl. \cite[S.226]{Bellmann2009}.}, aber andererseits auch Flexibilit�tspotentiale deckungsbeitragsgenerierend genutzt werden (z.B. durch Produktneueinf�hrungen) und so der unternehmerischen Direktive der Gewinnmaximierung dienen.\footnote{Vgl. \cite[S.226]{Bellmann2009}.}
\\\\
Flexibilit�t insofern im Kontext der Produktion zu betrachten, mandatiert die Implementation flexibilisierender Ma�nahmen f�r Produktionsfaktoren.
Die Implemenationsweise wird dabei bestimmt durch die Konzeption der Flexibilit�t, also deren angestrebten systemorientierten Auswirkungen.
\subsubsection{Flexibilit�tskonzeption}\label{flexibilitaetskonzeption}
Wie indiziert existiert kein einheitliches Begriffsverst�ndnis f�r produktionswirtschaftliche Flexibilit�t.
Einige Definitionsans�tze zentrieren z.B. adaptiven Systemcharakter als F�higkeit auf ver�nderte Rahmenbedingungen zu reagieren.\footnote{Vgl. hierzu z.B. \cite[S.71]{Jacob1982}; \\\cite[S.290]{Sethi1990}; \\\cite[S.22]{aprile2005operations} sowie \\\cite[S.143]{Garavelli2003}} 
Definitionsans�tze variieren im Bezug auf die Schwerpunktsetzung eines reaktiven oder proaktiven Profils\footnote{Vgl. dazu \cite[S.242]{Brehm2004} und zu den Ans�tzen \cite[S.9-10]{seebacher2013ansaetze} sowie \\\cite[S.73]{upton1994}}, stimmen aber darin insoweit �berein, dass jedwede reaktive oder proaktive im Rahmen von Flexibilit�tspotentialen ausgel�ste Ma�nahme zur tempor�ren, reversiblen und zielf�hrenden Systemver�nderung f�hren muss, egal ob der Ma�nahmenbezug antizipativ oder dekursiv ist.\footnote{Vgl. \cite[S.38]{Gerwin1987} sowie \\\cite[S.974]{Oke2005}}
In der Literatur variieren im Kontext der Flexibilit�t die Bezeichnungen, sodass die Gefahr entsteht, terminologisch ungenau zu werden.
Da die Begriffscharakterisierung allerdings ma�geblich die Anspr�che an m�gliche flexibilisierende Ma�nahmen definiert, ist eine Abgrenzung gegen vermeintliche Synonyme entscheidend.\\\\
Flexibilit�t dient einerseits der Erf�llung von Aufgaben, die zwar dynamisch entstehen, aber inhaltlich erwartet werden.\footnote{Vgl. \cite[S.76]{upton1994}.}
Agilit�t z.B. deckt dagegen unerwartete Aufgaben ab.\footnote{Vgl. \cite[S.33-34]{goransonagile}.}
Flexibilit�t bzw. korrespondierende Ma�nahmen dienen in der Produktion dabei wie angedeutet zur Vermeidung von Kosten oder Leistungsverlusten.\footnote{Vgl. \cite[S.1080]{Upton1997ProcessRI}.}
Von Wandlungsf�higkeit (auch Adaptivit�t\footnote{Vgl. \cite{voigt2005begriffsbestimmung}.}) unterscheidet sich Flexibilit�t vor allem durch den proaktiven Anspruch auszul�sender Ma�nahmen.\footnote{Vgl. \cite[S.215]{seebacher2011}.}
Flexibilit�t zielt zudem auf dynamische Ver�nderungen. Insgesamt ist die Anpassungsf�higkeit eine Teileigenschaft der Flexibilit�t.\footnote{Vgl. \cite[S.19]{seebacher2013ansaetze}.}
Wandlungsf�higkeit meint im Gegensatz zu Flexibilit�t einen stetigen Prozess, wohingegen Flexibilit�t punktuell und interimistisch wirkt\footnote{Vgl. \cite[S.2] {westkaemper2009}.} und nicht durch rahmengebende Zeit- und Kostenvorgaben einen Wandlungskorridor vorgibt.\footnote{Vgl. \cite[S.22]{seebacher2013ansaetze} als Interpretation von \cite[S.121]{Wiendahl2009}.}
Dar�ber hinaus ist das Leistungsspektrum der Flexibilit�t im Rahmen der beschriebenen Erwartungshaltung als Ma�nahme im Hinblick auf die antizipierten Ereignisse beschr�nkt, w�hrend z.B. Wandlungsf�higkeit auch im Voraus nicht spezifizierte Aufgaben in Angriff nehmen muss.\footnote{Vgl. \cite[S.22]{seebacher2013ansaetze}.}
Elastizit�t meint im produktionswirtschaftlichen Sinn vor allem die F�higkeit, z�gig in einen Ursprungszustand zur�ckkehren zu k�nnen.\footnote{Vgl. \cite[S.973]{Oke2005}.}
Folglich setzt produktionswirtschaftliche Flexibilit�t diese Elastizit�t voraus.\footnote{Vgl. \cite[S.23]{seebacher2013ansaetze}.}
Das Spektrum zutreffender charakterisierender Eigenschaften  ist in Tabelle \ref{tab:4} zusammengefasst.\\\\
Um einen zusammenfassenden Definitionsansatz zu erreichen, der gleicherma�en die morphologische Analyse von Seebacher und die Konstellation von Garrel et al. (s.o.) ber�cksichtigt, sind drei Definitionen kombinierbar.
Nach Garrel et al. ist ein Flexibilit�tsbedarf ,,eine durch Informationsdefizite bez�glich Ver�nderungen der System- und Umweltbedingungen entstehende Diskrepanz zwischen Ist- und Sollzustand des Systems, welcher bez�glich bestimmter Zielgr��en am effizientesten durch Aktivierung eines Flexibilit�tspotentials gedeckt werden kann''.\footnote{\cite[S.85]{vonGarrel2014}.}
Sie definieren das verbundene Flexibilit�tspotential als ,,die Gesamtheit aller vorhandenen Handlungsoptionen, welche genutzt werden k�nnen, um einen Flexibilit�tsbedarf befriedigen zu k�nnen''.\footnote{\cite[S.83]{vonGarrel2014}.}
Als Definition, die vor dem Hintergrund der bisher nicht konsensf�higen Begriffsdiskussion auch keinen Anspruch auf Allgemeing�ltigkeit darstellt, sondern vor allem formuliert, welches Ziel auch im konzeptionellen Ergebnis angestrebt wird, kann am Ende die darauf aufbauende Definition von Jeske et al. erweitert werden:
Flexibilit�t ist die F�higkeit ,,einer Organisation, sich an �ndernde organisationsinterne oder -externe Bedingungen''\footnote{\cite[S.21]{Jeske2011ErfolgsfaktorF}.} anzupassen ,,und zwar sowohl als Reaktion auf aktuellen Anpassungsbedarf als auch vorausschauend auf m�gliche zuk�nftige Anforderungen''\footnote{\cite[S.21]{Jeske2011ErfolgsfaktorF}.}, welche im Rahmen der im Unternehmensumfeld �blichen Ver�nderungen wahrscheinlich eintreten und zur Wettbewerbsf�higkeit notwendigerweise zu beherrschen sind.
%----Tabelle 
\begin{table}[H]
\setlength\extrarowheight{1pt} %etwas mehr Luft
\captionsetup{justification=centering} %Caption zentrieren
\centering
\includegraphics[width=0.7\textwidth]{eigenschaften_flexibilitaet.pdf}
\captionof{table}[Begriffsabgrenzung Flexibilit�t]{Begriffsabgrenzung Flexibilit�t\footnotemark}\label{tab:4}
\end{table}
\addtocounter{footnote}{+0}\footnotetext{Zusammenfassung der morphologischen Analyse von \cite[S.16-25]{seebacher2013ansaetze}.}
%----Tabelle 
\subsubsection{Betrachtungsgegenst�nde}
In der Literatur sind zahlreiche Ans�tze zu ermitteln, wobei vor allem das Abstraktionsniveau bei der Quantifizierung stark variiert.
Aufbauend auf der Flexibilit�tskonzeption in \ref{flexibilitaetskonzeption} k�nnen Implementationsziele f�r flexibilisierende Ma�nahmen identifiziert werden.
Dazu sind z.B. die Bereiche aus dem bereichsklassifizierenden Ansatz (vgl. \ref{bereichsklassifizierenderansatz}) einschl�gig.
Bei diesen bescheinigt Gottmann den Bereichen Beschaffung/Lieferanten, Anlagen und Produktionsprozesse, Personal und Kunden Einschl�gigkeit von Flexibilit�t.
%----Tabelle 
\begin{table}[H]
\setlength\extrarowheight{1pt} %etwas mehr Luft
\captionsetup{justification=centering} %Caption zentrieren
\centering
\includegraphics[width=0.9\textwidth]{gottmann_flexibilitaet.pdf}
\captionof{table}[Einsatzbereiche Flexibilit�t nach Gottmann]{Einsatzbereiche Flexibilit�t nach Gottmann\footnotemark}\label{tab:5}
\end{table}
\addtocounter{footnote}{+0}\footnotetext{Zusammenfassung der einschl�gigen Einsatzbereiche nach \cite[S.49-74]{Gottmann2019}.}
%----Tabelle 
\noindent Der Ansatz liefert zwar m�gliche Ans�tze zur Vertiefung wie die Flexibilisierung der Kapazitierung, allerdings bleibt z.B. die ,,Flexibili[sierung] von Lieferanten''\footnote{\cite[S.51]{Gottmann2019}.} als M�glichkeit zur Lagermengenreduktion ein eher diffuses Methodensurrogat ohne evidente Messungsans�tze.
Die personelle Fluktuation scheint zwar messbar, doch Gottmanns Interpretation als eindeutig antinomische Beziehung zwischen hoher Fluktuation und Personalflexibilit�t\footnote{Vgl. \cite[S.58]{Gottmann2019}.} ist zu stark situationsabh�ngig, um sie als substantielle Konzeptionsgrundlage zu betrachten. 
Letztlich sind es auch Details wie die Suggestion, die Entkopplung der Produktionsprozesse von Arbeitszeitmodellen als Flexibilit�tsma� zu beurteilen und deren Mangel an konkreten Methoden, die Gottmanns Modell weniger als ersch�pfende Methodik, sondern eher als Ideenportfolio f�r die Konzeption qualifizieren, zumal Gottmanns Interpretation von Flexibilit�t keinem einheitlichen Muster folgt.
Als Beispiel sei der Indikator von Sozialkompetenz genannt, deren Insuffizienz die Autorin in diesem Zusammenhang als Signal f�r notwendige Personalakquise interpretiert.\footnote{Vgl. \cite[S.58]{Gottmann2019}.}\\\\
 Tats�chlich sind in der Literaturrecherche bis zu 50 verschiedene Flexibilit�tsbegriffe bzw. Flexibilit�tsarten zu ermitteln, die m�gliche Betrachtungsgegenst�nde angeben\footnote{Vgl. \cite[S.121]{Meffert1999}; \\\cite[S.296ff]{Sethi1990} sowie \\\cite{vokurka}.} und dabei unter anderem auf Kernelemente des \gls{PPS}-Prozesses referenzieren.
 Nach der Analyse von Sethi/Sethi rekurrieren diese jedoch auf dieselben Themenbereiche, sodass eigentlich im produktionswirtschaftlichen Umfeld nur elf distinkte Flexibilit�tsarten zu bestimmen sind (vgl. Abbildung \ref{img:sethi_flexibilitaet}).\footnote{\cite[S.298-313]{Sethi1990}.}
\begin{figure}[H]
\centering
\includegraphics[width=1\linewidth]{sethi_flexibilitaet}
\captionof{figure}[Flexibilit�tsarten im System nach Sethi]{Flexibilit�tsarten im System nach Sethi\footnotemark}
\label{img:sethi_flexibilitaet}
\end{figure}
\addtocounter{footnote}{+0}\footnotetext{In Anlehnung an \cite[S.297]{Sethi1990}.}
\noindent F�r die in diesen Flexibilit�tsarten indizierten Betrachtungsgegenst�nde bzw. ,,Flexibilit�tstr�ger'' und abstraktere Ans�tze werden in folgenden Kapitel Ans�tze zur Messung untersucht.\\\\
Daneben, also abseits des Betrachtungsgegenstands, wird allerdings unter Ber�cksichtigung der Begriffscharakterisierung der Flexibilit�t (vgl. Tabelle \ref{tab:4}) eine Flexibilit�t hinsichtlich der Wirkungsintention �ber Dimensionen differenziert.
Dabei finden sich unterschiedlich stark differenzierende Auspr�gungsunterscheidungen.\footnote{Vgl. hierzu die umfassende Recherche von \cite[S.13-18]{horstmann2005}.}
Im Bezug auf das in \ref{Teilbereiche} und \ref{sec:aufgabenundziele} entwickelte dichotome Auspr�gungsverst�ndnis kann dieses auch hier beibehalten werden, zumal eine taktische Flexibilit�tsdimension h�ufig nur eine Graduierung ansonsten klar unterscheidbarer Eigenschaften bedeutet.\footnote{Vgl. hierzu die Recherche zu acht Unterscheidungsans�tzen von \cite[S.86]{vonGarrel2014}.}
%\begin{savenotes}
\begin{table}[H]
\setlength\extrarowheight{1pt} %etwas mehr Luft
\captionsetup{justification=centering} %Caption zentrieren
\centering
\scalebox{0.75}{
\begin{tabularx}{1.3 \textwidth}{|p{3.5cm}|p{2cm}|p{2cm}|X|}
\hline
\makecell[l]{Wirkungs-\\dimension} & \multicolumn{2}{|c|}{Merkmalsauspr�gung} & {Erl�uterung} \\ \hline
 \textbf{Ebene} & {strategisch} & {operativ} & {Die Wirkungsdimension bemisst sich an der Komplexit�t der Vorhersehbarkeit und am Grad der langfristigen Relevanz des Ma�nahmenerfolgs.\footnotemark} \\ \hline
 \textbf{Zeitraum} & {kurzfristig} & {langfristig} & {Der Zeitraum bezeichnet die Dauer zwischen Aktivierung bzw. Nutzung des Potentials und Inkrafttreten seiner Wirkung.\footnotemark 
 Die Dimension ist eng, wenn auch nicht untrennbar mit der Wirkungsebene verbunden (z.B. Rohstoffpreiskrise als kurzfristige strategische Herausforderung).\footnotemark} \\ \hline
 \textbf{Zeitpunkt} & {proaktiv} & {reaktiv} & {Der Zeitpunkt bezeichnet den zeitlichen Beginn der Nutzung des Flexibilit�tspotentials in Relation zu seiner ausl�senden Einflussgr��e (ex ante/ex post).\footnotemark} \\ \hline
 \textbf{Intention} & {offensiv} & {defensiv} & {Die Intention differenziert den Zeitpunkt der Ma�nahmenausl�sung im Bezug auf den Wettbewerb.\footnotemark 
 Zwar sind die Kombinationen ,,proaktiv-offensiv'' und ,,reaktiv-defensiv'' gel�ufig, doch nicht exklusiv g�ltig.\footnotemark} \\ \hline
 \textbf{Wirkungsweise} & {quantitativ} & {qualitativ} & {Die Wirkungsweise wird zwischen funktionalem und numerischem Charakter, also z.B. Qualifikations- gegen�ber Personalmengenver�nderung.\footnotemark} \\ \hline
 \textbf{Wirkungsfelder} & {extern} & {intern} & {Das Wirkungsfeld besteht im Kompromiss der Diversifizierung zur Reaktion auf Marktentwicklungen und einem stabilen Betrieb. Die externe Wirkung dient zur Positionierung eines Unternehmens im Markt und die interne Wirkung zur Gestaltung der Unternehmenspotentiale zu deren Erreichung.\footnotemark} \\ \hline
\end{tabularx}
}
\captionof{table}[Informationsversorgungsprozess]{Informationsversorgungsprozess\footnotemark}\label{tab:5}
\end{table}
\addtocounter{footnote}{-3}\footnotetext{Vgl. \cite[S.10-11]{JanEppink1978}.}
\addtocounter{footnote}{+1}\footnotetext{Vgl. \cite[S.72ff]{upton1994}.}
\addtocounter{footnote}{+1}\footnotetext{Vgl. \cite[S.31]{kunz2002} sowie \cite[S.14]{horstmann2005}.}
\addtocounter{footnote}{+1}\footnotetext{Vgl. \cite[S.73ff]{Evans1991}.}
\addtocounter{footnote}{+1}\footnotetext{Vgl. \cite[S.73ff]{Evans1991} sowie \cite[S.15]{horstmann2005}}
\addtocounter{footnote}{+1}\footnotetext{Vgl. \cite[S.15]{horstmann2005}.}
\addtocounter{footnote}{+1}\footnotetext{Vgl. \cite[S.1451-1453]{blyton1996}.}
\addtocounter{footnote}{+1}\footnotetext{Vgl. \cite[S138-139]{ansoff1975managing}.}
\subsubsection{Ans�tze zur Messung}\label{ansaetzezurmessung}
Die nach Sethi/Sethi identifizierten Flexibilit�tsarten sind die Ansatzpunkte, die gem�� der bisherigen Zusammenf�hrung ein m�glichst hinreichendes Bild hinsichtlich produktionswirtschaftlicher Flexibilit�t vermitteln sollen.
Neben diesen sollen zus�tzlich Messungsans�tze aus der Recherche von Bellmann et al. erg�nzt werden, um diese im Anschluss zu bewerten.\\\\
\textbf{Maschinenflexibilit�t} bezeichnet die Eigenschaft einer Maschine, unterschiedliche Operationen durchf�hren zu k�nnen, ohne R�stzeit zu ben�tigen oder -kosten zu verursachen.\footnote{Vgl. \cite[S.298]{Sethi1990}.}
Dadurch erm�glicht sie z.B. geringere Losgr��en\footnote{Vgl. \cite{ranta1988interactive}.}, h�here mittlere Maschinenauslastung und Lagermengenreduktion.\footnote{Vgl. \cite{hutchinson1984flexibility}.}\\
Ans�tze zur Messung sind:
\setcounter{equation}{0}
\begin{enumerate}
	\item als Absolutzahl von Operationen $o$ , die eine Maschine ausf�hren kann ohne eine definierte Grenze $k_{grenz}$ von R�stkosten $k_{r\ddot{u}st}$ (bzw. -zeit) zu ben�tigen\footnote{Vgl. \cite[S.299]{Sethi1990}.}
	\begin{equation}
		F_{maschine}=|\left\{o | k_{r\ddot{u}st}\leq k_{grenz}\right\}|
	\end{equation}
	\item als gewichtete Absolutzahl wie in 1. mit Gewichtung �ber die relative Relevanz der Operation\footnote{Vgl. \cite{brill1987measures}.}
	\begin{equation}
		F_{maschine}=\sum\limits_{ i=1 }^{ |\left\{o | k_{ruest}\leq k_{grenz}\right\}| }{ w_{i} \times o_{i}}
	\end{equation}
	\item als Verh�ltnis von r�stzeitbedingtem Stillstand zu produktivem Betrieb\footnote{Vgl. \cite{son1987economic}.}
	\begin{equation}
		F_{maschine}=\frac{t_{ruest}}{t_{prod}}
	\end{equation}
	\item als verbleibenden Wert einer Maschine f�r ein neues Produkt im Sinne der Obsoleszenzrate, wobei dieser Wert auch vom neuen Produkt abh�ngt\footnote{Vgl. \cite[S.300]{Sethi1990} als Interpretation von \cite{gustavsson1984flexibility} sowie \cite{lam1988measurement}.}
	\begin{equation}
		F_{maschine}=\frac{v_{rest}}{v_{invest}}
	\end{equation}
\end{enumerate}
\textbf{Materialflexibilit�t} bezeichnet die systemische F�higkeit, verschiedene Teilearten effizient im Produktionsweg zu positionieren und zu bearbeiten.\footnote{Vgl. \cite[S.300]{Sethi1990}.}
Dies erh�ht die Verf�gbarkeit von Maschinen, damit auch deren Auslastung und reduziert Durchlaufzeiten.\footnote{Vgl. \cite[S.300]{Sethi1990}.}\\
Ans�tze zur Messung sind:
\setcounter{equation}{0}
\begin{enumerate}
	\item als Anteil der m�glichen Wege durch das Produktionssystem im Vergleich zu einem universellen System\footnote{Vgl. zum universellen System: \cite{chatterjee1987planning}. Das universelle System bietet im Sinne einer Permutation ohne Wiederholungen eine vollst�ndig universelle Einsetzbarkeit.}
	\begin{equation}
		F_{material}=\frac{p_{moeglich}}{n_{maschine}!}
	\end{equation}
	\item auf Basis eines Maschinentypenrankings aufsteigend von F�rderbandanlagen zu autonomen Robotern\footnote{Vgl. \cite[S.5-7]{stecke1984variations}.}
\end{enumerate}
\textbf{Ablaufflexibilit�t} bezeichnet die F�higkeit eines Werkst�cks, in unterschiedlicher Verarbeitungsreihenfolge zu demselben Produkt verarbeitet zu werden.\footnote{Vgl. \cite[S.301]{Sethi1990}.}
Diese Eigenschaften beg�nstigen die Routingflexilibit�t und leiten sich dabei aus dem Design des Werkst�cks selbst ab, z.B. aus der Modularit�t.\footnote{Vgl. \cite[S.302]{Sethi1990}.}
Im Falle von z.B. Nicht-Verf�gbarkeit einer Maschine k�nnen Durchlaufzeiten reduziert werden.\footnote{Vgl. \cite[S.302]{Sethi1990}}\\
Ans�tze zur Messung sind:
\begin{enumerate}
	\item als Absolutzahl entwickelter Fertigungspl�ne\footnote{Vgl. \cite[S.302]{Sethi1990}. Dieser Ansatz scheint nicht besonders bem�ht, allerdings ist bei der Ablaufflexibilit�t vor allem der Einfluss auf die Systemflexibilit�t entscheidend.}
\end{enumerate}
\textbf{Prozessflexibilit�t} meint die Menge unterschiedlicher Werkst�cke, die ein System ohne gr��ere Konfigurations�nderungen verarbeiten kann.\footnote{Vgl. \cite[S.302]{Sethi1990}}\\
Diese F�higkeit erm�glicht es, auf Nachfrageschwankungen zu reagieren, da gleichzeitig verschiedene Produkte erzeugt werden k�nnen\footnote{Vgl. \cite[S.302]{Sethi1990}} und dadurch Losgr��en und Lagermengen zu reduzieren.\footnote{Vgl. \cite[S.S.114]{browne1984}}\\
Zur Messung bestehen zahlreiche Ans�tze:
\setcounter{equation}{0}
\begin{enumerate}
	\item als (offensichtliche) Absolutzahl gleichzeitig produzierbarer Produkte\footnote{Vgl. \cite[S.303]{Sethi1990}}
	\item als absolute Anzahl distinkter Teilst�ckarten (sofern z�hlbar) in der Menge gleichzeitig produzierbarer Produkte\footnote{Vgl. \cite[S.S.114]{browne1984} sowie \\\cite{Gerwin1987}}
	\item als Bandbreite der Gr��en, Formen etc. von gleichzeitig produzierbaren Produkten, sofern die Teile nicht z�hlbar sind\footnote{Vgl. \cite[S.303]{Sethi1990}}
	\item als Absolutzahl gleichzeitig produzierbarer Produkte $p$ mit dem Wert $v$ unter Beibehaltung der bestm�glichen Produktionsmenge in Relation zum optimalen Produkt $p_{opt}$
	\begin{equation}
		F_{prozess}=|\left\{p | n_{p} \times v_{p} = n_{p_{opt}} \times v_{p_{opt}} \right\}|
	\end{equation}
	\item als H�he der (Um-)R�stkosten zwischen unterschiedlichen Aufgaben des aktuellen Produktionsprogramms
	\item als zu maximierendes Verh�ltnis zwischen dem monet�r ausgedr�ckten Gesamterzeugnisvolumen $v$ (value) und Kosten der teilst�ckbearbeitungsbedingten Wartekosten $c_{wart}$ innerhalb einer Periode\footnote{Vgl. \cite{son1987economic}}
	\begin{equation}
		F_{prozess}=\frac{ v_{ges}}{c_{wart}}
	\end{equation}
	\item als relativ-wahrscheinlichen Anteil nicht produktionsf�higer Produkte $P_{n-prod}$ in einem System produktionsf�higer Produkte $P_{prod}$ auf Basis von deren auftragsbedingten Produktionsnotwendigkeitswahrscheinlichkeit $p_i$\footnote{Vgl. \cite[S.16]{buzacott1982}. Dieser Ansatz ist jedoch insofern problematisch, dass ohne Begrenzung der Produkte die Flexibilit�t zwangsl�ufig gegen 0 konvergiert, vgl. \cite{jaikumar1984}. Hinzu kommt die nicht praxistaugliche Messbarkeit von Auftragswahrscheinlichkeiten in dynamischen M�rkten.}
	\begin{equation}
		\frac{p_{n-prod}}{p_{n-prod}+p_{prod}}
				\times
		\sum\limits_{i=1}^{n_{P_{n-prod}}}{p_i}	
	\end{equation}
\end{enumerate}
\textbf{Routingflexibilit�t} bedeutet die F�higkeit eines Systems, ein Produkt auf unterschiedlichen Wegen im Produktionssystem herzustellen.\footnote{Vgl. \cite[S.305]{Sethi1990}}
Diese unterschiedlichen Wege k�nnen aus unterschiedlichen Maschinen oder T�tigkeiten oder einer anderen Reihenfolge der Operationen bestehen.\footnote{Vgl. \cite[S.305]{Sethi1990}}\\
Im Gegensatz zur Ablaufflexibilit�t bezieht sich diese also auf das Produktionssystem und nicht auf das Werkst�ck.
Routingflexibilit�t erm�glicht effizientere Terminierung aufgrund von gleichm��igerer Lastverteilung.\footnote{Vgl. \cite[S.306]{Sethi1990}}
Au�erdem kann das System auf Vorf�lle wie Maschinenausf�lle reagieren und seine Produktion beibehalten (ggf. reduzieren, aber nicht einstellen).\footnote{Vgl. \cite[S.306]{Sethi1990}}\\
Ans�tze zur Messung sind:
\setcounter{equation}{0}
\begin{enumerate}
	\item als absoluter Durchschnitt m�glicher Produktionswege �ber alle Produkte\footnote{Vgl. \cite{chatterjee1987planning} and \cite{chung1989}}
	\item als Anteil der m�glichen Wege $p$ durch das Produktionssystem zu zwischen Arbeitsstationen oder Maschinen m�glichen Verbindungen $n$\footnote{Vgl. \cite{carter1986}}
	\begin{equation}
		F_{routing}=\frac{p_{moeglich}}{\frac{n_{maschine} \times (n_{maschine} - 1)}{2}}
	\end{equation}
	\item als relative Durchlaufzeitreduktion bei der Nutzung dynamischer Produktionswege im Vergleich zu statischen Produktionswegen\footnote{Vgl. \cite{chung1989}, in der Praxis allerdings entweder sch�tzkalkulatorisch oder nur durch Alternativbetrieb zu erfassen und daher nicht sonderlich praxistauglich.}
	\item als relative Durchlaufzeiterh�hung beim Eintritt unerw�nschter Vorf�lle wie Maschinenausf�lle im Vergleich zum regul�ren Betrieb\footnote{Vgl. \cite{browne1984}. Dieser Ansatz wirkt validierend und bietet die praxistaugliche M�glichkeit der empirischen Validierung.}
	\begin{equation}
		F_{routing}=\frac{t_{opt}}{t_{ausfall} + t_{opt}}
	\end{equation}
\end{enumerate}
\textbf{Produktflexibilit�t} ist das Ma� f�r die Einfachheit, mit der Teile im Produktionsprozess durch andere ersetzt werden k�nnen.\footnote{Vgl. \cite[S.304]{Sethi1990}}
Diese �nderungen sind allerdings ausnahmslos mit Rekonfigurationsma�nahmen verbunden und dies stellt den Unterschied zur Prozessflexibilit�t dar.\footnote{Vgl. \cite[S.304]{Sethi1990}}\\
Durch diese Flexibilit�t wird dem Unternehmen eine h�here Innovationsf�higkeit erm�glicht, da neue Produktdesigns leichter in den Markt gebracht werden k�nnen.\footnote{Vgl. \cite{carter1986}}\\
Ans�tze zur Messung sind:
\setcounter{equation}{0}
\begin{enumerate}
	\item als absolute Kosten oder Zeit f�r die Ver�nderung der in der Produktion befindlichen Produktteilst�cke\footnote{Vgl. \cite{browne1984}}
	\item die Kosten bzw. Zeit wie in 1. in Relation zu den Gesamtkosten der Produktion\footnote{Vgl. \cite[S.305]{Sethi1990}}
	\item als Verh�ltnis des gesamten monet�ren Produktionsvolumens $v$ zu den gesamten R�stkosten $k_{r\ddot{u}st}$\footnote{Vgl. \cite{son1987economic}}
	\begin{equation}
		F_{prod} = \frac{k_{r\ddot{u}st}}{v}
	\end{equation}
	\item als Absolutzahl innerhalb einer Periode neu eingef�hrter Teile\footnote{Vgl. \cite{jaikumar1986}}
	\item als absoluter Wertzuwachs neuer Produkte, die im Produktionssystem mit einer definierten Kostengrenze an neuem Produktionsmaterial hergestellt werden k�nnen, wobei diese Grenze als Opportunit�tskosten f�r Nicht-Einf�hrung zu verstehen und �ber stochastische Modelle zu ermitteln ist\footnote{Vgl. \cite{jaikumar1984}}
\end{enumerate}
\textbf{Volumenflexibilit�t} bezeichnet die F�higkeit eines Produktionssystems, grundlegend verschiedene Durchsatzmengen zu unterst�tzen.\footnote{Vgl. \cite[S.307]{Sethi1990}}
Dabei werden allerdings nur tats�chlich machbare Volumina ber�cksichtigt.\\
Gerade zyklisches bzw. saisonales Gesch�ft wird dadurch beg�nstigt, auf Nachfrageschwankungen reagieren zu k�nnen und somit ist der Korridor unterst�tzter Volumina soweit wie m�glich zu maximieren.\footnote{Vgl. \cite[S.307]{Sethi1990}}\\
Ans�tze zur Messung sind:
\setcounter{equation}{0}
\begin{enumerate}
	\item als kleinste Produktionsmenge $v_{min}$ bei der das Produktionssystem mit den Fixkosten $k_{fix}$ und den variablen Kosten $k_{var}$ zum Verkaufspreis $p$ noch profitabel arbeitet\footnote{Vgl. \cite{browne1984}, wobei hier nur die praktische Untergrenze bestimmt wird.}
	\begin{equation}
		 F_{vol} = \frac{k_{fix}}{p - k_{var}}
	\end{equation}
	\item als naheliegende Generalisierung von 1. die Bestimmung der Gr��e Volumenkorridors, in dem das Produktionssystem profitabel arbeiten kann\footnote{Vgl. \cite[S.308]{Sethi1990}}
	\begin{equation}
		F_{vol} = v_{max} - v_{min}
	\end{equation}
	\item als Verh�ltnis durchschnittlicher Volumenschwankung zur Volumenobergrenze\footnote{Vgl. \cite{Gerwin1987}}
	\begin{equation}
		F_{vol}= \sqrt{\frac{\left(\displaystyle\sum_{i=n}^{m}\left({v_{ist_i+1} - v_{ist_i}}\right)^{2}\right) }{m-n}} / v_{max}
	\end{equation}
	\item als bin�res Erf�llungskriterium der Stabilit�t von Produktionskosten bei tats�chlich festgestellter Volumenschwankungen\footnote{Vgl. \cite{falkner1986}}. Hierbei ist die Funktionselastizit�t der Kostenfunktion entscheidend.
	Mit variablen Kosten $v$, Produktionsvolumen $V$ und den Fixkosten $f$ ergeben sich produktionsvolumenabh�ngige durchschnittliche Produktionskosten $\overline{k}(V) = \frac{f+v}{V}$.  
	Die Elastizit�t dieser Funktion berechnet sich zu
	\begin{equation}
		\epsilon(\overline{k}(V)) = \frac{\overline{k}'(V) \times V }{\overline{k}(V)} = -\frac{f}{f+v}
	\end{equation}
	Da die Elastizit�t demnach mit h�heren Fixkosten sinkt, ist ein hoher Anteil variabler Kosten als hohe Volumenflexibilit�t interpretierbar.
	\item als Verh�ltnis zwischen Leerlaufzeit und Produktivbetriebszeit als Kapazit�tsreserve zur Aktivierung\footnote{Vgl. \cite[S.308]{Sethi1990}, wobei dieser Ansatz die Schw�che aufweist, dass irrationale �berkapazit�t als hohe Flexibilit�t interpretiert werden muss. Das Ma� ist somit in seinem zeitabh�ngigen Verlauf zu betrachten.}
	\begin{equation}
		F_{vol} = \frac{t_{leer} - t_{wartung}}{t_{prod}}
	\end{equation}
\end{enumerate}
\textbf{Erweiterungsflexibilit�t} ist das Ma� f�r die Einfachheit, mit der Kapazit�ten oder F�higkeiten im Sinne anderer Flexibilit�tsarten aufgebaut werden k�nnen.\footnote{Vgl. \cite[S.309]{Sethi1990}}
Im Gegensatz zur Volumenflexibilit�t, die vor allem auf variable Bearbeitung von Bestandsm�rkten abzielt, ist die Intention der Erweiterungsflexibilit�t, die maximale Kapazit�t zu erh�hen und neue Technologien f�r neue M�rkte zu etablieren.\footnote{Vgl. \cite[S.309]{Sethi1990}}
Diese Flexibilit�t erm�glicht die sukzessive Adaption der Produktion bei Expansionsvorhaben und reduziert diesbez�gliche Implementationszeit und -kosten.\footnote{Vgl. \cite[S.309]{Sethi1990}}\\
Ans�tze zur Messung sind:
\setcounter{equation}{0}
\begin{enumerate}
	\item als das zu minimierende Verh�ltnis zwischen den Kosten zur Verdopplung eines Produktionssystemoutputs zu der urspr�nglichen Investition\footnote{Vgl. \cite{carter1986}}
	\begin{equation}
		F_{erweiter} = \frac{k_{doppel}}{k_{invest}}
	\end{equation}
\end{enumerate}
\textbf{Programmflexibilit�t} bezeichnet die F�higkeit eines Produktionssystems zum hinreichend langen autonomen Betrieb.\footnote{Vgl. \cite[S.310]{Sethi1990}}
Diese F�higkeit reduziert aufgrund der verbesserten R�stzeiten die Durchlaufzeit.
Autonome Lauff�higkeit geht h�ufig auch mit geringeren Toleranzen und h�herer Qualit�t einher.\footnote{Vgl. \cite[S.310]{Sethi1990}}
Ferner erh�ht sich der effektive Durchsatz des Produktionssystems.\\
Ans�tze zur Messung sind:
\setcounter{equation}{0}
\begin{enumerate}
	\item als das Verh�ltnis von autonomer Laufzeit (zweite und dritte Schicht) zu �berwachter Laufzeit (erste Schicht)\footnote{Vgl. \cite[S.299]{Sethi1990}}
	\begin{equation}
		F_{programm} = \frac{t_{auto}}{t_{manuell}}
	\end{equation}
\end{enumerate}
\textbf{Produktionsflexibilit�t} bezeichnet die Grundgesamtheit aller produktionsf�higen Produkte, die ohne gr��ere Investitionen im System hergestellt werden k�nnen.\footnote{Vgl. \cite[S.311]{Sethi1990}}
Dadurch grenzt sich die Produktionsflexibilit�t von der Produktflexibilit�t ab, da durchaus nennenswerte Rekonfiguration m�glich sein kann, solange Investitionen in Anlagen vermieden werden.
Sie ist vor allem in M�rkten, die eine hohe Frequenz von Neueinf�hrungen aufweisen zur Wettbewerbsf�higkeit relevant.\footnote{Vgl. \cite[S.311]{Sethi1990}}
Dar�ber hinaus wirkt die F�higkeit risikostreuend.\\
Ans�tze zur Messung sind:
\setcounter{equation}{0}
\begin{enumerate}
	\item als Absolutzahl produktionsf�higer Produkte bzw. \\Teilst�ckkombinationen\footnote{Vgl. \cite{chatterjee1984}}
\end{enumerate}
\textbf{Marktflexibilit�t} bezeichnet die F�higkeit eines Produktionssystems, sich an ein sich ver�nderndes Marktumfeld anzupassen.\footnote{Vgl. \cite[S.312]{Sethi1990}}
Diese Modifizierung bezieht sich auf die Rearrangierung vorhandener Produktionsfaktoren oder deren zweckm��ige Erweiterung.
Dadurch k�nnen sich Produktionssysteme auf die Anforderungen des Marktes risiko�rmer einstellen und Trends unter Umst�nden schneller adaptieren als der Wettbewerb.\\
Ans�tze zur Messung sind:
\begin{enumerate}
	\item als gewichteter Wert der Kosten zur Einf�hrung eines neuen Produktes, zur Erh�hung bzw. Verringerung eines Produktionsvolumens um einen definierten Umfang und zur Erh�hung der Produktionsgesamtkapazit�t\footnote{Vgl. \cite[S.313]{Sethi1990}}
	\item als kalkulatorische Lagerfehlbestandskosten oder verz�gerungsbedingte Produktionskostenver�nderungen\footnote{Vgl. \cite[S.313]{Sethi1990}}
\end{enumerate}
Neben diesen weitestgehend konkreten Messungsans�tzen existieren wie indiziert abstraktere Verfahren, die nach Meinung der jeweiligen Verfasser entweder zentrale Faktoren oder gesamtsystemische Flexibilit�t fokussieren.
Dahingehend sind die Auswirkungen zun�chst teilweise schwer in tats�chliche Flexibilit�tstr�ger zu �berf�hren, allerdings findet im folgenden Kapitel diesbez�glich eine strukturierte Analyse statt.
Diese abstrakten Ans�tze wurden von Bellmann et al. in �hnlicher Weise aggregiert wie die bereits behandelten Ans�tze, die von Sethi/Sethi zusammengestellt wurden.
Das Modell von Marschak/Nelson interpretiert Flexibilit�t von Entscheidungen z.B. als das Ma� der ,,Teilmengenbeziehung der Menge der nach der Anfangsentscheidung noch bestehenden Handlungsm�glichkeiten''.\footnote{\cite[S.230]{Bellmann2009} in Anlehnung an \cite[S.42ff]{Marschak1962}, wiederum zitiert nach \cite[S.98-99]{Pibernik2001}}
Tats�chlich ist dieser Ansatz aber auf z.B. die Ablaufflexibilit�t oder Routingflexibilit�t anzuwenden, bei denen Optionsvielfalt nach Entscheidungen konstituierend ist.
Andere Ans�tze betrachten den Gesamtwert einer Produktion in Varianten der Flexibilit�t und Inflexibilit�t.
Jacob nennt z.B. die Entwicklungsflexibilit�tsma�zahl als ,,Quotienten des Gewinns bei optimaler Anpassung bei prophetischem Wissen und dem Gewinn bei optimaler Anpassung entsprechend einer Entscheidung, jeweils vermindert um den Gewinn bei Nicht-Anpassung.''\footnote{\cite[S.230]{Bellmann2009} nach \cite[S.322ff]{jacob1974}}
Ein �hnlicher Ansatz von Hanssmann betrachtet strategische Flexibilit�t als ,,Quotient aus Gesamterfolg der Strategie und Gesamterfolg bei optimaler Anpassung, jeweils vermindert um den Gesamterfolg bei Inflexibilit�t''.\footnote{Vgl. \cite[S.231]{Bellmann2009} nach \cite[S.228ff]{hanssmann1978}}
Diese Modelle, angedeutet bei der Produktflexibilit�t, sind dabei immer mit Unsicherheiten konnotiert, insbesondere den Prognosen �ber das Marktverhalten und optimale Entscheidungen.
Eine vollst�ndige �bersicht aller von Bellmann et al. aggregierten Modelle findet sich in Anhang \ref{b1}.
\subsubsection{Bewertung der Ans�tze zur Messung}
Als letzte Ma�nahme vor der Adaption von Methoden auf die IT-Organisation sind die Messungsans�tze zu bewerten, um die Qualit�t dieser Grundlage zu analysieren.
Beim Vergleich der 19 von Bellmann et al. identifizierten Modelle und den von Sethi/Sethi vorgeschlagenen Messungsmethoden der genannten elf Flexibilit�tsarten lassen sich res�mierend prim�r drei verschiedene Herangehensweisen zur Modelldefinition und darauf aufbauende Beurteilungsma�st�be erkennen.\footnote{Vgl. hierzu \cite[S.233]{Bellmann2009}, die Klassifikationsmerkmale sollen allerdings hier anders interpretiert werden, da Kapazit�t auch als Indikator wirkt.}\\\\
Eine Herangehensweise bezieht sich auf produktionswirtschaftliche Indikatoren empirisch festzustellender Eigenschaften, die Flexibilit�t direkt quantifizieren.
Dazu z�hlen z.B. die Ans�tze von Carter und Chen/Chung, die Wege durch das Produktionssystem ermitteln und in Bezug zu maximal m�glichen Wegen setzen.
Auch F�higkeitsindikatoren wie die Anzahl gleichzeitig produzierbarer Produkte oder grenzkostenneutrale Maschinenoperationen entsprechen diesem Verfahrensansatz.\\\\
Der n�chste Ansatz sind Modelle, die abstrakt auf potentiellen Entscheidungen basieren bzw. systemische Interdependenzen von Handlungsoptionen insofern quantifizieren, als dass Optionsmengen verringernde Interdependenzen als Inflexibilit�t verstanden werden.
Diese Modelle, z.B. formuliert durch Marschak/Nelson\footnote{In �hnlicher Form allerdings auch zahlreiche andere, z.B. \cite{Gupta1968}, vgl. Anhang \ref{b1}}, setzen also deduktionsf�hige Kenntnisse �ber das Produktionssystem voraus, die sowohl technologische als auch materielle Einschr�nkungen ber�cksichtigen, also umfangreiche kombinatorische Modelle auf unterschiedliche Produktionsstufen bzw. Subsysteme anwenden und diese entweder quantitativ vergleichen oder die Subsysteme ordinal bewerten.\\\\
Daneben existiert noch die Herangehensweise �konomischer Bewertung.
Diese Modelle konstruieren in der Regel Aussagen �ber monet�re Gr��en, ausgedr�ckt �ber Funktionen, die diese Gr��en multifaktoriell beeinflussen, also verschiedene interne und externe Einflussgr��en ber�cksichtigen und in einem Wert wiedergeben.
Diese Aussagen werden dann entweder mit Werten optimaler Parametrierung verglichen, die Varianz verschiedener Parametrierungen begutachtet (niedrige Varianz bedeutet hohe Flexibilit�t) oder die Diskrepanz zur pessimalen Parametrierung bewertet.
Beispiele sind die abstrakten Modelle von Hannsmann\footnote{Vgl. \cite{hanssmann1978}}, Jacob\footnote{Vgl. \cite{jacob1974}} sowie von Jaikumar zur Produktflexibilit�t\footnote{Vgl. \cite{jaikumar1986}}, die Gewinne bei prophetischem Wissen, optimaler Strategie und Negativ-Szenarien einsch�tzen und vergleichen.
Diese Modelle sind insgesamt deutlich komplexer anzuwenden, da sie nicht von messbaren oder simulierbaren Eigenschaften ausgehen, sondern diffizile Beziehungen als scheinbar simpel und teilweise unifaktoriell (Strategie-Erfolgs-�quivalenz von Hannsmann) ausgelegt werden.
Sie trivialisieren daher insofern produktionswirtschaftliche Funktionalit�t oder sind zumindest aufgrund des hohen Abstraktionsniveaus nicht mehr praxistauglich, da ein Vergleich mit einer optimalen Strategie, welche sich ohnehin nur stochastisch ermitteln lassen, keine besonders konkreten Handlungsempfehlungen f�r einzelne Flexibilit�tstr�ger mehr zul�sst.\\\\
Bellmann et al. fixieren letztlich sieben Kriterien zur Bewertung von Messmodellen\footnote{Vgl. hier und im folgenden \cite[S.234-235]{Bellmann2009}, wobei Anwendungsaufwand tabellarisch nicht aufgef�hrt wird.}:
\begin{enumerate}
	\item Orientierung an realen Flexibilit�tstr�gern
	\item Beachtung von Teilflexibilit�ten
	\item Ausrichtung auf zuk�nftige Ver�nderungen
	\item Ber�cksichtigung einer stochastischen Umwelt
	\item Annahme rationaler Aktivit�ten
	\item Betrachtung mehrerer Perioden
	\item Anwendungsaufwand
\end{enumerate}
Tats�chlich scheinen insbesondere abstrakte Modelle den Bezug zu Implementationsans�tzen zu verlieren.\footnote{Belllmann et al. interpretieren diesen Umstand f�r die Ans�tze von Jacob, Hannsmann sowie Schneeiwei� \& K�hn zwar anders, bleiben aber eine Definition f�r dieses Kriterium schuldig, sodass die Einsch�tzung letztlich nicht nachvollziehbar wird.}
Auch der Aufwand besonders abstrakter Methoden gestaltet sich unpraktisch hoch.\footnote{Vgl. \cite[S.235]{Bellmann2009}, viele Methoden sind sogar praktisch weitestgehend unerprobt.}
Der Pr�misse, Steuerungsans�tze zu liefern, kommen viele Methoden also nicht nach, da sie monet�re, gesamtsystemische Ans�tze darstellen, die eher zur Unternehmensbewertung dienen k�nnen, als zur Initiierung von �nderungsvorhaben des Produktionssystems zu fungieren.\\\\
Flexibilit�t wird grunds�tzlich als vorteilhafte und den Unternehmenswert steigernde Eigenschaft wahrgenommen.\footnote{Vgl. \cite[S.236]{Bellmann2009}; \\\cite[S.280ff]{Burmann2003}; \cite[S.104]{vonGarrel2014}; \cite[S.127ff]{Bellmann2009} sowie \\\cite[S.56-66]{Moos2010}} 
Daher scheint es sinnvoll, vor allem reale Flexibilit�tstr�ger zu bewerten und sie nicht anhand von Annahmen �ber dadurch bedingte Entwicklungsm�glichkeiten zu beurteilen.
Dar�ber hinaus gibt es zahlreiche Flexibilisierungsma�nahmen, f�r die bei Ma�nahmenimplementation ein Erreichungsgrad gemessen werden kann.
Dieser Ansatz ist bisher in einschl�giger Literatur nicht zu identifizieren.\\\\
Eine abschlie�ende Beurteilung zur produktionswirtschaftlichen Flexibilit�t, die auch durch z.B. Bellmann et al. gest�tzt wird, ist, dass es bisher kaum bis keine praktisch etablierten, aussagekr�ftigen Beurteilungsmethoden gibt.
Ein integrierter Ansatz fehlt vollst�ndig.
Solch ein Ansatz w�re gerade vor dem Hintergrund der hierarchisch-symbiotischen Beziehung von Flexibilit�tsarten allerdings w�nschenswert.\footnote{Vgl. \cite[S.320]{Sethi1990}}
\subsection{Flexibilit\"at im Anwendungskontext der IT}
Obwohl Flexibilit�t als konzeptioneller Einflussfaktor wie beschrieben bisher im Bezug auf die betriebliche IT-Organisation kaum ber�cksichtigt wird, scheint es aufgrund des beschriebenen Einflusses naheliegend, dass positive Ergebnisse zumindest teilweise �bertragbar sind.
 Inwieweit allerdings die konstituierenden Mechanismen und Rahmenbedingungen dabei zu adaptieren oder zu interpretieren sind, soll nachfolgend gekl�rt werden.
% Die Produktionsfaktoren in der IT k�nnen nur teilweise in denen in der Produktion identifiziert werden.
% Zwar existiert eine technische Infrastruktur, doch diese kann nicht direkt wertsch�pfend eingesetzt werden, z.B. zur Erzeugung eines verkaufsf�higen Endprodukts, sondern ist in der Regel durch darauf aufbauende Software zur Leistungsbereitstellung zu einem wertsch�pfungsf�higen System zu erg�nzen. 
% Der letztlich wertsch�pfende Aspekt sind entweder darauf basierende Automation von wertsch�pfenden T�tigkeiten oder die wertsch�pfende Verwendung durch einen Anwender.
% In diesem Sinne sind technische Anlagen nicht direkt, sondern nur indirekt wertsch�pfend.
\subsubsection{Adaption der Flexibilit�tskonzeption}
Das in \ref{flexibilitaetskonzeption} entwickelte Begriffsverst�ndnis ist auf Kompatibilit�t zu der IT zu �berp�fen.
Hinsichtlich sowohl reaktiver als auch proaktiver Aktionsf�higkeit ist insofern keine Pr�ferenz zu gestalten, als dass die IT unabh�ngig von ihrer Stellung als interner Leistungserbringer, Innovator oder wie auch immer aufgefasster Rolle untrennbar mit den meisten gesch�ftlichen Aspekten verbunden ist\footnote{Vgl. \cite[S.16]{jorfi2011relationships}} und dadurch h�ufig strategische Relevanz in unterschiedlichsten Gesch�ftsbereichen hat.\footnote{Vgl. \cite[S.21]{Reinheimer2016}.}
Insofern muss die IT nicht nur flexibel auf Anforderungen reagieren k�nnen, sondern proaktiv strategische Flexibilit�tsbedarfe ermitteln und ber�cksichtigen.
Entscheidend zur Ma�nahmenkonzeption sind dabei die Aspekte der moderaten Unsicherheitsbeherrschung in Form zu erwartender �nderungsnotwendigkeiten in begrenztem Umfang, also die Definition auf einen klar umrissenen Anwendungsfall.
Insofern ist die aufgestellte Definition von Jeske et al. auch auf die IT und ihr Leistungsportfolio anzuwenden.\\\\
Weniger eindeutig sind allerdings die �nderungsimpulse, also die Ausrichtung auf externe und interne Ausl�ser.
Die Produktionsfaktoren in der IT k�nnen nur teilweise in denen in der Produktion identifiziert werden.
Zwar existiert eine technische Infrastruktur, doch diese kann nicht direkt wertsch�pfend eingesetzt werden, z.B. zur Erzeugung eines verkaufsf�higen Endprodukts, sondern ist in der Regel durch darauf aufbauende Software zur Leistungsbereitstellung zu einem wertsch�pfungsf�higen System zu erg�nzen. 
Der letztlich wertsch�pfende Aspekt ist entweder darauf basierende Automation von wertsch�pfenden T�tigkeiten oder die wertsch�pfende Verwendung durch einen Anwender.
In diesem Sinne sind technische Anlagen nicht direkt, sondern nur indirekt wertsch�pfend.
Gem�� Carrs These ,,IT doesn't matter''\footnote{Vgl. \cite[S.6]{carr2003doesn}} w�ren externe Impulse insoweit zu vernachl�ssigen, als dass die Leistungserbringung der IT als bin�res Kriterium zu werten w�re und, sofern die Leistungserbringung erfolgt, nur interne Impulse zu befriedigende Bedarfe indizieren k�nnen.\\
Die allerdings angesprochene Verstrickung mit dem Gesch�ft und daraus f�r die IT resultierende strategische Interdependenz l�sst sich allerdings erweitern auf den auch gesch�ftsmodellseitig steigenden Einfluss von IT, z.B. durch direkt IT-seitige Leistungserbringung f�r den Kunden in vielseitiger Form bis hin zum vollst�ndig digitalen Gesch�ftsmodell.\footnote{Vgl. \cite[S.2-3]{al2008defining}.}
Zus�tzlich k�nnen Innovationen daf�r sorgen, dass Flexibilit�tspotentiale leichter und effektiver implementiert werden, z.B. durch die Einf�hrung von neuer Software, die leichter auf betriebliche Anforderungen anzupassen ist.
Auch die Ber�cksichtigung externer Impulse ist daher sinnvoll und die entwickelte Definition letztlich soweit abstrakt, dass diese auch f�r die IT vollst�ndig �bernommen werden kann.%\\\\
\subsubsection{Adaption der Betrachtungsgegenst�nde}\label{adaptionbetrachtungsgegenstaende}
Die Adaption der Betrachtungsgegenst�nde dient dazu, Implementationsziele f�r flexibilisierende Ma�nahmen in der IT zu identifizieren.
Einerseits k�nnen diese �ber die beschriebenen Systematisierungsans�tze aus dem \gls{P-C} �bertragen werden.
Zus�tzlich soll allerdings �berpr�ft werden, ob damit die Betrachtungsgegenst�nde ersch�pfend identifiziert sind, oder sie erg�nzt werden m�ssen, um die Einsatzzwecke von Flexibilit�t in der IT exhaustiv darstellen zu k�nnen.\\\\
Der bereichsklassifizierende Ansatz (vgl. \ref{bereichsklassifizierenderansatz}) war bereits zuvor als zwar nicht ausreichend differenziert und inhaltlich teilweise unschl�ssig identifiziert worden, doch er indizierte zumindest ein umfangreiches Portfolio von Bereichen.
Diese k�nnen jeweils auf �bertragbarkeit gepr�ft werden.\\\\
\textbf{Beschaffung} hat in der IT nicht die gleiche Bedeutung wie in der Produktion.
W�hrend die Produktion operativ von durchgehend der Belieferung abh�ngig ist, da diese Quelle eines ma�geblichen Inputs ist, hat der Einkauf in der IT vor allem strategische Relevanz im Sinne langfristiger Investitionen.\footnote{Vgl. \cite[S.3]{luzzini2014organizing}.}
Der wenig in der Literatur betrachtete Aspekt der operativen Beschaffung ist unkritisch und im Sinne der Effizienz daher schlicht bestm�glich zu automatisieren.\footnote{Vgl. \cite[S.3]{luzzini2014organizing}.}
Die Flexibilisierungsans�tze von Gottmann zur Beschaffung zielen insgesamt vor allem auf die \gls{JIT-L}-Bef�higung, die f�r die IT keine nennenswerten Vorteile verspricht, da ein kontinuierlicher Materialfluss in der Regel nicht essentiell f�r den IT-Betrieb ist.\\\\
\textbf{Anlagen} sind wie beschrieben in der IT zwar in unterschiedlicher Form (Betriebsmittel der Endbenutzer wie PC und Peripherie, Rechenzentrum als zentrale Diensteplattform, Netzwerktechnik etc.) vorhanden, m�ssen aber anders als in der Produktion betrachtet werden.
Der Forderung von Variantenreichtum und Losgr��enminimierung aus der Produktion steht in der IT vor allem der Wunsch nach einer bestm�glichen Ausrichtung der IT an (sich ggf. �ndernden) betrieblichen Anforderungen gegen�ber, wobei die Geschwindigkeit dieses Vorgangs ma�geblich f�r die Erfolgsbetrachtung ist.\footnote{Vgl. \cite[S.2]{Frschle2016}, der Autor geht in diesem Zusammenhang auch auf den Tradeoff von Spezialisierung und Flexibilit�t ein.}
Diese Ziele beinhalten einerseits die kapazitive Reaktion der Infrastruktur auf wechselnde Leistungsanforderungen und andererseits inhaltliche und strukturelle Aspekte der Software, die Anpassungsf�higkeit und Erweiterbarkeit bedingen.
Diese beiden Betrachtungsgegenst�nde, Infrastruktur und Software, sind also als m�gliche Flexibilit�tstr�ger abzuleiten.
\\\\
\textbf{Personal} hat in der IT einen mindestens genauso entscheidenden Einfluss wie in der Produktion.
Auch mit fortschreitender Automation von T�tigkeiten sind, selbst unter der Annahme, dass perspektivisch nahezu jede T�tigkeit automatisiert werden kann\footnote{Vgl. \cite{susskind2015future}.}, in der IT Fachkr�fte n�tig, die diesen Prozess ausf�hren und begleiten.
Die Komplexit�t dieser Aufgaben wird in der Praxis von IT-Personal umgesetzt, das sich dabei immer st�rker aus gut ausgebildeten Fachkr�ften zusammensetzt.\footnote{Vgl. \cite[S.2]{Frschle2016}.}
Die Arbeit des Personals in der IT besteht letztlich aus wesentlich anderen T�tigkeiten als in der Produktion, wird aber durch �hnliche organisatorische Aspekte wie Fluktuation und Qualifikation konstituiert. 
Auch Arbeitszeitmodelle sind hier genauso auf das Personal zu attribuieren.
Letztlich ist also schon auf Basis der Indikatoren, die Gottmann nennt, die Flexibilit�tsbetrachtung von Personal auf die IT �bertragbar.
\\\\
F�r den Bereich \textbf{Kunden} beschreibt Gottmann die Flexibilit�t als Variantenreichtum und Durchlaufzeitreduktion.\footnote{Vgl. \cite[S.74]{Gottmann2019}.}
Diese Aspekte darin zu verorten scheint allerdings deplatziert.
Vielmehr w�re in der Kundenperspektive die flexible Anpassung am Markt zu verstehen, welche auf Einschl�gigkeit zu pr�fen w�re.
Variantenreichtum z.B. ist zwar ein Aspekt der Flexibilit�t, allerdings im bereichsklassifizierenden Modell nach Gottmann missverst�ndlich eingeordnet.\\\\
Eindeutiger ist in dieser Hinsicht der Ansatz von Sethi/Sethi, deren Flexibilit�tsarten diese Aspekte ber�cksichtigen und die daher in �hnlicher Weise auf �bertragbarkeit zu pr�fen sind.\\\\
\textbf{Maschinenflexibilit�t} ist wie beschrieben im Bezug auf Infrastruktur nicht insofern zu betrachten, als dass Maschinen, also z.B. PCs oder Server nicht der ma�geblich wertsch�pfende Bestandteil von IT-Systemen sind, sondern lediglich Anwender und Software eine Plattform geben.
Da sich Software auf den stark vereinheitlichten Hardware-Plattformen in der Regel �bergreifend ausf�hren l�sst\footnote{Die Prozessoren von AMD und Intel, zwischen denen sich der Markt f�r Server- und Desktop-Prozessoren aufteilt, arbeiten mit demselben Befehlssatz.}, mangelt es hier an der Diskrepanz zwischen diesbez�glicher Infrastruktur in der Operationenkompatibilit�t bzw. der Notwendigkeit zur Umr�stung der Infrastruktur, um bestimmte Software auszuf�hren.
Auch R�stzeiten sind bei dieser technischen Infrastruktur nicht nennenswert vorhanden, sofern von Anschaltvorg�ngen abgesehen wird, welche allerdings in Relation zum Betrieb marginal sind oder wie bei zentraler Rechenzentrumsinfrastruktur im Dauerbetrieb ebenfalls entfallen.\\\\
\textbf{Materialflexibilit�t} und \textbf{Ablaufflexibilit�t} entfallen aufgrund der fehlenden Abh�ngigkeit von Werkst�cken.\\\\
\textbf{Prozessflexibilit�t} geht ebenfalls von Werkst�cken aus, wobei hier Adaptionsm�glichkeiten zur Bereitstellung von IT-Diensten bestehen, auf die allerdings im Zusammenhang mit der Produktionsflexibilit�t genauer eingegangen werden soll.\\\\  
\textbf{Routingflexibilit�t} scheint ebenfalls schwer adaptierbar, da IT-Wertsch�pfung vor allem in Form von digitalen Inhalten und Informationen erfolgt, welche keine Abh�ngigkeit zur Maschinenreihenfolge haben.\\
Die Relevanz der Verarbeitungsreihenfolge von Informationen bzw. Daten wird zudem durch zahlreiche M�glichkeiten der verteilten und parallelen Verarbeitung weiter relativiert.\footnote{Vgl. \cite{Baun2015}.}
Lediglich die Gew�hrleistung des technischen Betriebs bei Ausfall von Komponenten scheint betrachtungsf�hig.
Dar�ber hinaus ist von technischer bzw. infrastruktureller Seite allerdings eine Routingflexibilit�t nicht adaptierbar.\\
Ein Ansatzpunkt sind allerdings IT-Projekte, deren Ablauf im klassisch plangetriebenen Ansatz von kaskadierenden Projektinhalten ausgeht, deren Reihenfolge sich wiederum durch die jeweilig dazwischen bestehenden und sich mit Projektdauer kumulierenden Interdependenzen ergibt.\footnote{Vgl. \cite{Madauss2017}.}
Die hierbei genutzten Methoden zur Planung und Pufferung von Abl�ufen sind in der Produktion ebenfalls vertreten und kompatibel einsetzbar, z.B. die Netzplantechnik.\footnote{Vgl. \cite{netzplan1976}.}
Agiles Projektmanagement verfolgt einen weniger starren Ansatz und zielt damit genau auf diese Problematik.\footnote{Vgl. \cite[S.3-15]{agile2017}.}
Routingflexibilit�t ist daher insofern auf Projekte (auch Softwareentwicklungsprojekte) zu adaptieren.
Mit Interpretation eines Projektes als Werkst�ck kann auch die Ablaufflexibilit�t darauf angewendet werden.\\ 
Ein weiterer Ansatzpunkt ist die betrieblich verwendete Software bzw. deren Interoperabilit�t mit verschiedenen betrieblichen Prozessen.\\\\ 
\textbf{Produktflexibilit�t} w�re aufgrund der Unabh�ngigkeit von Werkst�cken zu vernachl�ssigen.
Abstrakt betrachtet werden allerdings Bestandteile der Wertsch�pfungskette substituiert und dieser Vorgang wiederum kann auf Software bzw. Softwarebestandteile �bertragen werden, die aufgrund technischer oder prozessualer Anforderungen ersetzt werden.\\
�hnliche Vorg�nge k�nnen auch f�r Infrastrukturelemente betrachtet werden.\\
Bez�glich der Software bestehen ferner bilaterale Wechselwirkungen zu den darauf basierenden Gesch�ftsprozessen.
Eine �nderung im Gesch�ftsprozess bzw. die Ersetzung eines Prozessschrittes kann hierbei gleicherma�en die Anpassung einer Software bedeuten (z.B. bei monolithisch ausgelegter Software) oder die Reorchestrierung von Subdiensten.\\
F�r technische bzw. funktionale �nderungsanforderungen bez�glich der angesprochenen Elemente lassen sich also Prozesse und erneut Aspekte der Infrastruktur sowie der Software ableiten.\\\\
Die \textbf{Volumenflexibilit�t} kann nicht auf Volumina der erbrachten Leistung in Form der Unterst�tzung von Gesch�ftsprozessen �bertragen werden, da schlicht die tempor�re und insofern reversible Unterst�tzung von Gesch�ftsprozessen keinen Vorteil verspricht.
Der initiale Aufwand f�r diese Unterst�tzung in Form der Implementation oder alternativ Beschaffung und anschlie�ender Anpassung und Bereitstellung stellt kein Konstrukt dar, dessen Aufhebung kapazitive Reserven f�r andere Systeme schafft.\\
Von infrastruktureller Seite ist allerdings die Kapazit�t ein zu betrachtendes Kriterium.
Einerseits k�nnen z.B. projektbedingt h�here Kapazit�ten als normal ben�tigt werden, andererseits stellt die pr�ventive Schaffung irrational hoher Kapazit�tsreserven (z.B. durch �berdimensionierung des Rechenzentrums) bei �berm��iger Nicht-Inanspruchnahme eine Fehlinvestition dar.\\
Die Volumenflexibilit�t ist also vor allem auf die technische Infrastruktur zu �bertragen.\\\\
�hnlich verh�lt es sich mit der \textbf{Erweiterungsflexibilit�t}.
Da diese �hnliche Effekte wie die Volumenflexibilit�t erzielen soll, aber vor allem deutliche Kapazit�tserh�hungen mit �konomischen Synergiewirkungen anstrebt, sind haupts�chlich infrastrukturelle Erweiterungen Bestandteil diesbez�glicher Betrachtung.\\
Eine abstraktere Perspektive erm�glicht allerdings auch die Erweiterung von Software bzw. der Anwendungslandschaft zur Integration neuer Technologien (wie Data Science) oder deren Erweiterung bzw. Ver�nderung zur Unterst�tzung zus�tzlicher Anwendungsf�lle, z.B. Gesch�ftsprozessen.\\\\
Die \textbf{Programmflexibilit�t} ist diffiziler zu adaptieren. 
Einerseits ist der Betrieb von Infrastruktur - egal ob PC oder Server - isoliert betrachtet kein wertsch�pfender Vorgang und andererseits per Definition autonom, da diese Anlagen keinen beaufsichtigten Betrieb erfordern.
Automation ist allerdings ein Aspekt, der davon losgel�st f�r ganze Gesch�ftsprozesse betrachtet werden kann.
Die Automation kann dabei End-To-End-Prozesse wie automatisierte Bedarfsmeldung �ber Bestellung bis zur vollst�ndigen buchhalterischen Abwicklung bedeuten.\footnote{Vgl. \cite{doelle2017}.}\\
Ein weiterer Aspekt f�r den autonomen Betrieb der Infrastruktur kann im Wartungsaufwand identifiziert werden.
Zwar sind diese Systeme, sowohl zentral als auch dezentral autonom lauff�hig, m�ssen aber in der Regel technisch aktuell gehalten werden und erfordern somit administrative Verwaltung, die aufgrund des Routinecharakters ebenfalls Automatisierungspotential aufweist.
�ber die Programmflexibilit�t l�sst sich die IT-gest�tzte Automation von T�tigkeiten innerhalb und au�erhalb der IT als Betrachtungsgegenstand ermitteln.
\\\\
Die Tatsache, dass IT erst durch Bereitstellung und Nutzung von IT-Diensten wertsch�pfend wird, kommt auch bei der \textbf{Produktionsflexibilit�t} zum Tragen.
Der Wertsch�pfungsbeitrag durch die IT besteht vor allem in der optimal an fachbereichsspezifischen Anforderungen ausgerichteten Bereitstellung von IT-Diensten.\footnote{Vgl. \cite[S.8]{Reinheimer2016}}
Insofern kann dies auch als das ,,Produkt'' unternehmerischer IT verstanden werden.
Die IT-Dienste sind insofern also als Betrachtungsgegenstand zu identifizieren.
Dazu geh�ren auch die durch diese IT-Dienste gest�tzten Prozesse bzw. die Gesamtheit aller existierenden Gesch�ftsprozesse, also inklusive der nicht durch die IT gest�tzten Prozesse.
IT-gest�tzte Prozesse k�nnen dabei als produktionsf�hig betrachtet werden, w�hrend nicht IT-gest�tzte einem nicht befriedigtem Bedarf bzw. nicht bestehender Nachfrage entsprechen.\\\\
Die \textbf{Marktflexibilit�t} letztlich versucht, vorhandene Produktionsfaktoren so zu rearrangieren oder zu erweitern, dass ein Produktionsvolumen l�ngerfristig erh�ht wird oder neue Produkte, also neue IT-Dienste eingef�hrt werden.
Die Wiederverwendung von Bestandteilen der IT-Organisation (infrastrukturell oder virtuell) zu diesem Zweck oder die Erweiterung der Hardware- und Softwaresysteme zur Bereitstellung neuer IT-Dienste k�nnen insofern aus dieser F�higkeit als Betrachtungsgegenst�nde abgeleitet werden.\\\\ 
Mit diesen Erkenntnissen k�nnen nun die Betrachtungsgegenst�nde bzw. Implementationsziele f�r Flexibilisierungsma�nahmen festgelegt werden.
Unter Ber�cksichtigung der Ausf�hrungen von Garrel et al.\footnote{Siehe diesbez�glich \cite[S.105]{vonGarrel2014}.}, Byrd/Turner\footnote{Siehe diesbez�glich \cite[S.191]{byrdturner2000}.}, Gottmann\footnote{Siehe diesbez�glich \cite[S.49-75]{Gottmann2019}.}, Kempkes et al.\footnote{Siehe diesbez�glich \cite[S.58]{Kempkes2018}. An dieser Stelle sei darauf verwiesen, dass die Print- und Digitalversionen abweichende Seitenzahlen aufweisen und sich die Angaben auf die Printversion beziehen.}, Wiedenhofer\footnote{Siehe diesbez�glich \cite[S.10]{Wiedenhofer2013}.} sowie Winkler/Sobernig\footnote{Siehe diesbez�glich \cite[S.7]{winkler2008}} k�nnen diesbez�glich Indikatoren �bernommen, adaptiert oder vorgeschlagen werden (dargestellt in Tabelle \ref{tab:indikatoren}).
\begin{table}[H]
\setlength\extrarowheight{1pt} %etwas mehr Luft
\captionsetup{justification=centering} %Caption zentrieren
\centering
\includegraphics[width=\textwidth]{indikatoren.pdf}
\captionof{table}[Betrachtungsgegenst�nde und Indikatoren in der IT]{Betrachtungsgegenst�nde und Indikatoren in der IT\footnotemark}\label{tab:indikatoren}
\end{table}
\addtocounter{footnote}{+0}\footnotetext{Eigene Darstellung}
F�r diese k�nnen nun  Messungsans�tze definiert werden.
%Insgesamt ist festzuhalten, dass der Reihenfolgen- bzw. Belegungsaspekt aus der Produktion in der IT nicht dieselbe Bedeutung hat wie in der Produktion bzw. im \gls{PPS}-Prozess.
%Die Terminierung von Auftr�gen ist insofern in der IT auch zu vernachl�ssigen, da IT-Dienste entweder einmalig implementiert werden und fortan genutzt werden k�nnen oder Termine im Rahmen von Projekten festgelegt werden und somit der Fokus nicht auf einem Auftrag, sondern einem Projekt liegt.




%Ma�nahmen - Projekte
%Ma�nahmen - Modularit�t etc.

%In diesem Zusammenhang ist Software insofern Flexibilit�tstr�ger, als das Prozesse nicht nur von monolithischer Software abgewickelt werden k�nnen, sondern einzelne Softwarebestandteile wiederverwendet werdenn.
\newpage
\subsubsection{Adaption von Ans�tzen zur Messung}
Messungsans�tze wie von Sethi/Sethi vorgeschlagen erlauben vor allem die Zustandsbewertung des Flexibilit�tsaspekts, z.B. die Anzahl m�glicher Wege durch das Produktionssystem.
Die Ans�tze sind im Prinzip einer Bestandsaufnahme sinnvoll, sie besitzen allerdings nicht immer direktes Steuerungpotential, geben also nicht direkt Ma�nahmen zur Steuerung vor (zumindest insofern, als dass ,,Erweiterung der Wege durch die Produktion'' keine umsetzungstaugliche Vorgabe ist) und die Wertbeitragsbemessung davon entkoppelt wird.\\
Wo m�glich, soll der Integrationsansatz, der im \gls{P-C} bisher vermisst wird, dadurch transportiert werden, dass der Implementationsgrad von Flexibilisierungsma�nahmen, die jeweils einen oder mehrere Indikatoren betreffen, identifiziert und auf Effizienz- oder Effektivit�tsma�e, die dadurch beeinflusst werden, projiziert wird.
Diese Kombination gibt Aufschluss �ber die Auswirkungen der Ma�nahmen und erm�glicht so auch die Beurteilung im zeitlichen Verlauf, also dar�ber, ob ein steigendes Erreichungsma� auch einen steigenden Wertbeitrag bedingt.\\\\
Beim \textbf{Personal} stehen Indikatoren wie Qualifikation und Einsetzbarkeit im Sinne von Universalit�t in Verbindung. 
Hoher Qualifikationsgrad kann auf die F�higkeit, verschiedene Aufgaben in der IT �bernehmen zu k�nnen, �bertragen werden.\footnote{Extremf�lle mit hoher Spezialisierung bilden hier ggf. Ausnahmen, wenn dieses Personal f�r sehr spezifische Aufgaben eingesetzt wird.}
Als Ma�nahmen diesbez�glicher Flexibilisierung bestehen z.B. M�glichkeiten in der gezielten Personalentwicklung\footnote{Vgl. \cite[S.37-72]{vonSchneyder2007}.} sowie Job-Enrichment und Job-Rotation\footnote{Vgl. \cite[S.111]{vonGarrel2014}.}.
Insgesamt ist das Ziel der Personalflexibilit�t in dieser Hinsicht daher, einzelne T�tigkeiten in der IT nicht auf Schl�sselpersonen zu konzentrieren, sondern das Know-How z.B. f�r Krankheits- und K�ndigungsf�lle im Personal zu verteilen.
\\
Die Arbeitszeit kann �ber Ma�nahmen wie Gleitzeit (z.B. mit Zeitkonten) und Teilzeit insofern flexibilisiert werden, dass Dauer und Lage der Arbeitszeit anders als in statischem Einschicht-Betrieb verschoben werden. 
Die dem Personal dadurch erm�glichte Freiheit kann zudem Auswirkungen auf die Motivation und somit auf die Produktivit�t haben.\footnote{Vgl. \cite[S.556]{Haemmerle2017}.}
\\
Ebenfalls in Verbindung mit der Einsetzbarkeit steht die Fluktuation, die in ausgepr�gter Form daf�r sorgen kann, dass Personal zu wenig Betriebskenntnis hat und au�erdem anteilig zu viel Zeit auf Einarbeitung entf�llt.
Hierbei besteht au�erdem der mit der Qualifikation in Verbindung stehende Effekt, dass Know-How zwar langfristig aufgebaut werden muss, aber schnell durch Abwanderung verloren gehen kann.
\\
Die Mobilit�t in Form der Unabh�ngigkeit von einem festen B�roarbeitsplatz versetzt Personal vor allem in die Lage, zu jeder Zeit ohne Ortswechsel bestenfalls alle f�r sie notwendigen T�tigkeiten ausf�hren zu k�nnen.
Dar�ber hinaus kann f�r ein Unternehmen auch die Notwendigkeit zum Aufbau der B�rokapazit�ten entfallen.
\\
Personalkapazit�t kann insofern flexibilisiert werden, als dass Personal auftragsabh�ngig variabel eingesetzt werden kann, also quantitative, zeitliche oder intensit�tsseitige Anpassungen m�glich sind.\footnote{Vgl. \cite[S.552]{Haemmerle2017}.}
Diese M�glichkeit besteht z.B. im Personalleasing.\footnote{Vgl. \cite[S.47]{Krueger2018}.}
Eine weitere M�glichkeit stellt der projekt-, t�tigkeits- oder zeitspezifische Einsatz von Dienstleistungsunternehmen, z.B. f�r Systemwechsel, f�r die nicht fristgerecht interne Kapazit�ten mit entsprechender Qualifikation aufgebaut werden k�nnen, also tempor�res Outsourcing dar.
Trivialere Varianten sind �berstundenaufbau und -abbau.
Zur Arbeitszeit bestehen insofern Verbindungen, die in der Praxis stark vertreten sind.\footnote{Vgl. hierzu die FlexPro-Studie: \cite[S.552]{Haemmerle2017}.}\\\\
%----Messung
Ans�tze zur Beurteilung des Erreichungsgrades der Flexibilisierung sind:
\setcounter{equation}{0}
\begin{enumerate}
	\item als gesamtpersonelle qualifikatorische Ersetzungsf�higkeit im Gruppen-\\Flexibilit�tsgrad\footnote{Vgl. \cite[S.56]{grothe1992} nach \cite[S.200]{vonSchneyder2007}.}\\
		\begin{equation}
			F_{gruppe}= \frac{\sum\limits_{ i=1 }^{ n }{ ers(i) }}{n(i)-(n(i)-nM(i))}
		\end{equation}
		Wobei $ers(i)$ der Anteil der Mitarbeiter ist, der den Mitarbeiter $i$ qualifikatorisch ersetzen kann, $n(i)$ die Anzahl der Mitarbeiter und $nM(i)$ der betrachtete Mitarbeiter, der ersetzt werden soll.
	\item als Anteil der Mitarbeiter in mitgestaltbaren, von statischer Einschicht-Voll\-zeit\-t�\-tig\-keit abweichenden Arbeitszeitverh�ltnissen $n_{flex}$ zu denen, die sich in letzterem Arbeitszeitverh�ltnis befinden $n_{statisch}$\footnote{Vgl. \cite[S.1, Kennzahl 1]{bmf2020}.}
		\begin{equation}
			F_{zeit}= \frac{n_{flex}}{n_{flex}+n_{statisch}}
		\end{equation}
	\item als Anteil der Mitarbeiter die technologisch zur Telearbeit bef�higt sind $n_{mobil}$ zu denen, die einen festen B�roarbeitsplatz ben�tigen $n_{fest}$\footnote{Vgl. \cite[S.1, Kennzahl 2]{bmf2020}.}
		\begin{equation}
			F_{mobilitaet}= \frac{n_{mobil}}{n_{mobil}+n_{fest}}
		\end{equation}
	\item als Fluktuationsrate einer Periode $p$ mit der Anzahl der aufgel�sten Arbeitsverh�ltnisse $n_{abgang}$ und neu geschlossenen Arbeitsverh�ltnisse $n_{zugang}$ zu verbliebenen Mitarbeitern\footnote{,,Schl�ter-Formel'', vgl. z.B. \cite[S.198]{vonSchneyder2007}. Es existieren zwar auch andere Berechnungsmethoden, aber ohne Periodenbezug.}
		\begin{equation}
			F_{fluk}= \frac{n_{abgang}(p)}{n_{zugang}(t)+n_{bleib}(p)}
		\end{equation}
	\item als zu realisierender Kapazit�tskorridor einer Periode $p$ im Verh�ltnis zur regul�ren Kapazit�t $K_{std}$, wobei die vollst�ndige Aktivierung von Ressourcen des Personalleasings und Dienstleistern sowie maximale Aussch�pfung von �berstunden die Obergrenze und deren vollst�ndige Deaktivierung sowie gr��tm�glicher �berstundenabbau die Untergrenze markieren
	\begin{equation}
			F_{kap}= \frac{(K_{std}+K_{dl}+K_{lease}+K_{ueber}) - ({K_{std}-K_{unter})}}{K_{std}}
		\end{equation}
\end{enumerate}
\setcounter{equation}{0}
%----Wertbeitrag
Diesen Kennzahlen, die eine Flexibilisierung des Personals andeuten, k�nnen wie anfangs angedeutet Effizienz- oder Effektivit�tskennzahlen gegen�bergestellt werden, die das Wertsch�pfungsma� der Ma�nahmen anzeigen.
Im zeitlichen Verlauf ist damit m�glich, zu beurteilen, ob diese Flexibilisierung unternehmerisch zielf�hrend war.
\begin{enumerate}
	\item Da bei Arbeitsortmobilit�t und flexiblen Arbeitszeitmodellen Implikationen zur Mitarbeiterproduktivit�t bestehen, 
	die bzgl. Arbeitszeitflexibilit�t positive Effekte indizieren\footnote{Vgl. \cite{garnero2014part} und \\ \cite{kunn2013part}}, f�r Arbeit abseits des B�ros allerdings gemischte Resultate hervorbringen\footnote{\cite{hill2003does}}, sollte ein zeitabh�ngiges Produktivit�tsma� bestimmt werden - eine Produktivit�tsbewertung durch Vorgesetzte als ein zwar vages, daf�r leicht realisierbares Ma� stellt hier situationsunabh�ngig ein stets praxistaugliches Minimum dar - und die Produktivit�t $P$ betroffener Mitarbeiter gegen�ber den anderen ermittelt werden, um festzustellen, dass eine Fortf�hrung der Ma�nahmen empfehlenswert ist.
		\begin{equation}\tag{W: (2) + (3)}
		P_{rel}	= \frac{\int\limits_{t_0}^{t_n}P_{flex}(t) dt}{\int\limits_{t_0}^{t_n}P_{statisch}(t) dt}
		\end{equation}
	\item Die Steigerung der Gruppenflexibilit�t sollte mit der Reduktion von krankheits- und fluktuationsbedingten Ausfallkosten $K_{ausfall}$ verbunden sein. Hierbei ist eine Steigerung (also ein Ma� $> 1$) der Effektivit�t der Gruppenflexibilit�t $E_{F_{gruppe}}$ zwischen zwei Perioden $t_1$ und $t_2$ anzustreben.
	\begin{equation}\tag{W: (1) + (4)}
			E_{F_{gruppe}} = \frac{F_{gruppe}(t_2)}{F_{gruppe}(t_1)}/\frac{K_{ausfall}(t_2)}{K_{ausfall}(t_1)}
		\end{equation}
	\item Zur Fluktuation existieren dar�ber hinaus verschiedene Varianten der Fluktuationskosten pro Mitarbeiter sowie den gesamten Fluktuationskosten, die im zeitlichen Verlauf betrachtet werden k�nnen.\footnote{Siehe hierzu \cite[S.199-200]{vonSchneyder2007}}  
\end{enumerate}
%----Ende
Flexibilit�t der \textbf{Infrastruktur} kann in sofern �ber Redundanz aufgebaut werden, als dass Betriebsausf�lle (bzw. zumindest Nicht-Verf�gbarkeit von IT-Diensten) nicht zwingend die Folge von infrastrukturellen Teil-Ausf�llen sein m�ssen.
Diesbez�gliche Redundanz kann also �ber die funktionale Entkopplung der Software und IT-Dienste von Infrastrukturelementen erreicht werden, was im Regelfall eine mehrfache Bereitstellung dieser bedeutet.
Konkret manifestiert sich dies in der Bereitstellung von Infrastruktur �ber das Mindestma� hinaus, z.B. in Form von Servern, Datenspeichern, Datenleitungen oder Netzwerkkomponenten bis hin zum Betrieb multipler Rechenzentren.\footnote{Vgl. \cite[S.160-194]{oracle2017}}
%Wertsch�pfung
Den durch Aufbau der Redundanz entstehenden Kosten stehen die Opportunit�tserl�se f�r besagte verhinderte Ausf�lle gegen�ber.
\\
Damit in Verbindung steht die Kapazit�tsflexibilit�t, da die in der Redundanz aufgebaute �berkapazit�t zumindest theoretisch zur Verf�gung steht.
Praktisch ist diese allerdings dadurch gebunden, dass sie im Fehlerfall zur �bernahme der betriebskritischen IT-Dienste dienen muss.\\
Kapazit�tsflexibilit�t in der Infrastruktur besteht also darin, auf erh�hten und verringerten Leistungsbedarf reagieren zu k�nnen.
Diese Reaktionsf�higkeit ist z.B. bei Wachstum durch Expansion oder Unternehmensk�ufe hilfreich, aber auch bei auftragsbedingten Leistungsbedarfsspitzen, die durch damit verbundene Berechnungs- oder Datenhaltungsanforderungen hervorgerufen werden k�nnen.
Dahingehend entstehen im Gegenzug Leistungsbedarfstiefs, wenn diese Anforderungen tempor�r aus betrieblichen Gr�nden nicht vorliegen.
\\
M�glichkeiten der kapazitiven Flexibilisierung bestehen wie in der Produktion also in dem pr�ventiven Aufbau von �berkapazit�t durch Infrastruktur im Eigenbetrieb.
IT-Infrastruktur ist hinsichtlich Skalierbarkeit flexibler Kapazit�t allerdings in �hnlicher Weise wie produktionswirtschaftliche Kapazit�t zu behandeln, darf also auch nicht mit irrationaler �berkapazit�t realisiert werden.
Kapazit�t kann aber z.B. �ber Cloud-Dienstleister bezogen werden\footnote{Vgl. \cite{Krintz2018}.} und erzeugt dabei im Pay-As-You-Go-Modell variable Kosten mit schnell reversiblen Verbindlichkeitsverh�ltnissen.\footnote{Vgl. \cite{chang2012}} 
Eine �konomisch sinnvolle Variante besteht also in der angemessenen, d.h. kostenmixoptimalen Nutzung solcher flexibel nutzbaren externen Ressourcen bei gleichzeitig zu maximierender Auslastung der internen Ressourcen.\\
Die dar�ber hinaus in Verbindung stehende Erweiterbarkeit zielt auf die langfristige Erh�hung der internen Ressourcen.
Sie soll m�glichst granular, also inkrementell ausfallen.\\
Diese F�higkeit erm�glicht, erweiterte Kapazit�tsbedarfe nicht dauerhaft �ber gemietete Fremdinfrastruktur abdecken zu m�ssen, sondern m�glichst exakt bedarfsdeckend eine �konomischere Alternative im Eigenbetrieb aufzubauen.
Kapazit�tserweiterungen sollten also m�glichst in der H�he der Kapazit�t eines einzelnen Infrastrukturelements erg�nzt werden k�nnen. 
Den pessimalen Fall dieser Erweiterung stellt die Kapazit�tsverdopplung als Minimum des Erweiterungsfalls dar.\\
%Automation?
%----Messung
Ans�tze zur Messung dieser Flexibilit�tsaspekte sind:
\setcounter{equation}{0}
\begin{enumerate}
	\item als Verh�ltnis operativ festgestellter und vom System unterst�tzter Schwankungen der Kapazit�tsauslastung $K_{prod}$ zur maximal m�glichen Auslastung $K_{max}$\footnote{Solche Daten sind in der IT �blicher Weise durch Monitoring-Systeme erfasst.}
	\begin{equation}
		F_{kap}= \sqrt{\frac{\left(\displaystyle\sum_{i=n}^{m}\left({K_{prod_i+1} - K_{prod_i}}\right)^{2}\right) }{m-n}} / K_{max}
	\end{equation}
	\item als Verh�ltnis der Infrastrukturkomponenten, die ausfallen m�ssen bis die Lauff�higkeit eingeschr�nkt wird zu der Gesamtanzahl Infrastrukturkomponenten, die zur IT-Dienst-Bereitstellung dienen
	\begin{equation}
		F_{red}= \frac{n_{ausfall}}{n_{gesamt}}
	\end{equation}
	\item als die minimale Kapazit�tserweiterung $E_{min}$ im Verh�ltnis zur durchschnittlichen Kapazit�t $K$ einer der bestehenden Infrastukturkomponenten mit deren Anzahl $n$
	\begin{equation}
		F_{erweiter}= \frac{E_{min}}{K/n}
	\end{equation}
\end{enumerate}
%----Wertbeitrag
Die Wertbeitragsbemessung dieser Ma�nahmen kann ebenfalls verfolgt werden:
\begin{enumerate}
	\item als die H�he der durch Redundanzf�lle verhinderten Ausf�lle erzeugten Opportunit�tserl�se in Verbindung mit dem ansonsten zu erwartenden Umsatz $U(t)$ im Zeitraum $t$ mit der Dauer der Redundanzf�lle $t_{red}$
	\begin{equation}\tag{W: (1)}
		E_{red}= \frac{t_{red}}{t} \times U(t)
	\end{equation}
\end{enumerate}
%----Ende
Flexibilit�t von \textbf{Software} ist vorwiegend als mittelbare Flexibilit�t zu verstehen, da erst durch funktionale Orchestrierung IT-Dienste zustande kommen, welche dann das Produkt der IT darstellen.
Intrinsische Flexibilit�t der Software hat also vor allem zum Ziel, diesen Vorgang der Orchestrierung durch inh�rente Simplizit�t, Adaptivit�t (strukturelle Ver�nderbarkeit und inhaltliche Konfigurierbarkeit) und Wiederverwendbarkeit (Universalit�t) zu st�tzen und so �konomisch zu gestalten, indem daf�r weniger Zeit aufgewendet werden muss und �nderungen folglich auch schneller den Endabnehmer erreichen.\\
Ver�nderungen der Software k�nnen jedoch nicht nur in der Entwicklung, sondern auch zur Laufzeit (in der Regel als Konfigurierbarkeit interpretiert) betrachtet werden. 
Diese ist jedoch nur ma�voll empfehlenswert, da andernfalls Systeme dadurch selbst notwendige Einschr�nkungen verfehlen\footnote{Vgl. \cite[S.211-217]{hruschka2018}.}, sodass zur Erreichung der gew�nschten Effekte vor allem die entwicklungsseitigen Aspekte zu betrachten sind.
\\
Komplexit�t von Anpassungen besteht vor allem in deren Abh�ngigkeiten und der dadurch bestehenden Aufwandskaskade.\\
Eine M�glichkeit besteht in der Programmierung durch Reduktion des Kopplungsgrades und gleichzeitiger Erh�hung der Koh�sion\footnote{Vgl. \cite[S.197-238]{Kappel1996}}, zielt also auf die innere Architektur der Software.
Dazu bestehen zahlreiche Varianten in der Ausf�hrung von Architekturstilen und der Verwendung entsprechender Entwurfsmuster.\footnote{Vgl. \cite{Eden2006MeasuringSF}.}
Eine andere M�glichkeit besteht in der Modularisierung, also in der �u�eren Softwarearchitektur.
Technisch voneinander getrennte, funktional stark begrenzte, aber daf�r vielf�ltig einsetzbare Anwendungen, die untereinander kommunizieren, zielen darauf ab.\footnote{Vgl. \cite[S.31-35]{wolff2018microservices}.}
Die Herstellung solcher Microservice-Architekturen, die Funktionen granular segmentieren, simplifizieren demnach auch die Ersetzung einzelner Funktionen im System sowie die Erweiterung durch zus�tzliche Services.\footnote{Vgl. \cite[S.35]{wolff2018microservices}.}
\\
Durch weniger spezifisch, folglich universeller einsetzbare, kleine Software wird auch der Indikator der Wiederverwendbarkeit ber�cksichtigt.
Wie beschrieben entsteht der IT-Dienst durch die Orchestrierung kleiner Services.
Im Rahmen der Segmentierung muss der Fokus also darauf liegen, die einzelnen Anwendungen so zu designen, dass die jeweilige Funktion nicht nur einmalig zu verwenden und damit zu anwendungsfallspezifisch ist.
Planungsgem�� ausgef�hrt entstehen dabei wiederverwendbare Anwendungen, deren Funktion auf weitere Anwendungsf�lle �bertragbar ist.
\\
Die Bereitstellungszeit von Software kann in Abh�ngigkeit davon betrachtet werden.
Da Software allerdings in der Regel im Rahmen von Projekten implementiert wird und in deren Peripherie weitere Einflussfaktoren existieren, wird dieser Umstand im Rahmen der letzten Dimension, der Projekte, gew�rdigt.
%Monolithen?
%----Messung
Ans�tze zur Messung dieser Flexibilit�tsaspekte sind:
\setcounter{equation}{0}
\begin{enumerate}
	\item als durchschnittlichen Wiederverwendungsgrad von Softwareelementen $S$ innerhalb der IT-Dienste mit der Anzahl $n_{dienst}$. Diese Kennzahl honoriert sowohl erfolgreich wiederverwendete monolitische Software als auch zielf�hrend ausgelegte Modularit�t.
	\begin{equation}
		F_{wieder} = \frac{\sum\limits_{ i=1 }^{ n_{Dienst} }{\bigl|\{S_i | S_i \in Dienst \}\bigl|  }}{n_{Dienst}} 
	\end{equation}
	\item als Vernetzungsgrad unterschiedlicher Software mit deren Gesamtanzahl $n_S$ �ber die Gesamtheit der IT-Dienste hinsichtlich deren Kommunikation mit den Verbindungen $v$; die Kennzahl bestraft insbesondere isolierte, monolitische Software in gr��eren Anwendungslandschaften und honoriert dagegen umfangreiches Schnittstellendesign bzw. dessen ausgiebige Nutzung.
		Hierbei sind Maximalwerte zu definieren, um nicht der Illusion zu verfallen, dass jedes System tats�chlich eine Verbindung zu jedem anderen System ben�tigt.
	\begin{equation}
		F_{komm}=\frac{v}{\frac{n_{S} \times (n_{S} - 1)}{2}}
	\end{equation}
	\item Die innere Flexibilit�t von Software kann einerseits �ber die H�ufigkeit der Nutzung von entkoppelnden Entwurfsmustern in einer Software beurteilt werden. Eine Bewertung muss hier generell vergleichend mit Augenma� erfolgen, da der Anwendungszweck und das Paradigma zwangsl�ufig Unterschiede hervorrufen.
		Weitere M�glichkeiten bestehen in dem auf Flexibilit�tsbeurteilung ausgelegten Scoring-Point-Modell \textit{FleXible Point}\footnote{Siehe dazu \cite{shen2006analysis}.} oder in der Beurteilung zur strukturellen Ver�nderung �ber die Evolutionskosten-Metrik.\footnote{Siehe dazu \cite{eden2006measuring}.}
\end{enumerate}
%----Wertbeitrag
Die Wertbeitragsbemessung dieser Messung kann ebenfalls verfolgt werden:
\begin{enumerate}
	\item Die Steigerung der �nderungsflexibilit�t sollte mit der Reduktion von �nderungsverursachten Kosten $K_{aender}$ verbunden sein. Hierbei ist eine Steigerung (also ein Ma� $> 1$) der Effektivit�t der �nderungsflexibilit�t $E_{F_{gruppe}}$ zwischen zwei Perioden $t_1$ und $t_2$ anzustreben.
		Diese Berechnungsmethode ist zwar insofern universell, als dass sie mit allen angegebenen Bewertungsma�st�ben aus (3) funktioniert, allerdings in der Praxis durch Spezialisten auszulegen, da �nderungen an Software ebenfalls schwer zu quantifizieren und daher schwer zu vergleichen sind.
		\begin{equation}\tag{W: (3)}
			E_{F_{aender}} = \frac{F_{aender}(t_2)}{F_{aender}(t_1)}/\frac{K_{aender}(t_2)}{K_{aender}(t_1)}
		\end{equation}
		
\end{enumerate}
%----Ende \\
Als wertsch�pfungsf�higes Endprodukt bestehen f�r die \textbf{IT-Dienste} in ihrer Flexibilit�t in Folge der dargelegten Ans�tze Abh�ngigkeiten zur Software, auf der sie basieren, zum Personal, das diese implementiert und betreibt und zur Infrastruktur, die die Software ausf�hrt und bereith�lt.
Manche Indikatoren wie die Verf�gbarkeit sind allerdings f�r den IT-Dienst als Zielgr��e relevanter als f�r die beeinflussenden Dimensionen.
Vereinfacht ausgedr�ckt ist f�r den Endbenutzer nur die Verf�gbarkeit des IT-Dienstes kritisch und im Falle der Nicht-Verf�gbarkeit irrelevant, ob es sich dabei um Fehler in Software oder Infrastruktur handelt.
Die aus der produktionswirtschaftlichen Flexibilit�t stammende Forderung nach Variantenreichtum l�sst sich zwar auf den Variantenreichtum von IT-Diensten �bertragen, aber die Komplexit�t in der Produktion resultiert dabei vor allem aus den Rahmenbedingungen des Produktionssystems, das verschiedene Produkte herstellen k�nnen muss.
F�r IT-Dienste besteht diese Schwierigkeit allerdings nicht in dem Ma�, da ein einmal bereitgestellter Dienst nicht zwangsl�ufig eine Restriktion f�r die Bereitstellung anderer Dienste impliziert.\\
Die Flexibilit�t der IT hinsichtlich der IT-Dienste manifestiert sich im Endeffekt vor allem in zwei Indikatoren, die den Endbenutzer direkt betreffen: der Bereitstellungszeit eines neuen oder ge�nderten IT-Dienstes und dessen Business Alignment, also am Grad der vom Fachbereich geforderten Funktionserf�llung.\\
Wie bereits bei der Software angedeutet sind Bereitstellungen bzw. Implementationen in der Regel projektartig und werden daher in der Projektdimension behandelt.
Hohes Business Alignment kann vor allem dadurch erreicht werden, dass Anforderungen des Fachbereichs differenziert und vor allem kontinuierlich in die IT transportiert und dort umgesetzt werden.
Auch ein diesbez�glicher Ansatz wird bei den Projekten er�rtert.
Um aber eine Direktive f�r das Business Alignment der IT-Dienste postulieren zu k�nnen, ist vor allem das gesamtbetriebliche Ziel einzubeziehen.
Dieses besteht darin, m�glichst viele T�tigkeiten, Vorg�nge und Prozesse durch IT-Dienste abzubilden und damit verbundene Vorteile nutzbar zu machen.
\\
%----Messung
Ans�tze zur Messung dieser Flexibilit�tsaspekte sind:
\setcounter{equation}{0}
\begin{enumerate}
	\item als Anteil der IT-gest�tzten T�tigkeiten bzw. Prozesse $p_{IT}$ in einem Unternehmen
	\begin{equation}
		F_{align}= \frac{p_{IT}}{p_{IT}+p_{nicht-IT}}
	\end{equation}
	\item als Verf�gbarkeit der Dienste unter Ber�cksichtigung von Offline-Wartungen und -Erg�nzungen, welche z.B. durch Online-Wartungsf�higkeit zu verhindern w�ren
	\begin{equation}
		F_{verf}= \frac{t_{n-verf}}{t_{verf}+t_{n-verf}}
		\end{equation}
	%\todo{Online-Updatef�higket}
	%\todo{Anteil offlinef�higer Prozesse}
	%\todo{Anteil tats�chlich genutzter IT-Dienste}
\end{enumerate}
%----Wertbeitrag
Die Wertbeitragsbemessung dieser Messung kann ebenfalls verfolgt werden:
\begin{enumerate}
	\item als Prozessqualit�t $Q_p$ im Vergleich der Prozessdurchf�hrungen vor und nach deren Digitalisierung; neben Prozessdurchlaufzeit $DLZ$ sind auch andere Kernfaktoren von Prozessen wie die Prozessfehlerrate einsetzbar
		\begin{equation}\tag{W: (1)}
			Q_p = DLZ_{nach-IT}/DLZ_{vor-IT}
		\end{equation}
\end{enumerate}
%----Ende \\
Die letzte zu betrachtende Dimension stellen die in der IT relevanten \textbf{Projekte} dar.
Deren Bedeutung ist bereits verdeutlicht durch den Status als Quelle entscheidender Artefakte wie Software.
Plangetriebene Projektmanagementans�tze f�hren dabei wie indiziert h�ufig zu nicht zufriedenstellenden Projektergebnissen (vgl. \ref{Relevanz}).
Dieser Zustand ist ma�geblich auf strukturelle Schwachstellen statischer, plangetriebener Projektmanagementmethoden zur�ckzuf�hren, insbesondere schwankender Projektumfang sowie unklare und infolgedessen nicht erreichte Anforderungen.\footnote{Vgl. \cite{taylor2000}.}
An genau diesen Schwachstellen setzen agile Methoden an.\footnote{Vgl. \cite{beck2001agile}.}
Naheliegend scheint also die Ma�nahme, Projekte tendenziell st�rker agil als plangetrieben auszulegen und dadurch auch typische Sach- und Formalziele wie Budget und Termin besser einzuhalten.\footnote{Vgl. bzgl. Projektzielen \cite[S.174-177]{Moersdorf1998}.}
Gerade f�r Softwareprojekte bietet sich dieses Vorgehen an.\footnote{Vgl. \cite{Baumeister2014}.}
Der darin enthaltene Aspekt der Kunden- bzw. Endabnehmerintegration adressiert den angesprochenen kontinuierlichen Transport von Anforderungen in die IT-Abteilung und geht darauf mit zyklischer Validierung gelieferter Fortschrittsinkremente durch den Kunden ein.
\\
Eine weitere Ma�nahme, die an dieser Stelle ansetzt und die Bereitstellungszeit wie zuvor indiziert beeinflusst ist das DevOps-Paradigma.
Dieses strebt organisatorisch eine Integration von Entwicklung (Development) und Betrieb (Operations) an, um Verantwortlichkeit transparent zu machen, alle Beteiligten am Prozess zusammenzubringen und den interdisziplin�ren Austausch zu f�rdern.\footnote{Vgl. \cite{ebert2016}.}
�nderungsanforderungen werden so schneller umgesetzt und die Zusammenarbeit verbessert.\footnote{Vgl. \cite{RiunguKalliosaari2016}.}\\
In Zusammenhang mit DevOps und agilen Projektmanagementmethoden eignen sich Prinzipien wie Continous Delivery, die durch Integration von Entwicklerteam und Kunde eine sehr fr�he Lieferung und stetige Erweiterung erzielen sollen.\footnote{Vgl. \cite{Meyer2014}}
%----Messung
Ans�tze zur Messung dieser Flexibilit�tsaspekte sind:
\setcounter{equation}{0}
\begin{enumerate}
	\item als Anteil der mit agilen Methoden durchgef�hrten Projekte im Unternehmen
	\begin{equation}
		F_{projekt} = \frac{n_{agil}}{n_{agil}+n_{plan}}
	\end{equation}
	\item als zu minimierender durchschnittlicher Umfang der an den Endbenutzer ausgelieferten �nderungen $U_i$ an einer Software in Relation zu deren urspr�nglicher Gr��e $U$, wobei hier die Quantifizierung des �nderungsumfangs Interpretationsspielraum bietet (z.B. als Anzahl Codezeilen oder subjektive Einsch�tzung des Funktionsumfangs auf einer Bewertungsskala)
	\begin{equation}
		F_{aender} = \frac{\sum\limits_{i=1}^{n}{U_i}}{n}/U
	\end{equation}
	\item als zu maximierende Frequenz von Releases $r$, also ausgelieferter Software�nderungen oder -erg�nzungen innerhalb einer Periode $p$
	\begin{equation}
		F_{liefer} = \frac{r}{p}
	\end{equation}
\end{enumerate}
%----Wertbeitrag
Die Wertbeitragsbemessung dieser Messung kann ebenfalls verfolgt werden.
\begin{enumerate}
	\item als Sach- bzw. Formalzieleinhaltungsf�higkeit der agilen Projekten gegen�ber den plangetriebenen, z.B. als vergleichende Budgeteinhaltung der flexiblen Budgets $BF$ und der plangetriebenen Budgets $PF$; diese Methode bedingt eine Budgetindikation f�r agile Projekte
	\begin{equation}\tag{W: (1)}
		E_{agil} = \frac{\sum\limits_{i=1}^{n}{\left(BF_{plan} - BF_{ist}\right)}}{\sum\limits_{i=1}^{n}{\left(BP_{plan} - BP_{ist}\right)}}
	\end{equation}
	\todo{zufriedenheit? fehleranf�lligkeit? akzeptanzrate hinsichtlich Missnutzung?}
\end{enumerate}
%----Ende 

%Bei Betrachtung der durchschnittlichen relativen Schwankung der Kapazit�tsauslastung $K_{schwank}$ einer gemessenen Stichprobe zu den Zeitpunkten $t_n$ bis $t_m$ berechnet sich diese zu
%\begin{equation}
%K_{schwank} = \sqrt{\frac{\left(\displaystyle\sum_{i=n}^{m}\left({K_{prod_i+1} - K_{prod_i}}\right)^{2}\right) }{m-n}}.
%\end{equation}
%Um eine bestm�gliche Auslastung der eigenen Infrastruktur zu erreichen, w�re das Verh�ltnis von $K_{schwank}$ zur durchschnittlichen Auslastung $\overline{K}$ zu minimieren, somit ist Auslastung
%\begin{equation}
%K = K_{schwank} - \overline{K} = 0
%\end{equation}
%anzustreben.
%Da dies im Regelfall nicht gegeben ist, und 
%
%\begin{enumerate}
%	\item Erweiterbarkeit
%	\begin{equation}
%		\frac{Minimale Kapazitaetserweiterung}{\frac{Gesamtkapazitaet}{Anzahl Infrastrukturelemente}}
%	\end{equation}
%	\item Die Redundanz ergibt sich aus der Anzahl von Infrastrukturkomponenten im Verh�ltnis zu deren gesamten Anzahl, die ausfallen m�ssen, damit das Gesamtsystem nicht mehr operabel ist oder alternativ die Anzahl, bei der das System nicht mehr die �blicherweise geforderte Leistung erbringen kann.
%	\item Die Verf�gbarkeit ist ein simples Verh�ltnis der Zeitr�ume, in denen die Infrastruktur funktional zur Bereitstellung von IT-Diensten zur Verf�gung steht, zu denen, in denen es nicht der Fall ist.
%	Dieses Ma� ist also dahingehend zu maximieren.
%\end{enumerate}
%%-------------



\section{Rahmenwerk f�r Flexibilit�t in der IT}
\subsection{Konzeptionelle Idee}
Da wie dargelegt Wechselwirkungen zwischen verschiedenen flexibilisierbaren Betrachtungsgegenst�nden in der IT bestehen, scheint es naheliegend, diesen Ansatz zur Integration in einem Rahmenwerk zur Bewertung der IT-Flexibilit�t zu nutzen.
Einerseits existieren wie dargelegt M�glichkeiten zum gezielten Aufbau von Flexibilit�tspotentialen in verschiedenen Dimensionen bzw. f�r verschiedene Betrachtungsgegenst�nde, deren Implementation wie angegeben beurteilbar ist, andererseits sind dadurch Effekte zu erzielen, deren Auftreten ebenfalls messbar ist.\\
Insofern besteht also ein integriert operationalisierbares Konstrukt, das zur koh�renten Steuerung des Aufbaus dieser Flexibilit�tspotentiale und gleichzeitig dauerhaft zur �berpr�fung des Nutzens eingesetzt werden kann.
Die Ma�nahmen zum Aufbau der Flexibilit�tspotentiale entsprechen kurz-, mittel- und langfristigen Vorhaben, da z.B. der Infrastrukturaufbau aufgrund des 72-monatigen Abschreibungszeitraums eher als strategische Investition get�tigt wird\footnote{Vgl. \cite[Ziffer 6.14.3.1]{afa-tabelle}.}, Personalentwicklung ebenfalls eine langwierige Aufgabe ist, aber die Durchf�hrung von agilen Softwareprojekten zur Herstellung von Microservices durchaus auch k�rzere Zeitr�ume als ein Jahr in Anspruch nehmen kann. 
Unter Ber�cksichtigung der Tatsachen, dass die Ma�nahmen sowohl strategischer als auch taktisch-operativer Natur sein k�nnen, ist das Konzept der \gls{BSC} die angemessenste Variante zur Modellierung der bisherigen Konzeption.
\subsection{Dimensionsdefinition}
Wie in \ref{BSC} angesprochen liegt die Voraussetzung f�r die Koh�sion der mit einer \gls{BSC} angestrebten Steuerung in der Definition von Ursache-Wirkungs-Ketten der Dimensionen.
Je nach Vorhaben ist allerdings vorher sogar eine Definition der zu untersuchenden Dimensionen vorzunehmen.\\
Unter Ber�cksichtigung der in \ref{adaptionbetrachtungsgegenstaende} ermittelten Betrachtungsgegenst�nde sind diesbez�gliche Kandidaten ersichtlich, sodass als Dimensionen der Flexibilit�t das Personal, die technologischen Sichtweisen Infrastruktur/Software sowie die Projekte in Frage kommen.
Zur ganzheitlichen Steuerung eines Unternehmens sind zudem der Einfluss auf die Kunden- sowie die Finanzperspektive zu ber�cksichtigen.
Als Basis von Gesch�ftsprozessen ist die Interpretation der IT-Dienste als Substitution der Prozessperspektive als eine im Modellierungsvorhaben sinnvolle Abwandlung einzustufen.
Die Aggregation der flexibilit�tstragenden Betrachtungsgegenst�nde in einer Dimension, die die Lern- und Entwicklungsperspektive ersetzt, ist eine Ma�nahme, die bereits in �hnlicher Weise Vorhaben f�r produktionswirtschaftliche Flexibilit�t gepr�gt hat.\footnote{Vgl. \cite[S.9]{winkler2008} und \cite[S.57]{Kempkes2018}.}
Mit dieser Konstellation k�nnen nun Wirkungsketten modelliert werden (vgl. Abbildung \ref{img:wirkungsketten}).
Dieses Vorhaben sollte in der Praxis Top-Down mit der Direktive finanziellen unternehmerischen Erfolgs als Ausgangspunkt erfolgen\footnote{Vgl. \cite[S.48]{horvath1998balanced}.}.\\
Ausgehend von der Lern- und Entwicklungsperspektive, die durch die Flexibilit�tspotentiale ersetzt wird, kann alternativ auch eine Bottom-Up-Modellierung vorgenommen werden.\footnote{Vgl. \cite[S.181ff]{horvath2001} nach \cite[S.141]{siebert2011balanced}.}
 \begin{figure}[H]
\centering
\includegraphics[width=1\linewidth]{bsc_wirkungsketten}
\captionof{figure}[Wirkungsketten]{Wirkungsketten\footnotemark}
\label{img:wirkungsketten}
\end{figure}
\addtocounter{footnote}{+0}\footnotetext{Eigene Darstellung in Anlehnung an \cite[Abb. 3]{winkler2008}.}
\noindent Die ausgew�hlten Ziele der Kunden- und Finanzperspektive sind �bliche Aspekte der Unternehmenssteuerung\footnote{Vgl. zur Auswahl \cite[S.181ff]{horvath2001} nach \cite[S.141]{siebert2011balanced} sowie zur Modellierung \cite[S.9]{winkler2008}.}, die zur praktischen Verwendung ggf. anzupassen und zu parametrieren sind.
Die Wirkungen der Flexibilit�tspotentiale k�nnen wie zuvor beschrieben modelliert werden und lassen zumindest intuitiv die Verbindung zu gesamtunternehmerischen Zielen zu.
\subsection{FlexIT zur Wertbeitragsbemessung}
Die dargelegten Verh�ltnisse rechtfertigen tats�chlich die Integration in einer \gls{BSC}-Adaption f�r IT-Flexibilit�t.
 \begin{figure}[H]
\centering
\includegraphics[width=0.75\linewidth]{flex_it}
\captionof{figure}[FlexIT]{FlexIT\footnotemark}
\label{img:flex_it}
\end{figure}
\addtocounter{footnote}{+0}\footnotetext{Eigene Darstellung in Anlehnung an \cite[Abb. 1]{Kempkes2018}.}
\noindent Das Resultat, das hier nun mit der Bezeichnung \textit{FlexIT} versehen wird, kann die vorgeschlagenen Ma�nahmen wie agile Durchf�hrung von Projekten, Modularisierung der Anwendungslandschaft, Aufbau von Infrastrukturredundanz etc. steuern, indem es sie strukturell anweist sowie laufend begleitet und bewertet.\\
Die Ma�nahmen, deren Gesamtheit hier nicht ersch�pfend und allgemeing�ltig definiert werden kann, sind dazu vorab in einer IT-Strategie zu definieren und die  Wirkungsketten auf Kompatibilit�t zu hierarchisch �bergeordneten Zielen zu pr�fen.
Unternehmensspezifisch sind auch die jeweiligen Zielwerte der Flexibilit�tskennzahlen zu definieren, da eine vollst�ndige Implementation der Ma�nahmen wie 100\% Mobilit�t des Personals durch ausbleibende Kommunikation in B�ror�umen kontraintuitive Effekte haben kann und nicht proportional zum Implementationsgrad produktivit�tsf�rdernd wirkt.
Optimale bzw. Grenzwerte, die multifaktoriell von der Unternehmensorganisation beeinflusst werden, sind also ebenfalls unternehmensspezifisch, aber modellintern durch die dargelegten Kennzahlkombinationen zu ermitteln.
Ein diesbez�gliches Beispiel best�nde in der schrittweisen Einf�hrung von Arbeitsplatzmobilit�t f�r Teile des Personals und deren Einsatz im Homeoffice, die zun�chst zur Steigerung des verbundenen Produktivit�tsma�es f�hren w�rde.
Ab einem gewissen Anteil, z.B. �ber 50\% Implementationsgrad, w�rde ggf. aber wieder r�ckl�ufige Produktivit�t erzielt, da die direkte Kommunikation zu stark gesunken ist, weshalb keine weitere Ma�nahmenimplementation mehr vorzunehmen und stattdessen 50\% als Zielwert festzusetzen w�re.\\
Eine gleichwertige Parameterkombination ist, z.B. mit den in dieser Arbeit definierten Kennzahlen, auch f�r �brige Ma�nahmen jeweils zu ermitteln und fortw�hrend die Strategie daran zu validieren.
%-Beispiel von Kennzahlensteigerung zu Wertbeitragssteigerung
%\subsection{Interpretation als Werttreiber}

\section{Fazit und Potential}
Mit Abschluss der Konzeption kann nun die Zielerreichung �berpr�ft, die Ergebnisqualit�t und -bedeutung diskutiert sowie ein Blick auf darauf aufbauende M�glichkeiten geworfen werden.\\\\
Das Ziel, Flexibilit�t in der IT bewertbar und steuerbar zu machen, ist insofern erreicht, als dass das konzeptionelle Ergebnis die in \ref{sec:aufgabenundziele} aufgestellten Ergebniskriterien erf�llt:\\
Das in Form einer \gls{BSC} entworfene Resulat zentriert Informationen zur Steuerung von Ma�nahmen unterschiedlichen zeitlichen Horizonts.
Die Ma�nahmen, die die dabei die Flexibilisierung interner Verh�ltnisse anvisieren, um auf eine dynamische Umwelt reagieren zu k�nnen, entsprechen insoweit der Controlling-Konzeption von K�pper et al. und Horv\'ath et al, als dass deren ebenda dargestellte Anspr�che ber�cksichtigt sind.\\
Die modellierten Verbindungen zu gesamtunternehmerischen Zielen der Kunden- und Finanzperspektiven lassen letztendlich sogar die Vermutung einer ganzheitlich m�glichen Steuerung zu.
Die Flexibilisierbarkeit entscheidender IT-Inhalte und deren Tragweite indizieren, dass eine konsequente Flexibilisierung - deren Ideen-Ursprung wie anfangs indiziert in dynamischen Rahmenbedingungen liegt - essentielle Bestandteile der Betrachtung einer IT-Strategie abdecken kann.\\
\\
Ein diesbez�glicher empirischer Beweis zur Validierung des Modells steht allerdings aus, weshalb nur Analogien zu Erkenntnissen �hnlicher Forschung ermittelt werden k�nnen. 
Das Modell, dass allerdings unternehmens�bergreifende Vergleichbarkeit von Flexibilit�t und deren Auswirkungen erm�glicht, ist dahingehend modellimmanent validierbar.
In dieser Hinsicht ist das Modell auch bisherigen Flexibilit�tsuntersuchungen in der IT auch insoweit voraus, als dass z.B. die nicht eindeutig messbaren, zumal nicht eindeutig definierten, gesamtsystemischen Indikatoren von Byrd/Turner (Integration, Konnektivit�t, Modularit�t)\footnote{Vgl. \cite{byrdturner2000}.} nicht unternehmens�bergreifend und nicht im Detail, d.h. je Ma�nahme, sondern nur im in Verbindung validierbar sind.
Nichtsdestotrotz l�sst die Einstufung der Verbindung zum Unternehmenserfolg, die in qualitativer Erhebung damit erbracht wurde\footnote{Vgl. \cite{Tallon20030UF}.}, den Schluss zu, dass der ganzheitliche Steuerungsansatz, der in dieser Arbeit konzipiert ist, einen validen Zugang darstellt und die durch Flexibilit�t angestrebten Ziele die vermutete Tragweise besitzen k�nnen.
\\
\\
Nichtsdestotrotz bleiben Gestaltungsm�glichkeiten des Modells bestehen.
Die Flexibilit�tsans�tze aus dem \gls{L-C} z.B., die mit denen des \gls{P-C} in Verbindung stehen (vgl. \ref{bsc_quellen}), weisen ggf. auch Adaptionspotential auf und k�nnen andere Messungsans�tze liefern.\\\\
F�r die Flexibilit�t der IT-Organisation ist letztlich zu konstatieren, dass die Betrachtung weder in der Breite noch der Tiefe stattgefunden hat wie f�r produktionswirtschaftliche Flexibilit�t, die trotzdem noch nicht zu allgemeinem Konsens gelangt ist.
Auch f�r die IT ist daher in Anbetracht der f�r sie ebenfalls geltenden dynamischen Umweltbedingungen anzunehmen, dass die Diskussion um Flexibilit�t �hnliche Ausma�e annehmen kann und vielleicht auch sollte.\\\\
Diese Arbeit leistet dazu einen konzeptionellen Beitrag.
%Multiprojektcontrolling einbeziehen
%auf logistikcontrolling eingehen

\newpage
\backmatter
\addcontentsline{toc}{section}{Quellenverzeichnis}
\printbibliography
\appendix
\addcontentsline{toc}{section}{Anhang}
\renewcommand\thesection{\Alph{section}}
  \begin{appendix}
  \let\svaddcontentsline\addcontentsline
\renewcommand\addcontentsline[3]{%
  \ifthenelse{\equal{#1}{lof}}{}%
  {\ifthenelse{\equal{#1}{lot}}{}{\svaddcontentsline{#1}{#2}{#3}}}}
 \renewcommand\thefigure{\thesubsection.\arabic{figure}}  
  \setcounter{figure}{0}

\section*{Anhang}  
  \section{Einsatzbereiche nach Gottmann}
  \subsection{Beschaffung/Lieferanten}\label{a1}
    \begin{figure}[!htb]
	\centering
	\includegraphics[width=0.7\linewidth]{gottmann_1}
	\captionof{figure}[Zielgr��en, Einflussfaktoren und Indikatoren in der Beschaffung]{Zielgr��en, Einflussfaktoren und Indikatoren in der Beschaffung\footnotemark}
	\label{img:gottmann_1}
	\end{figure}
	\addtocounter{footnote}{+0}\footnotetext{\cite[Abb. 3.5]{Gottmann2019}}
	
  \subsection{Anlagen und Produktionsprozesse}\label{a2}
    \begin{figure}[!htb]
	\centering
	\includegraphics[width=0.7\linewidth]{gottmann_2}
	\captionof{figure}[Zielgr��en, Einflussfaktoren und Indikatoren f�r Anlagen und Produktionsprozesse]{Zielgr��en, Einflussfaktoren und Indikatoren f�r Anlagen und Produktionsprozesse\footnotemark}
	\label{img:gottmann_2}
	\end{figure}
	\addtocounter{footnote}{+0}\footnotetext{\cite[Abb. 3.6]{Gottmann2019}}
	\newpage

  \subsection{Personal}\label{a3}
    \begin{figure}[!htb]
	\centering
	\includegraphics[width=0.8\linewidth]{gottmann_3}
	\captionof{figure}[Zielgr��en, Einflussfaktoren und Indikatoren f�r Personal]{Zielgr��en, Einflussfaktoren und Indikatoren f�r Personal\footnotemark}
	\label{img:gottmann_3}
	\end{figure}
	\addtocounter{footnote}{+0}\footnotetext{\cite[Abb. 3.7]{Gottmann2019}}
	
  \subsection{Qualit�t}\label{a4}
    \begin{figure}[!htb]
	\centering
	\includegraphics[width=0.8\linewidth]{gottmann_4}
	\captionof{figure}[Zielgr��en, Einflussfaktoren und Indikatoren in der Qualit�t]{Zielgr��en, Einflussfaktoren und Indikatoren in der Qualit�t\footnotemark}
	\label{img:gottmann_4}
	\end{figure}
	\addtocounter{footnote}{+0}\footnotetext{\cite[Abb. 3.8]{Gottmann2019}}
	\newpage
	
  \subsection{Material und Logistik}\label{a5}
    \begin{figure}[!htb]
	\centering
	\includegraphics[width=0.8\linewidth]{gottmann_5}
	\captionof{figure}[Zielgr��en, Einflussfaktoren und Indikatoren in der Logistik]{Zielgr��en, Einflussfaktoren und Indikatoren in der Logistik\footnotemark}
	\label{img:gottmann_5}
	\end{figure}
	\addtocounter{footnote}{+0}\footnotetext{\cite[Abb. 3.9]{Gottmann2019}}
	
	 \subsection{Organisation/Auftragsabwicklung}\label{a6}
    \begin{figure}[!htb]
	\centering
	\includegraphics[width=0.8\linewidth]{gottmann_6}
	\captionof{figure}[Zielgr��en, Einflussfaktoren und Indikatoren in der Organisation]{Zielgr��en, Einflussfaktoren und Indikatoren in der Organisation\footnotemark}
	\label{img:gottmann_6}
	\end{figure}
	\addtocounter{footnote}{+0}\footnotetext{\cite[Abb. 3.10]{Gottmann2019}}
	\newpage

  \subsection{Kunden}\label{a7}
    \begin{figure}[!htb]
	\centering
	\includegraphics[width=0.8\linewidth]{gottmann_7}
	\captionof{figure}[Zielgr��en, Einflussfaktoren und Indikatoren in Richtung Kunde]{Zielgr��en, Einflussfaktoren und Indikatoren in Richtung Kunde\footnotemark}
	\label{img:gottmann_17}
	\end{figure}
	\addtocounter{footnote}{+0}\footnotetext{\cite[Abb. 3.11]{Gottmann2019}}
\section{Messmodelle zur Flexibilit�t nach Bellmann et al.}\label{b1}
%Fu�noten verteilen

%longtable
\begin{longtable}{| p{.18\textwidth} | p{.80\textwidth} |} 
\hline
\textbf{Modell von...} & \textbf{misst die} \\ \hline
\makecell[l]{Marschak/\\Nelson} & {Flexibilit�t von Entscheidungen �ber die Teilmengenbeziehung der Mengen der nach der Anfangsentscheidung noch bestehenden Handlungsm�glichkeiten\footnotemark} \\ \hline
\makecell[l]{Gupta/\\Rosenhead} & {Flexibilit�t einer Entscheidungsfolge �ber das Verh�ltnis der Anzahl der akzeptablen Endzust�nde zu der Gesamtzahl der akzeptablen Endzust�nde\footnotemark} \\ \hline
{Jacob} & {Bestandsflexibilit�tsma�zahl sowie Entwicklungsflexibilit�tsma�zahl als Quotienten des Gewinns bei optimaler Anpassung bei ,,prophetischem Wissen'' und dem Gewinn bei optimaler Anpassung entsprechend einer Entscheidung, jeweils vermindert um den Gewinn bei Nicht-Anpassung\footnotemark} \\ \hline
{Mahlmann} & {Flexibilit�t einer Entscheidung sowie Flexibilit�t des Planungsprozesses. Flexibilit�t des Planungsprozesses wird bestimmt �ber die Menge der Planrevisionszeitpunkte und Flexibilit�t der Entscheidung �ber die Anzahl von Entscheidungsm�glichkeiten, die �ber die Menge an Entscheidungsm�glichkeiten einer optimalen Strategie hinaus bestehen.\footnotemark} \\ \hline
{Hanssmann} & {Flexibilit�t einer Strategie �ber den Gesamterfolg, der damit erreicht wird. Dazu wird eine Quotient aus Gesamterfolg der Strategie und Gesamterfolg bei optimaler Anpassung, jeweils vermindert um den Gesamterfolg bei Inflexibilit�t, gebildet.\footnotemark} \\ \hline
\makecell[l]{Eversheim/\\Schaefer} & {Flexibilit�t als Kapazit�t eines Produktionssystems. Flexibilit�t wird als quantitative oder qualitative �berkapazit�t verstanden. Dementsprechend soll das Kapazit�tsangebot auf die Kapazit�tsnachfrage abgestimmt werden.\footnotemark} \\ \hline
\makecell[l]{Lassere/\\Roubellat} & {Flexibilit�t einer Entscheidung �ber das Volumen des bestehenden Handlungsraumes. Die Nebenbedingungen einer zu w�hlenden Zielfunktion (z.B. Minimierung der Lagerkosten) bilden dabei die Beschr�nkungen des Handlungsraumes. Zur Berechnung des Volumens wird ein von Laster aufgestelltes und bew�hren Theorem verwendet.\footnotemark} \\ \hline
{Kumar} & {Flexibilit�t eines Fertigungssystems �ber die Anzahl der Wahlm�glichkeiten im System (z.B. Anzahl der Aggregate oder Routen) und die Verf�gbarkeit der jeweiligen Wahlm�glichkeiten. Zur Berechnung zieht der Autor ein Entropiema� aus der Thermodynamik bzw. Informationstheorie.\footnotemark} \\ \hline
{Pauli} & {Flexibilit�t �ber Kennzahlen. So steht. bspw. das Verh�ltnis von Anzahl wirtschaftlicher Absatzregionen f�r die statische �rtliche Flexibilit�t.\footnotemark} \\ \hline
{Wolf} & {Flexibilit�t als Kapazit�t eines Produktionssystems, die in Matrizen beschrieben wird. Wenn die Kapazit�tsnachfrage gr��er als das Kapazit�tsangebot ist, besteht eine Kapazit�tsl�ckenmatrix. Interpretiert als Flexibilit�tsl�cken, ist ein System um so flexibler, desto mehr Flexibilit�tsl�cken gedeckt werden k�nnen.\footnotemark} \\ \hline
\makecell[l]{Mandelbaum/\\Buzacott} & {Flexibilit�t einer Entscheidung �ber die M�chtigkeit der Menge der Folgereaktionen. D.h., je mehr Handlungsoptionen eine Entscheidung erm�glicht, desto flexibler ist sie.\footnotemark} \\ \hline
\makecell[l]{Schneewei�/\\K�hn} & {Flexibilit�t einer Aktionenfolge �ber ein Verrichtungsma�. Das Verrichtungsma� gibt an, welche Systemzielwirkung eine Aktionenfolge besitzt, z.B. im Bezug auf Kosten, Gewinn etc. Die Flexibilit�t einer Aktionenfolge wir berechnet als Quotient aus dem Verrichtungsma� der Aktionenfolge und dem Verrichtungsma� einer optimalen Aktionsfolge bei Sicherheit �ber zuk�nftige Zust�nde, jeweils vermindert um das Verrichtungsma� einer Aktionenfolge, die keine Anpassung vorsieht.\footnotemark} \\ \hline
{Ost} & {Flexibilit�t von Maschinen �ber Kennzahlen, die f�r einen Teilflexibilit�tenkatalog bestimmt werden. Durch die Zuteilung eines Wahrheitsgrades im Rahmen einer fuzzy-logic-Berechnung erstellt der Autor eine Hierarchie von Flexibilit�tsbedarfen f�r die Teilflexibilit�ten.\footnotemark} \\ \hline
{Chen/Chung} & {Flexibilit�t flexibler Fertigungssysteme (FFS) �ber Kennzahlen in Form von Bearbeitungsflexibilit�t und Routenflexibilit�t. Die Ma�zahl f�r Bearbeitungsflexibilit�t ist der Quotient aus Anzahl der f�r ein FFS durchf�hrbaren Aufgaben und Gesamtzahl aller Aufgaben eines Produktionssystems, Routenflexibilit�t der Quotient aus der Summe aller m�glichen Wege der Werkl�cke durch ein FFS und Menge aller Werkst�cke.\footnotemark} \\ \hline
\makecell[l]{Meier-\\Barthold} & {Flexibilit�t von Probleml�sungsverfahren �ber die Menge zul�ssiger Entscheidungsfolgen eines Probleml�sungsverfahrens und M�chtigkeit der Menge der zul�ssigen Entscheidungsfolgen eines ,,best case'' bzw. den Quotient aus Volumen der Menge zul�ssiger Entscheidungsfolgen eines Probleml�sungsverfahrens und Volumen der Menge der zul�ssigen Entscheidungsfolgen eines ,,best case''.\footnotemark} \\ \hline
{Nagel} & {Flexibilit�t eines Produktionssystems �ber Flexibilit�tskosten, die durch Ressourcenmehrbedarfe entstehen. Um diese Mehrbedarfe ermitteln zu k�nnen, nutzt die Autorin den System Dynamics-Ansatz [sic!] zur Erfassung des gesamten Ressourceneinsatzes und -bedarfes.\footnotemark} \\ \hline
\makecell[l]{Corsten/\\G�ssinger} & {Flexibilit�t eines Produktionssystems f�r Dienstleistungen �ber die Berechnung der Volumina, die durch Mengen an ,,l�sbaren Problemen'' und ,,akzeptablen Probleml�sungen'' bestimmt sind. beide Mengen sind jeweils durch die Entscheidung �ber das Produktionssystem festgelegt.\footnotemark} \\ \hline
{Mirschel} & {Flexibilit�t von Produktionssystemen mit Kennzahlen. Zur Operationalisieren von Flexibilit�tspotentialen unterscheidet der Autor Teilflexibilit�ten auf drei Ebenen und nutzt bspw. Kapazit�tsdifferenzen zwischen minimaler und maximaler Kapazit�t eines Aggregates als Messgr��en.\footnotemark} \\ \hline
\makecell[l]{Realoptions-\\bewertungs-\\modelle} & {Flexibilit�t von Realoptionen. Analog Finanzoptionen erlauben es Realoptionen eine zuk�nftige Wahl zwischen zwei Zust�nden (z.B. Erweiterung oder Nicht-Erweiterung der Kapazit�t einer Anlage zu treffen, was eine Form der Flexibilit�t interpretiert wird. [sic!] Instrumente zur Finanzoptionsbewertung k�nnen dementsprechend genutzt werden, um die Flexibilit�t von Realoptionen zu messen.\footnotemark} \\ \hline
\caption[Messmodelle zur Flexibilit�t nach Bellmann et al.]{Messmodelle zur Flexibilit�t nach Bellmann et al.\footnotemark}\label{tab:b1}
\end{longtable}
\addtocounter{footnote}{-19}\footnotetext{Vgl. \cite[S.42ff]{Marschak1962} zitiert nach \cite[S.98f]{Pibernik2001}}
\addtocounter{footnote}{+1}\footnotetext{Vgl. \cite[S.B18ff]{Gupta1968}}
\addtocounter{footnote}{+1}\footnotetext{Vgl. \cite[S.322ff]{jacob1974}}	
\addtocounter{footnote}{+1}\footnotetext{Vgl. \cite[S.124ff]{Mahlmann1976}}	
\addtocounter{footnote}{+1}\footnotetext{Vgl. \cite[S.228ff]{hanssmann1978}}	
\addtocounter{footnote}{+1}\footnotetext{Vgl. \cite[S.229ff]{eversheim1980}}	
\addtocounter{footnote}{+1}\footnotetext{Vgl. \cite[S.447ff]{lasserre1985} zitiert nach \cite[S.31ff]{MeierBarthold1999}}	
\addtocounter{footnote}{+1}\footnotetext{Vgl. \cite[S.957ff]{KUMAR1987}}	
\addtocounter{footnote}{+1}\footnotetext{Vgl. \cite[S.87ff]{pauli1987}}	
\addtocounter{footnote}{+1}\footnotetext{Vgl. \cite[S.25ff]{Wolf1989}}	
\addtocounter{footnote}{+1}\footnotetext{Vgl. \cite[S.17ff]{Mandelbaum1990}}	
\addtocounter{footnote}{+1}\footnotetext{Vgl. \cite[S.380ff]{schneeweiss1990}}	
\addtocounter{footnote}{+1}\footnotetext{Vgl. \cite[S.39ff]{ost1993}}	
\addtocounter{footnote}{+1}\footnotetext{Vgl. \cite[S.397ff]{CHEN1996}}	
\addtocounter{footnote}{+1}\footnotetext{Vgl. \cite[S.51ff]{MeierBarthold1999}}	
\addtocounter{footnote}{+1}\footnotetext{Vgl. \cite[S.37ff]{Nagel2003}}	
\addtocounter{footnote}{+1}\footnotetext{Vgl. \cite[S.39ff]{Corsten2006}}
\addtocounter{footnote}{+1}\footnotetext{Vgl. \cite[S.104ff]{mirschel2007messung}}	
\newpage\addtocounter{footnote}{+1}\footnotetext{Vgl. \cite{trigeorgis}, \cite{bengtsson2002} und \cite{Damisch2002}}	
\addtocounter{footnote}{+1}\footnotetext{Tabelle und s�mtliche Verweise entnommen aus \cite[S.230-233, Abbildung 2]{Bellmann2009}}	
	
	
  \end{appendix}
  
\end{document}

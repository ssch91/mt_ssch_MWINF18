
%section als 1 statt 1.1, damit Chapter nicht ben�tigt wird
\renewcommand{\thesection}{\arabic{section}}
%auch ab Ebene 3 bis 5 nummerieren
\setcounter{secnumdepth}{5}
%Stil der Kopf- und Fu�zeilen
\newpagestyle{mainmatter}{%
  \setheadrule{1pt}
  \sethead{\thesection ~\sectiontitle}{}{\thesubsection ~\subsectiontitle}%
  \setfoot{}{\thepage}{}
}
\newpagestyle{rest}{%
  \sethead{}{}{}
  \setfoot{}{\thepage}{}
}
%damit auch der Name der Subsection erkannt wird
\settitlemarks{section,subsection,subsubsection}
%anwenden des Stils



%Bilder in Unterordner suchen
\graphicspath{{image/}}
%Zeilenabstand
\setstretch{1.5}
%Bib-Ressourcen einbinden
\addbibresource{bibliography.bib}
%Tiefe des Inhaltsverzeichnisses
\setcounter{tocdepth}{4}
%neue Seite nach Kapitelende
\newcommand{\sectionbreak}{\clearpage}


%URLs im Literaturverzeichnis umbrechen
\setcounter{biburllcpenalty}{7000}
\setcounter{biburlucpenalty}{8000}

%Fu�zeilen nicht auf n�chster Seite fortf�hren
\interfootnotelinepenalty=10000

%Kurzreferenz bei zweiter Benutzung ohne "wie Anm. x"
\renewcommand\citenamepunct{\addcomma\space}
\renewbibmacro*{cite:seenote}{}

%Vertikale Zentrierung von Zelleninhalten
\renewcommand\tabularxcolumn[1]{m{#1}}
\newcolumntype{C}{>{\Centering\arraybackslash}X} % centered "X" column




%Seitennummerierugen
\newcounter{savepage}
\def\frontmatter{
	\pagestyle{rest}
    \pagenumbering{Roman}
    \setcounter{page}{1}
}

\def\mainmatter{
	\pagestyle{mainmatter}
	\setcounter{savepage}{\number\value{page}}
	\newpage
	\pagenumbering{arabic}
}

\def\backmatter{
	\pagestyle{rest}
  	\newpage
	\pagenumbering{Roman}
	\setcounter{page}{\value{savepage}}
}
%section als 1 statt 1.1, damit Chapter nicht ben�tigt wird
\renewcommand{\thesection}{\arabic{section}}

%auch ab Ebene 3 bis 5 nummerieren
\setcounter{secnumdepth}{5}

%Stil der Kopf- und Fu�zeilen
\newpagestyle{mainmatter}{%
  \setheadrule{1pt}
  \sethead{\thesection ~\sectiontitle}{}{\thesubsection ~\subsectiontitle}%
  \setfoot{}{\thepage}{}
}
\newpagestyle{rest}{%
  \sethead{}{}{}
  \setfoot{}{\thepage}{}
}
%damit auch der Name der Subsection erkannt wird
\settitlemarks{section,subsection,subsubsection}

%Bilder in Unterordner suchen
\graphicspath{{image/}}

%Zeilenabstand
\setstretch{1.3}

%Bib-Ressourcen einbinden
\addbibresource{bibliography.bib}

%Tiefe des Inhaltsverzeichnisses
\setcounter{tocdepth}{5}

%neue Seite nach Kapitelende
\newcommand{\sectionbreak}{\clearpage}


%URLs im Literaturverzeichnis umbrechen
\setcounter{biburllcpenalty}{7000}
\setcounter{biburlucpenalty}{8000}

%Fu�zeilen nicht auf n�chster Seite fortf�hren
\interfootnotelinepenalty=10000

%Kurzreferenz bei zweiter Benutzung ohne "wie Anm. x"
\renewcommand\citenamepunct{\addcomma\space}
\renewbibmacro*{cite:seenote}{}

%Vertikale Zentrierung von Zelleninhalten
\renewcommand\tabularxcolumn[1]{m{#1}}
\newcolumntype{C}{>{\Centering\arraybackslash}X} % centered "X" column
\newcolumntype{L}{>{\raggedleft\arraybackslash}X} % centered "X" column
\newcolumntype{R}{>{\raggedright\arraybackslash}X} % centered "X" column
\newcolumntype{M}[1]{>{\raggedright \arraybackslash\hspace{0pt}}m{#1}} 
\newcolumntype{Y}[1]{>{\raggedleft}m{#1}}
\newcolumntype{P}[1]{>{\centering\arraybackslash}p{#1}}


%Fu�noten kleiner machen
\renewcommand{\footnotesize}{\scriptsize}

%Enumerates in Enumerates erm�glichen also 1., 2., 2.1., 2.2., 3., etc.
%\renewcommand{\labelenumii}{\theenumii}
%\renewcommand{\theenumii}{\theenumi.\arabic{enumii}.}

%mit \tocless sections erzeugen, die nicht im toc auftauchen
\newcommand{\nocontentsline}[3]{}
\newcommand{\tocless}[2]{\bgroup\let\addcontentsline=\nocontentsline#1{#2}\egroup}


\tikzset{
  basic/.style  = {draw, text width=2.5cm, drop shadow, font=\sffamily, rectangle},
  root/.style   = {basic, rounded corners=2pt, thin, align=center,
                   fill=green!30},
  level 2/.style = {basic, rounded corners=6pt, thin,align=center, fill=green!60,
                   text width=8em},
  level 3/.style = {basic, thin, align=left, fill=pink!60, text width=6.5em}
}

%subparagraph no indent
\makeatletter
\renewcommand\subparagraph{%
 \@startsection {subparagraph}{5}{\z@ }{3.25ex \@plus 1ex
 \@minus .2ex}{-1em}{\normalfont \normalsize \bfseries }}%
\makeatother

%engeres enumerate
\newenvironment{tightenumerate}{
\begin{enumerate}{
\setlength{\topsep}{-1ex}
\setlength{\itemsep}{0pt}
\setlength{\leftmargin}{4ex}
}
}{
\end{enumerate}
\vspace{1ex}
}

%engeres itemize
\newenvironment{tightlist}{
\begin{list}{\textbullet}{
\setlength{\topsep}{-1ex}
\setlength{\itemsep}{-1ex}
\setlength{\leftmargin}{4ex}
}
}{
\end{list}
\vspace{1ex}
}




%
%%Erstellen eines Anhangsverzeichnises und Inhaltsverzeichnis aus einem
%\makeatletter% --> De-TeX-FAQ
%\newcommand*{\maintoc}{% Hauptinhaltsverzeichnis
%  \begingroup
%    \@fileswfalse% kein neues Verzeichnis �ffnen
%    \renewcommand*{\appendixattoc}{% Trennanweisung im Inhaltsverzeichnis
%      \value{tocdepth}=-10000 % lokal tocdepth auf sehr kleinen Wert setzen
%    }%
%    \tableofcontents% Verzeichnis ausgeben
%  \endgroup
%}
%\newcommand*{\appendixtoc}{% Anhangsinhaltsverzeichnis
%  \begingroup
%    \edef\@alltocdepth{\the\value{tocdepth}}% tocdepth merken
%    \setcounter{tocdepth}{-10000}% Keine Verzeichniseintr�ge
%    \renewcommand*{\contentsname}{% Verzeichnisname �ndern
%      Anhangsverzeichnis}%
%    \renewcommand*{\appendixattoc}{% Trennanweisung im Inhaltsverzeichnis
%      \setcounter{tocdepth}{\@alltocdepth}% tocdepth wiederherstellen
%    }%
%    \tableofcontents% Verzeichnis ausgeben
%    \setcounter{tocdepth}{\@alltocdepth}% tocdepth wiederherstellen
%  \endgroup
%}
%\newcommand*{\appendixattoc}
%\g@addto@macro\appendix{% \appendix erweitern
%  \if@openright\cleardoublepage\else\clearpage\fi% Neue Seite
%  \phantomsection
%  \addcontentsline{toc}{chapter}{\appendixname}% Eintrag ins Hauptverzeichnis
%  \addtocontents{toc}{\protect\appendixattoc}% Trennanweisung in die toc-Datei
%}
%
%\g@addto@macro\appendix{%
%  %\pagenumbering{Roman}%
%  \addtocontents{toc}{\protect\renewcommand*{\protect\@pnumwidth}{3.5em}}%Seitenzahlen im Anhangverzeichnis nicht �ber Seitenrand hinaus. Problem entstand durch die langen r�mischen Zahlen. Siehe auch: http://www.komascript.de/node/608
%}
%\makeatother

%-----------------------------
\makeatletter% --> De-TeX-FAQ
\newcommand*{\maintoc}{% Hauptinhaltsverzeichnis
  \begingroup
    \@fileswfalse% kein neues Verzeichnis �ffnen
    \renewcommand*{\appendixattoc}{% Trennanweisung im Inhaltsverzeichnis
      \value{tocdepth}=-10000 % lokal tocdepth auf sehr kleinen Wert setzen
    }%
    \tableofcontents% Verzeichnis ausgeben
  \endgroup
}
\newcommand*{\appendixtoc}{% Anhangsinhaltsverzeichnis
  \begingroup
    \edef\@alltocdepth{\the\value{tocdepth}}% tocdepth merken
    \setcounter{tocdepth}{0}% Keine Verzeichniseintr�ge
    \renewcommand*{\contentsname}{% Verzeichnisname �ndern
     Anhangsverzeichnis}%
    \renewcommand*{\appendixattoc}{% Trennanweisung im Inhaltsverzeichnis
      \setcounter{tocdepth}{\@alltocdepth}% tocdepth wiederherstellen
    }%
    \tableofcontents% Verzeichnis ausgeben
    \setcounter{tocdepth}{\@alltocdepth}% tocdepth wiederherstellen
  \endgroup
}
\newcommand*{\appendixattoc}{% Trennanweisung im Inhaltsverzeichnis
}
\g@addto@macro\appendix{% \appendix erweitern
  \if@openright\cleardoublepage\else\clearpage\fi% Neue Seite
  \addcontentsline{toc}{section}{\appendixname}% Eintrag ins Hauptverzeichnis
  \addtocontents{toc}{\protect\appendixattoc}% Trennanweisung in die toc-Datei
}
\makeatother









%Seitennummerierungen
\newcounter{savepage}
\def\frontmatter{
	\pagestyle{rest}
    \pagenumbering{Roman}
    \setcounter{page}{1}
}

\def\mainmatter{
	\pagestyle{mainmatter}
	\setcounter{savepage}{\number\value{page}}
	\newpage
	\pagenumbering{arabic}
}

\def\backmatter{
	\pagestyle{rest}
  	\newpage
	\pagenumbering{Roman}
	\setcounter{page}{\value{savepage}}
}